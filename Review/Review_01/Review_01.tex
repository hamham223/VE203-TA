\documentclass[xcolor=table]{beamer}
\renewcommand\thesection{\arabic{section}}
\newcommand{\myfont}{\rmfamily\normalsize\upshape\mdseries}
\newcommand{\degree}{^\circ}
\title{\sffamily Review I(Slides 4 - 71)}
\subtitle{\textbf{Sets \& Logics}\\ Does the barber ever shave himself?}
\institute[UM-SJTU JI]{University of Michigan-Shanghai Jiao Tong University Joint Institute}
\author{HamHam}
\usepackage{graphicx}
\usepackage{picinpar}
\usepackage{indentfirst}
\usepackage{chemformula}
\usepackage{geometry}
\usepackage{subfigure}
\usepackage{appendix}
\usepackage{amsfonts,amsmath,amssymb}
\usepackage{bm,bbm}
\usepackage{enumerate}
\usepackage{float}
\usepackage{geometry}
\usepackage{latexsym}
\usepackage{listings}
\usepackage{multicol,multirow,multido}
\usepackage{tabularx}
\usepackage{ulem}
\usepackage{tikz}
\usepackage{color,xcolor}
\usepackage{cite}
\usepackage{setspace}
\usepackage{hyperref}
\usepackage{textpos}
\usepackage{booktabs}
\usepackage{diagbox}
\usepackage{pgf}
\usepackage{JI_MathCourse_Notations}
\usepackage{forest}
\usepackage{prooftrees}
\usepackage{mathrsfs}
\usepackage{pifont}
\usetheme[dove]{Boadilla}
\usecolortheme{dolphin}
%\pgfdeclaremask{figmask}{or_circuit.jpg}
%\pgfdeclareimage[mask=figmask,width=0.6\textwidth]{or_circuit}{or_circuit_1.png}
\useoutertheme{miniframes}
\begin{document}
    \usebackgroundtemplate{\tikz\node[opacity=0.25]{
    \includegraphics[width=\paperwidth,
    height=\paperheight]{hamster.jpg}
    };}
\begin{titlepage}
    \begin{center}
        VE203 - Discrete Mathmatics 
    \end{center}
\end{titlepage}
\myfont

\section{Abouts}
\begin{frame}
    \frametitle{Abouts}

    \begin{itemize}
        \item VE203 TA $\times 2$ / TA Mentor / intel BigDL intern
        \item \url{https://github.com/hamham223}
        \item RC is estimated to be about 1 hour
        \item OH might be at early eight, hahaha!
    \end{itemize}
    
    \vspace{2em}

    \begin{block}{Question before we start}
        \hspace{2em}
        Ask yourself: Why did you select the course, discrete math this semester? What do you expect to gain? 
    \end{block}

\end{frame}

\begin{frame}
    \frametitle{Recommended Books \& Websites}
    \begin{itemize}
        \item Kenneth, H.Rosen. Translated by Xu Liutong etc. \itshape Discrete Mathematics amd Its Applications\myfont, 
        Eightth Edition, Chinese Abridgement. China Machine Press, 2019 print.
        \item E. Knuth, Donald. Translated by Su Daolin. \textit{The art of Computer Programming}, third edition.
        Beijing: National Defense Industry Press, 2007.6 print.
        \item \url{www.mhhe.com/rosen}
        \item \url{https://leetcode.com/}
    \end{itemize}

\end{frame}

\section{Sets}
\begin{frame}
    \frametitle{Set Operations}
    \begin{itemize}
        \item union \& intersection
        \item set difference
        \item symmetric difference
        \item power set
        \item cardinality
        \item cartesian product
        \item Venn Diagram v.s. Euler Diagram 
    \end{itemize}
    \vv
    \begin{block}{Question}
        How to compare the cardinality of two infinite sets?
    \end{block}
\end{frame}
\begin{frame}
    \frametitle{Exercise}
    1. Let $A, B, M$ be three sets and $A, B \siq M$. Show that
    \begin{enumerate}
        \item $A \cup(B \cap C)=(A \cup B) \cap(A \cup C)$
        \item $A-(B \cup C)=(A-B) \cap(A-C)$
        \item $(A - B) \cup (B - A) = (A \cup B) - (A \cap B)$
    \end{enumerate}
    \vv
    \red{It's too boring! Let's just do the second one!}
\end{frame}


\section{Logicals}
\begin{frame}
    \frametitle{Logical Operations}
    Five operations you need to be vary familiar with:
    $$\neg ~~\wedge ~~\vee ~~  \rarrow ~~  \leftrightarrow$$

    What about these?
    $$\vdash ~~ \models ~~ \Rightarrow ~~ \Leftrightarrow ~~ \equiv$$

    Please refer to:
    \begin{itemize}
        \item \url{https://www.zhihu.com/question/21191299}
        \item \url{https://www.reddit.com/r/logic/comments/3nftuh/what_is_the_difference_between_and}
    \end{itemize}
    \begin{block}{Strategy}
        \begin{itemize}
            \item Change $p \rarrow q$ to $\neg p \vee q$
            \item Truth Table
            \item Be careful! $\LRarrow$ or $\lrarrow$?
        \end{itemize}
    \end{block}
\end{frame}
\begin{frame}
    \frametitle{Exercise}
    2. Prove that
    \begin{itemize}
        \item $P \rarrow (Q \rarrow R) \LRarrow (P \wedge Q) \rarrow R$
        \item $((P \vee Q) \wedge \neg Q) \rarrow P$ is a \textbf{\textcolor[rgb]{0,0.6,0.2}{tautology}}
        \item $(A\rarrow(B\rarrow C))\rarrow (B\rarrow(A\rarrow C))$ is a tautology
    \end{itemize}
    \vv \vv
    \red{Q: Simplifying experssions v.s. Truth tree? \\ \vv
    A: Children make choices, let's try both!}
\end{frame}

\begin{frame}[fragile]{Truth Tree Example}
\begin{minipage}[b]{0.43\linewidth}
    Steps to go:
    \begin{itemize}
        \item Setup counter-examples
        \item Apply rules: stcking first!!
        \item Check for contradictions, close contradicting branches
        \item Read the answer:
            \begin{itemize}
                \item All close: Valid
                \item Even one open: Invalid
            \end{itemize}
    \end{itemize}
    \begin{block}{Question}
        \hspace{2em}
        Is the truth tree unique?
    \end{block}
\end{minipage}
\begin{minipage}[b]{0.42\linewidth}
\begin{forest}
[$A \rightarrow \(B \rightarrow C \)$, checked, for tree={s sep=2mm}
    [$\lnot (B \rightarrow \(A \rightarrow C \))$, checked
        [$\lnot A$
            [$B$
                [$\lnot \(A \rightarrow C\)$, checked
                    [$A$]
                ]
            ]
        ]
        [$B \rightarrow C$, checked
            [$B$
                [$\lnot \(A \rightarrow C \)$, checked
                    [$\lnot B$]
                    [$C$
                        [$\lnot C$]
                    ]
                ]
            ]
        ]
    ]
]
\end{forest}
\end{minipage}
\end{frame}

\begin{frame}
    \frametitle{Introduction to boolean algebra}
    \vv
    If we regard $\vee$ as $+$, $\wedge$ as $\cdot$, then the equation 
        $$A\wedge(B\vee C\,)=(A\wedge B\,)\vee(A\wedge C\,)$$
    is just the distributivity law:
        $$A \cdot (B+C\,) = (A\cdot B) + (A \cdot C\,)$$
    \vv
    \textbf{\textit{\red{Do It Yourself:}}}\\
    Check whether the axiom P1 -- P9 for rational numbers also hold for such operations.
    \begin{block}{Note}
        \hspace{2em}
        This is not VE270! Don't mix up! It's always important to keep notations consistent!
    \end{block}
\end{frame}
\begin{frame}
    \frametitle{Properties}
    \hh
    We denote $\neg A$ as $\conj{A}$.
    And, 1 means true ($\top$), 0 means false ($\bot$). 
    We have the following: \vv
    \begin{itemize}
        \item $A \cdot 1 = A$
        \item $A + 1 = 1$
        \item $A + \conj{A} = 1$
        \item $A \cdot \conj{A} = 0$
        \item $\conj{\conj{A}}=A$
        \item $\conj{A+B}=\conj{A}\cdot \conj{B}$
        \item $\conj{A\cdot B}=\conj{A}+ \conj{B}$
        \item $A+AB=A$
        \item $\dots$
    \end{itemize}

\end{frame}

\begin{frame}
    \frametitle{\sout{DNF \& CNF} (Deleted)}
    \yellow{Definition:}
    \begin{itemize}
        \item  CNF: \textbf{product of sums} or an \textbf{AND of ORs}
        \item  DNF: \textbf{sum of products} or an \textbf{OR of ANDS}
    \end{itemize}
    \textcolor[rgb]{0.95,0.75,0.95}{Examples:}
    \begin{itemize}
        \item $(\neg p \vee  q \vee r) \wedge (\neg q \vee \neg r) \wedge (r)$
        \item $(\neg p \wedge q \wedge r) \vee (\neg q \wedge \neg r) $        
    \end{itemize}
    \begin{block}{Question}
        What's the DNF/CNF for a tautology?
    \end{block}
\end{frame}
\begin{frame}
    \frametitle{\sout{Exercise}}
    3. Suppose that a truth table in \textbf{n} propositional variables 
    is specified. Show that a compound proposition with 
    this truth table can be written to a well-determined DNF. \\(Take from Vv186 Assignment Exercise 1.4)
    \begin{table}[H]
        \begin{tabular}{ccc}
        $A$& $B$ &$f~(A,B)$  \\
        \toprule
        $T$ &$T$  & $T$ \\
        $T$ &$F$  & $F$ \\
        $F$ &$T$  & $T$ \\
        $F$ &$F$  & $F$ \\
        \bottomrule
        \end{tabular}
    \end{table}
    \vv
    $$f~(A,B)=(A \wedge B)\vee (\neg A \wedge B)$$
\end{frame}
\begin{frame}{Predicates}
    \par A function $P: X \to \{\top, \perp \}$ is called a \green{\textbf{predicate}} on its domain $X$. 
    \\\vv
    \par It is a declarative sentence involving variables, when the variables a substituted with appropriate individuals we obtain a \green{\textbf{proposition}}.
    \\\vv
    \begin{itemize}
        \item \textbf{Predicate:} $P(x): x>1$;
        \item \textbf{Proposition:} $P(0): 0>1$ (false); $P(2): 2>1$ (true).
    \end{itemize}

    \vv 
    \begin{block}{Question}
        \hh
        Statement = \_\_\_\_\_\_\_ + \_\_\_\_\_\_\_ 
    \end{block}
\end{frame}

\begin{frame}
    \frametitle{Logical Quantifiers}
    Why is:
    $$\exists y\forall xP(x, y) \Rightarrow  \forall x \exists yP(x, y)$$
    
    \begin{block}{Proof}
        Let $y=n$, we can now phrase ``for all $x$, $P(x,n)$ is true". It follows that, ``for all $x$, there exists an $y$ (which is actually $n$) such that $P(x,n)$ is true."
    \end{block}
    
    \vspace{1em}
    Why is:
    $$\forall x \exists y P(x, y) \not\Rightarrow  \exists y\forall x P(x, y)$$

    \begin{block}{Proof}
    A counter example is enough, let $P(x,y): x-y=1$. 
    \end{block}
\end{frame}
\begin{frame}
    \frametitle{A proof from hamster¿}
    Hamster wants to dis-prove $$\forall x \exists y P(x, y) \Rightarrow  \exists y\forall x P(x, y)$$

    so it uses truth tree. (forgive its bad \LaTeX~ skill TT.)
    \begin{itemize}
        \item[1.]$\forall x \exists y P(x, y)$, \blue{premise}
        \item[2.]$\neg \exists y\forall x P(x, y)$, \blue{$\neg$ consequent}
        \item[3.]$\forall y \exists x \neg P(x,y)$ \blue{(2)}
        \item[4.]$\forall x P(x,b) \backslash b$ \blue{(1)}
        \item[5.]$\forall y \neg P(a,y) \backslash a$ \blue{(3)}
        \item[6.]$P(a,b)$ \blue{(4)}
        \item[7.]$\neg P(a,b)$ \blue{(5)}
        \item[8.]\ding{55}
    \end{itemize}
\end{frame}
\section{End}

\begin{frame}
    \frametitle{Reference}
    \begin{itemize}
        \item Pictures from Dr. Horst Hohberger.
        \item Exercises from 2020-Ve203 Assignment2.
        \item Exercises from 2021-Vv186 Assignment1.
        \item Exercises from 2019--Vv186 TA-Zhang Leyang.
        \item Contents from 2020 Fall Ve203 TA-Peng Chengjun.
        \item Exercises from 2021 Fall Ve203 TA-Zhao Jiayuan.
        \item Exercises from my 2021--Vv186 Mid1 RC.
        \item Kenneth, H.Rosen. Translated by Xu Liutong etc. \itshape Discrete Mathematics amd Its Applications\myfont, 
                 Eightth Edition, Chinese Abridgement. China Machine Press, 2019 print.
    \end{itemize}
\end{frame}

\begin{frame}
    \centering
    \Huge{$\mathcal{THANKS}$!}
\end{frame}

\end{document}