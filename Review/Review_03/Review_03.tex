\documentclass{beamer}
\newcommand{\myfont}{\rmfamily\normalsize\upshape\mdseries}
\newcommand{\degree}{^\circ}
\title{\sffamily Review III(Slides 127 - 164)}
\subtitle{\textbf{Induction \& Relations }\\}
\institute[UM-SJTU JI]{University of Michigan-Shanghai Jiao Tong University Joint Institute}
\author{HamHam}
\usepackage{graphicx}
\usepackage{picinpar}
\usepackage{indentfirst}
\usepackage{chemformula}
\usepackage{geometry}
\usepackage{subfigure}
\usepackage{appendix}
\usepackage{amsfonts,amsmath,amssymb}
\usepackage{bm,bbm}
\usepackage{enumerate}
\usepackage{float}
\usepackage{geometry}
\usepackage{latexsym}
\usepackage{listings}
\usepackage{multicol,multirow,multido}
\usepackage{tabularx}
\usepackage{ulem}
\usepackage{tikz}
\usepackage{color,xcolor}
\usepackage{cite}
\usepackage{setspace}
\usepackage{hyperref}
\usepackage{textpos}
\usepackage{booktabs}
\usepackage{diagbox}
\usepackage{listings}
\usepackage{JI_MathCourse_Notations}
\usetheme[dove]{Boadilla}
\usecolortheme{dolphin}
%\pgfdeclaremask{figmask}{or_circuit.jpg}
%\pgfdeclareimage[mask=figmask,width=0.6\textwidth]{or_circuit}{or_circuit_1.png}
\useoutertheme{miniframes}
\begin{document}
    \usebackgroundtemplate{\tikz\node[opacity=0.25]{
    \includegraphics[width=\paperwidth,
    height=\paperheight]{hamster.jpg}
    };}
\begin{titlepage}
    \begin{center}
        VE203 - Discrete Mathmatics 
    \end{center}
\end{titlepage}
\myfont
\section{Induction}
\begin{frame}
    \frametitle{Updates}

    \begin{itemize}
        \item Next RC(6.12)'s location TBD
        \item Please take the Midterm RC Survey on canvas
        \item RC on 6.19 is canceled due to the Midterm
        \item We have one lecture on 6.25 Sunday TT
        \item RC on 6.26 is canceled  due to Dragon Boat Festival
    \end{itemize}

\end{frame}
\begin{frame}
    \frametitle{Induction ``Paradox" (Slides 93)}

    For any $n \in \mathbb{N} \cut \{0\}$, there exist integers $a_n$ and $b_n$ s.t.
    $$
    \(\frac{1+\sqrt{5}}{2}\)^n = \frac{a_n+b_n \sqrt{5}}{2}
    $$

    \begin{enumerate}
        \item Base case: $n=1$, clear by taking $a_1=b_1=1$
        \item Inductive case: assume the IH for $n\geq 1$,
        \begin{equation*}
        \begin{aligned}
            \(\frac{1+\sqrt{5}}{2}\)^{n+1} & = \frac{a_n + b_n \sqrt{5}}{2} \cdot \frac{1+\sqrt{5}}{2}\\
            & = \frac{(a_n+5b_n)/2 + ((a_n+b_n)/2) \sqrt{5}}{2}
        \end{aligned}
        \end{equation*}
    \end{enumerate}
    \begin{block}{Strategy: Enhance the original proposition}
        \hh $a_n + 5b_n$ and $a_n+b_n$ must be both even! Namely, $a_n$ and $b_n$ are both even or 
        both odd. Denoted as $2 \mid (a_n - b_n)$ or $a_n - b_n \equiv 1 \mod 2$.
    \end{block}
\end{frame}
\begin{frame}
    \frametitle{Structural Induction}
    \hh Let $B$ be a set and let $C_1, ..., C_n$ be construction rules. 
    Let $B$ be recursively defined to be the $\subseteq$-least set such that 
    $B \subseteq A$ and $A$ is closed under the rules $C_1, ..., C_n$. 
    Let $P(x)$ be a property. If
    \begin{enumerate}
        \item for all $b \in B$, $P(b)$ holds
        \item for all $a_1, ..., a_m$ and $c$ and $1\leq i\leq n$, 
        if $P(a_1), ..., P(a_m)$ all hold and $c$ is obtained from 
        $a_1, ..., a_m$ by a single application of the rule $C_i$, 
        then $P(c)$ holds
    \end{enumerate}
    Then $P(x)$ holds for every element in A.
\end{frame}
\begin{frame}
    \frametitle{Structural Induction Tutorial}
    \begin{block}{MIT 6.042J Example}
        \hh 
        Every $s$ in $M$ has the same number of ]'s and ['s.
    \end{block}
    Video available at:
    \begin{itemize}
        \item \url{https://www.youtube.com/watch?v=VWIDwHCGJDQ}
        \item \url{https://www.bilibili.com/video/BV1n64y1i777/}
    \end{itemize}

\end{frame}
\begin{frame}
    \frametitle{Exercise}
    Taken from Ve203 FA 2020 assignment 2: 
    \\ \vs{2em}
    1. Let $S \subset \bN$ be defined by
    \begin{itemize}
        \item $(0,0)\in S$
        \item $(a,b)\in S \Rarrow \left(\left(\left(a+2,b+3\right)\in S\,\right)\wedge \left(\left(a+3,b+2\right)\in S\,\right)\right)$ 
    \end{itemize}
    Use structural induction to show that $(a, b) \in S$ implies $5 \mid (a + b)$.
    \\\textbf{(3 Marks)}
\end{frame}
\section{Relations}
\begin{frame}
    \frametitle{Relation}
    \hh A subset $R \subset A \times B$ is called a (binary) relation from $A$ to $B$. 
    If $A = B$,	we say that $R$ is a \textbf{\textcolor[rgb]{0,0.6,0.2}{relation}} on $A$.\\ 
    \vv
    \begin{block}{Quick Check}
        \begin{itemize}
            \item domain$(R)=\{x \mid \exists y(x R y)\}$
            \item range$(R)=\{y \mid \exists x(x R y)\}$
            \item $R = \varnothing $: the empty relation
            \item $A = B$: identity relation
            \item The relation $A \times B$ itself? 
        \end{itemize}
    \end{block}

\end{frame}
\begin{frame}
    \frametitle{Functions}
    \hh A \textbf{\textcolor[rgb]{0,0.6,0.2}{function}} is a relation $F$ such that $$\forall x \in \operatorname{dom} F(\exists ! y(x F y)).$$
    
    \begin{block}{Quick Check}
        \begin{itemize}
            \item  For a function $F$ and a point $x \in \operatorname{dom} (F)$, the unique $y$ such that $xFy$ is called the \blue{value} of $F$ at $x$ and is denoted $F(y)$. 
            \item Given function $F : A \to B$, then $\forall x, y \in A(x = y \Rightarrow F(x) = F(y))$.
            \item Partial Function/ Total Tunction.
        \end{itemize}
    \end{block}

\end{frame}
\begin{frame}
    \frametitle{Operations on Functions}
    \par For \red{arbitrary} sets A, relations F, and functions G,
	\vv
    \begin{itemize}
		\item \blue{Inverse:} $F^T = F^{-1} = \{(y, x) \mid xFy\}$.
        \item \blue{Composition:} $F \circ G = \{(x, z) \mid \exists y \in A(xFy \wedge yGz)\}$.
		\item \blue{Restriction:} $F \mid A = \{(x, y) \in F \mid x \in A \}$.
		\item \blue{Image:} $F(A) = \operatorname{ran}(F \mid A) = \{y \mid (\exists x \in A) xFy \}$. 
	\end{itemize}
    \vv
    If $F$ is a function, then $F(A) = \{F(x) \mid x \in A\}$.
\end{frame}
\begin{frame}
    \frametitle{Exercise}
    2. Let $A: \bR^3 \to \bR^3 $ be given by
    \begin{small}
        \begin{equation*}
            A=\maThree{1}{2}{3}{2}{1}{2}{1}{1}{1}.
        \end{equation*}
    \end{small}
    Let 
    \begin{small}
        \begin{equation*}
            U=\text{span} {\left \{\begin{pmatrix}1\\1\\0\\\end{pmatrix}\right \} }
        \end{equation*} 
    \end{small}
    \begin{block}{Question}
        \begin{itemize}
            \item[-]What's the restriction of $A$ to $U\,$? 
            \item[-]What's the image of $U$ under $A$?
        \end{itemize}
    \end{block}
    (Taken from vv286 lecture slides)
\end{frame}
\begin{frame}
    \frametitle{*-jectivity}
    \parbox{\textwidth}{
		\par Given a function $F : A \to B$, with $\operatorname{dom} F = A$ and $ \operatorname{ran}(F) \subset B$, then
		\begin{itemize}
			\item[-] $F$ is \blue{injective or \red{one-to-one}} if $\forall x, y \in A(F(x) = F(y) \Rightarrow x = y)$;
			\item[-] $F$ is \blue{surjective or \red{onto}} if ran$(F) = B$;
			\item[-] $F$ is \yellow{bijective} if it is both injective and surjective.
		\end{itemize}
		\vs{0.3em}
		\par Given a function $F : A \to B$, $A \neq \varnothing$, then
		\begin{itemize}
			\item[-] There exists a function $G : B \to A$ (a \blue{``left inverse''}) such that
			$G \circ F = id_A \Leftrightarrow F$ is one-to-one;
			\item[-] There exists a function $G : B \to A$ (a \blue{``right inverse''}) such that
			$F \circ G = id_B \Leftrightarrow F$ is onto.
		\end{itemize}
        \begin{block}{Let $f : A \to B, g : B \to C$,}
            \begin{itemize}
                \item[-] If $g \circ f$ is injective, then $f$ is injective.
                \item[-] If $g \circ f$ is surjective, then $g$ is surjective
            \end{itemize}
        \end{block}
		%\par 
	}
\end{frame}
\begin{frame}
    \frametitle{Exercise}
    3. Recall that $\mathbb{Z}$ denotes the set of integers, $\mathbb{Z}^+$ the set of positive integers, and $\mathbb{Q}$ the set of rational numbers. 
    Define a function: 
    \begin{equation*}
    f: \mathbb{Z} \times \mathbb{Z}^+ \to \mathbb{Q},~~~~~~~~~~
    f~(p,q) = \frac{p}{q}.
    \end{equation*}
    \begin{enumerate}
    	\item Is $f$ an injection? Why?
    	\item Is $f$ a surjection? Why?
    	\item Is $f$ a bijection? Why?
    \end{enumerate}
\end{frame}
\begin{frame}
    \frametitle{Definition}
    \parbox{\textwidth}{
		\par A (binary) relation $R$ on $A$, \textit{i.e.,} $R \subset A \times A$, is
		\begin{itemize}
			\item[-] \textbf{ref\mbox{l}exive} if $aRa \Rightarrow \top$.
			\item[-] \textbf{symmetric} if $aRb \Leftrightarrow bRa$.
			\item[-] \textbf{transitive} if $aRb \wedge bRc \Rightarrow aRc$.
			\item[-] \textbf{anti-symmetric} if $aRb \wedge bRa \Rightarrow a = b$.
			\item[-] asymmetric if $aRb \wedge bRa \Rightarrow \perp$.
			\item[-] total if $aRb \vee bRa \Rightarrow \top$.
		\end{itemize}
		\par \phantom{ji}
		\par \textbf{(Non-strict) Partial order}: reflexive, antisymmetric, and transitive.
		\par \textbf{Equivalence relation}: reflexive, symmetric, and transitive.
        \par \textbf{Total order:} Partial order + total.
    }
\end{frame}
\begin{frame}
    \frametitle{Exercise}
    4. Recall that $\mathbb{R}$ denotes the set of real numbers, while $\mathbb{Z}$ denotes the set of integers. Define a relation $\sim$ on $\mathbb{R}$ by
    \begin{equation*}
        x \sim y \Leftrightarrow x - y \in \mathbb{Z}
    \end{equation*}
    for any $x, y \in \mathbb{R}$. Prove that $\sim$ is an equivalence relation.
\end{frame}
\begin{frame}
    \frametitle{Equivalence Class}
    Given an equivalence relation $R$ on $A$,
    \begin{itemize}
        \item \blue{Equivalence class containing $x$} $$\left[ x\right]_R = \{t \in A \mid xRt\}.$$
        \item This is also a partition for $A$.
        \item For $x , y \in A$, $$\left[ x \right]_R  = \left[ y \right]_R \Leftrightarrow xRy.$$
        \item Quotient set is given by $$A/R = \{[x]R \mid x \in A\}.$$
    \end{itemize}
\end{frame}

\section{End}

\begin{frame}
    \frametitle{Reference}

    \begin{itemize}
        \item Examples from Vv286 Lecture Slides.
        \item Exercises from 2021-Fall-Ve203 TA Zhao Jiayuan
    \end{itemize}

\end{frame}

\begin{frame}
    \centering
    \Huge{$\mathcal{THANKS}$!}
\end{frame}

\end{document}