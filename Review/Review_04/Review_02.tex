\documentclass{beamer}
\newcommand{\myfont}{\rmfamily\normalsize\upshape\mdseries}
\newcommand{\degree}{^\circ}
\title{\sffamily Review II(Slides 103 - 159)}
\subtitle{\textbf{Relations }\\}
\institute[UM-SJTU JI]{University of Michigan-Shanghai Jiao Tong University Joint Institute}
\author{HamHam}
\usepackage{graphicx}
\usepackage{picinpar}
\usepackage{indentfirst}
\usepackage{chemformula}
\usepackage{geometry}
\usepackage{subfigure}
\usepackage{appendix}
\usepackage{amsfonts,amsmath,amssymb}
\usepackage{bm,bbm}
\usepackage{enumerate}
\usepackage{float}
\usepackage{geometry}
\usepackage{latexsym}
\usepackage{listings}
\usepackage{multicol,multirow,multido}
\usepackage{tabularx}
\usepackage{ulem}
\usepackage{tikz}
\usepackage{color,xcolor}
\usepackage{cite}
\usepackage{setspace}
\usepackage{hyperref}
\usepackage{textpos}
\usepackage{booktabs}
\usepackage{diagbox}
\usepackage{listings}
%%%%%%%%%%%%%%%%%%%%%%%%%%%%%%%%%%%%%%%%%%%%%%%%%%%%%%%%%%%%%%%%%%%%%%%%%%%%%
\iffalse 
Copyright 2021 by Leyang Zhang, Yinchen Ni, Yuxiang Chen, Yue Huang

You should not spread this file without the permission from every one of 
    Leyang Zhang, Yinchen Ni and Yuxiang Chen 
\fi
%%%%%%%%%%%%%%%%%%%%%%%%%%%%%%%%%%%%%%%%%%%%%%%%%%%%%%%%%%%%%%%%%%%%%%%%%%%%%

\ProvidesPackage{JI_MathCourse_Notations}[2021/12/19]

\RequirePackage{amsmath}
\RequirePackage{amsthm}
\RequirePackage{amssymb}
\RequirePackage{bm}
\RequirePackage{bbm}
\RequirePackage{color}


%Typesetting

%colors 
\newcommand{\blue}[1]{\textcolor{blue}{#1}}                         %blue
\newcommand{\cha}[1]{\textcolor[rgb]{0.95,0.75,0.9}{#1}}            %cha-type color 
\newcommand{\gray}[1]{\textcolor{gray}{#1}}                         %grey
\newcommand{\pink}[1]{\textcolor{pink}{#1}}                         %pink
\newcommand{\red}[1]{\textcolor[rgb]{0.75,0,0}{#1}}                 %red
\newcommand{\yellow}[1]{\textcolor{orange}{#1}}                     %yellow

%declarations of Math lines, turn off code checker before using them
\newcommand{\beq}{\begin{equation}}
\newcommand{\beqa}{\begin{equation}\begin{aligned}}
\newcommand{\beqNo}{\begin{equation*}}
\newcommand{\beqaNo}{\begin{equation*}\begin{aligned}}
\newcommand{\eeq}{\end{equation}}
\newcommand{\eeqa}{\end{aligned}\end{equation}}
\newcommand{\eeqNo}{\end{equation*}}
\newcommand{\eeqaNo}{\end{aligned}\end{equation*}}

%space 
\newcommand{\hh}{\hspace{1em}}
\newcommand{\hs}[1]{\hspace{#1}}
\newcommand{\vs}[1]{\vspace{#1}}
\newcommand{\vv}{\vspace{1em}}


%-------------------------------------------------------------------


%Letter-like Notations 

%A - Z notations 
\newcommand{\bA}{\mathbb{A}}    \newcommand{\calA}{\mathcal{A}}     \newcommand{\fA}{\mathfrak{A}}
\newcommand{\bB}{\mathbb{B}}    \newcommand{\calB}{\mathcal{B}}     \newcommand{\fB}{\mathfrak{B}} 
\newcommand{\bC}{\mathbb{C}}    \newcommand{\calC}{\mathcal{C}}     \newcommand{\fC}{\mathfrak{C}} 
\newcommand{\bD}{\mathbb{D}}    \newcommand{\calD}{\mathcal{D}}     \newcommand{\fD}{\mathfrak{D}} 
\newcommand{\bE}{\mathbb{E}}    \newcommand{\calE}{\mathcal{E}}     \newcommand{\fE}{\mathfrak{E}} 
\newcommand{\bF}{\mathbb{F}}    \newcommand{\calF}{\mathcal{F}}     \newcommand{\fF}{\mathfrak{F}} 
\newcommand{\bG}{\mathbb{G}}    \newcommand{\calG}{\mathcal{G}}     \newcommand{\fG}{\mathfrak{G}} 
\newcommand{\bH}{\mathbb{H}}    \newcommand{\calH}{\mathcal{H}}     \newcommand{\fH}{\mathfrak{H}} 
\newcommand{\bI}{\mathbb{I}}    \newcommand{\calI}{\mathcal{I}}     \newcommand{\fI}{\mathfrak{I}} 
\newcommand{\bJ}{\mathbb{J}}    \newcommand{\calJ}{\mathcal{J}}     \newcommand{\fJ}{\mathfrak{J}} 
\newcommand{\bK}{\mathbb{K}}    \newcommand{\calK}{\mathcal{K}}     \newcommand{\fK}{\mathfrak{K}}  
\newcommand{\bL}{\mathbb{L}}    \newcommand{\calL}{\mathcal{L}}     \newcommand{\fL}{\mathfrak{L}} 
\newcommand{\bM}{\mathbb{M}}    \newcommand{\calM}{\mathcal{M}}     \newcommand{\fM}{\mathfrak{M}} 
\newcommand{\bN}{\mathbb{N}}    \newcommand{\calN}{\mathcal{N}}     \newcommand{\fN}{\mathfrak{N}}  
\newcommand{\bO}{\mathbb{O}}    \newcommand{\calO}{\mathcal{O}}     \newcommand{\fO}{\mathfrak{O}} 
\newcommand{\bP}{\mathbb{P}}    \newcommand{\calP}{\mathcal{P}}     \newcommand{\fP}{\mathfrak{P}} 
\newcommand{\bQ}{\mathbb{Q}}    \newcommand{\calQ}{\mathcal{Q}}     \newcommand{\fQ}{\mathfrak{Q}} 
\newcommand{\bR}{\mathbb{R}}    \newcommand{\calR}{\mathcal{R}}     \newcommand{\fR}{\mathfrak{R}} 
\newcommand{\bS}{\mathbb{S}}    \newcommand{\calS}{\mathcal{S}}     \newcommand{\fS}{\mathfrak{S}}     
\newcommand{\bT}{\mathbb{T}}    \newcommand{\calT}{\mathcal{T}}     \newcommand{\fT}{\mathfrak{T}} 
\newcommand{\bU}{\mathbb{U}}    \newcommand{\calU}{\mathcal{U}}     \newcommand{\fU}{\mathfrak{U}} 
\newcommand{\bV}{\mathbb{V}}    \newcommand{\calV}{\mathcal{V}}     \newcommand{\fV}{\mathfrak{V}} 
\newcommand{\bW}{\mathbb{W}}    \newcommand{\calW}{\mathcal{W}}     \newcommand{\fW}{\mathfrak{W}} 
\newcommand{\bX}{\mathbb{X}}    \newcommand{\calX}{\mathcal{X}}     \newcommand{\fX}{\mathfrak{X}} 
\newcommand{\bY}{\mathbb{Y}}    \newcommand{\calY}{\mathcal{Y}}     \newcommand{\fY}{\mathfrak{Y}} 
\newcommand{\bZ}{\mathbb{Z}}    \newcommand{\calZ}{\mathcal{Z}}     \newcommand{\fZ}{\mathfrak{Z}} 

%Math-specific notations 

\newcommand{\ep}{\epsilon}                                          %epsilon
\newcommand{\vep}{\varepsilon}                                      %variable epsilon


%-------------------------------------------------------------------


%Common Operations 

%sets 
\renewcommand{\Cap}{\bigcap}                                        %A intersect with B 
\newcommand{\card}{\text{card }}                                    %cardinality of a set
\renewcommand{\Cup}{\bigcup}                                        %A union with B 
\newcommand{\cut}{\backslash}                                       %A \ B 
\newcommand{\es}{\approx}                                           %equinumerosity between two sets 
\newcommand{\nes}{\not\approx}                                      %not Equinumerosity 
\newcommand{\siq}{\subseteq}                                        %A is contained in B 
\newcommand{\soq}{\supseteq}                                        %A is out of range of B 
\newcommand{\symd}{\Delta}                                          %symmetric difference 

%functions and function-related operations 
\newcommand{\ceil}[1]{\lceil {#1} \rceil}                           %smallest integer no less than #1 
\newcommand{\dist}[2]{\text{dist}\left ({#1},{#2} \right )}         %Distance function 
\newcommand{\floor}[1]{\lfloor {#1} \rfloor}                        %largest integer no greater than #1 
\newcommand{\inv}[1]{{#1}^{-1}}                                     %the inverse of a function, a number or a group element
\renewcommand{\mod}[1]{\,(\text{mod } {#1})}                        %modular of an integer #1 
\newcommand{\ran}{\text{ran }}                                      %the range of a function
\newcommand{\inflim}[1]{\varliminf_{#1}}                            %limit infimum of sequence or set
\newcommand{\limc}[2]{\lim_{#1 \to #2}}                             %limit 
\newcommand{\suplim}[1]{\varlimsup_{#1}}                            %limit supremum of sequence or set

%Others
\newcommand{\larrow}{\leftarrow}                                    %single left arrow, implies
\newcommand{\rarrow}{\rightarrow}                                   %single right arrow, implies 
\newcommand{\Larrow}{\Leftarrow}                                    %double left arrow, implies 
\newcommand{\Rarrow}{\Rightarrow}                                   %double right arrow, implies 
\newcommand{\lrarrow}{\leftrightarrow}                              %single left-right arrow 
\newcommand{\LRarrow}{\Leftrightarrow}                              %double left-right arrow, shorter than \iff 
\newcommand{\lowBrace}[2]{\underset{#1}{\underbrace{#2}}}           %under-brace with notes 
\renewcommand{\(}{\left (}                                          %left-half bracket, turn off code checker before using it 
\renewcommand{\)}{\right )}                                         %right-half bracket, turn off code checker before using it 
\newcommand{\<}{\langle}                                            %left-half angular bracket 
\renewcommand{\>}{\rangle}                                          %right-half angular bracket


%-------------------------------------------------------------------


%Patches  

%Analysis and Calculus 
\newcommand{\df}[2]{\frac{d{#1}}{d{#2}}}                            %one-variable differential operator
%\newcommand{\indf}[2]{\dfrac{d{#1}}{d{#2}}}                         %differential operator for in-line math mode
\newcommand{\im}{\text{Im }}                                        %taking the imaginary part of a number or a function 
\newcommand{\inrPdct}[2]{\left \langle{#1},{#2}\right \rangle}      %inner product 
\newcommand{\nDf}[2]{{#1}^{({#2})}}                                 %n-th derivative of some function 
\newcommand{\norm}[1]{\left \|#1\right \|}                                       %norm 
\newcommand{\parf}[2]{\frac{\partial{#1}}{\partial{#2}}}            %partial derivative of a function 
\newcommand{\parfwide}[2]{\partial{#1} / \partial{#2}}              %partial derivative of a function, wide form 
\newcommand{\re}{\text{Re }}                                        %taking the real part of a number or a function 

%Combinatorics 
\newcommand{\al}{{\aleph}_{0}}                                      %Aleph zero 
\newcommand{\all}{{\aleph}_{1}}                                     %Aleph one
\newcommand{\dbinomial}[2]{\begin{pmatrix}\!\!
                           \begin{pmatrix}{#1}\\{#2} 
                           \end{pmatrix}
                           \!\!\!
                           \end{pmatrix}}                           %double binomial 
\newcommand{\kpermu}[2]{{#1}^\text{\underline{#2}}}                 %k-permutation 
\newcommand{\logs}{\log^{*}}                                        %Iterated Logarithm 
\newcommand{\stirling}[2]{\left \{\begin{aligned} 
                                    {#1} \\ {#2} 
                                  \end{aligned} 
                          \right \}}                                %Stirling numbers of second kind 

%Graph Theory 
\newcommand{\comp}[1]{\text{comp} \left({#1}\right)}                %Component of graph 
\newcommand{\conj}{\overline}                                       %Complement graph of a given graph

%intersection of a collection of sets indexed by I 
\newcommand{\AI}{\cap_{i \in I} A_i} 
\newcommand{\BI}{\cap_{i \in I} B_i} 
\newcommand{\CI}{\cap_{i \in I} C_i} 
\newcommand{\DI}{\cap_{i \in I} D_i} 
\newcommand{\EI}{\cap_{i \in I} E_i} 
\newcommand{\FI}{\cap_{i \in I} F_i} 
\newcommand{\GI}{\cap_{i \in I} G_i} 
\newcommand{\HI}{\cap_{i \in I} H_i} 
\newcommand{\II}{\cap_{i \in I} I_i} 
\newcommand{\JI}{\cap_{i \in I} J_i} 
\newcommand{\KI}{\cap_{i \in I} K_i} 
\newcommand{\LI}{\cap_{i \in I} L_i} 
\newcommand{\MI}{\cap_{i \in I} M_i} 
\newcommand{\NI}{\cap_{i \in I} N_i} 
\newcommand{\OI}{\cap_{i \in I} O_i} 
\newcommand{\PI}{\cap_{i \in I} P_i} 
\newcommand{\QI}{\cap_{i \in I} Q_i} 
\newcommand{\RI}{\cap_{i \in I} R_i} 
%\renewcommand{\SI}{\cap_{i \in I} S_i}                             %\SI already defined, delete '%' if you want to renew it 
\newcommand{\TI}{\cap_{i \in I} T_i} 
\newcommand{\UI}{\cap_{i \in I} U_i} 
\newcommand{\VI}{\cap_{i \in I} V_i} 
\newcommand{\WI}{\cap_{i \in I} W_i} 
\newcommand{\XI}{\cap_{i \in I} X_i} 
\newcommand{\YI}{\cap_{i \in I} Y_i}
\newcommand{\ZI}{\cap_{i \in I} Z_i}                                

%limit as "letter" tends to infinity 
\newcommand{\limftya}{\lim_{a \to \infty}}                          
\newcommand{\limftyb}{\lim_{b \to \infty}} 
\newcommand{\limftyc}{\lim_{c \to \infty}} 
\newcommand{\limftyd}{\lim_{d \to \infty}} 
\newcommand{\limftye}{\lim_{e \to \infty}} 
\newcommand{\limftyf}{\lim_{f \to \infty}} 
\newcommand{\limftyg}{\lim_{g \to \infty}} 
\newcommand{\limftyh}{\lim_{h \to \infty}}
\newcommand{\limftyi}{\lim_{i \to \infty}}                          
\newcommand{\limftyj}{\lim_{j \to \infty}}                          
\newcommand{\limftyk}{\lim_{k \to \infty}}
\newcommand{\limftyl}{\lim_{l \to \infty}}
\newcommand{\limftym}{\lim_{m \to \infty}}
\newcommand{\limftyn}{\lim_{n \to \infty}}                          
\newcommand{\limftyo}{\lim_{o \to \infty}} 
\newcommand{\limftyp}{\lim_{p \to \infty}} 
\newcommand{\limftyq}{\lim_{q \to \infty}}
\newcommand{\limftyr}{\lim_{r \to \infty}}
\newcommand{\limftys}{\lim_{s \to \infty}}
\newcommand{\limftyt}{\lim_{t \to \infty}}
\newcommand{\limftyu}{\lim_{u \to \infty}}
\newcommand{\limftyv}{\lim_{v \to \infty}}
\newcommand{\limftyw}{\lim_{w \to \infty}} 
\newcommand{\limftyx}{\lim_{x \to \infty}}                          
\newcommand{\limftyy}{\lim_{y \to \infty}}                          
\newcommand{\limftyz}{\lim_{z \to \infty}} 

%matrices 
\newcommand{\colTwo}[2]{\begin{pmatrix}
                        #1 \\ #2 
                        \end{pmatrix}}                              %length-two column vector 
\newcommand{\colThree}[3]{\begin{pmatrix}
                        #1 \\ #2 \\ #3
                        \end{pmatrix}}                              %length-three column vector 
\newcommand{\colFour}[4]{\begin{pmatrix}
                        #1 \\ #2 \\ #3 \\ #4 
                        \end{pmatrix}}                              %length-four column vector 
\newcommand{\colFive}[5]{\begin{pmatrix}
                        #1 \\ #2 \\ #3 \\ #4 \\ #5
                        \end{pmatrix}}                              %length-five column vector 
\newcommand{\diagTwo}[2]{\begin{pmatrix} 
                         #1 &\, \\ 
                         \, &#2 \\ 
                         \end{pmatrix}}                             %two-by-two diagonal matrix 
\newcommand{\diagThree}[3]{\begin{pmatrix} 
                           #1 &\, &\, \\ 
                           \, &#2 &\, \\ 
                           \, &\, &#3 \\ 
                           \end{pmatrix}}                           %three-by-three diagonal matrix 
\newcommand{\diagFour}[4]{\begin{pmatrix} 
                           #1 &\, &\, &\, \\ 
                           \, &#2 &\, &\, \\ 
                           \, &\, &#3 &\, \\ 
                           \, &\, &\, &#4 \\ 
                           \end{pmatrix}}                           %four-by-four diagonal matrix }
\newcommand{\maTwo}[4]{\begin{pmatrix} 
                        #1 &#2 \\  
                        #3 &#4 \\ 
                       \end{pmatrix}}                               %two-by-two matrix
\newcommand{\maTwoThree}[6]{\begin{pmatrix} 
                            #1 &#2 &#3 \\ 
                            #4 &#5 &#6 \\ 
                            \end{pmatrix}}                          %two-by-three matrix
\newcommand{\maThreeTwo}[6]{\begin{pmatrix}
                            #1 &#2 \\ 
                            #3 &#4 \\ 
                            #5 &#6 \\ 
                            \end{pmatrix}}                          %three by two matrix 
\newcommand{\maThree}[9]{\begin{pmatrix}
                         #1 &#2 &#3 \\ 
                         #4 &#5 &#6 \\ 
                         #7 &#8 &#9 \\ 
                         \end{pmatrix}}                             %three by three matrix

%determinants 
\newcommand{\deTwo}[4]{\det \begin{pmatrix}
                            #1 &#2 \\ 
                            #3 &#4 \\ 
                            \end{pmatrix}}                          %determinant of two by two matrix 
\newcommand{\deThree}[9]{\det \begin{pmatrix}
                              #1 &#2 &#3 \\ 
                              #4 &#5 &#6 \\ 
                              #7 &#8 &#9 \\ 
                              \end{pmatrix}}                        %determinant of three by three matrix 

%span 
\renewcommand{\span}{\text{span }}                                  %the linear span of a set
\newcommand{\spans}[1]{\text{span} \{#1\}}                          %the linear span of elements 

%Topology
\newcommand{\hBall}[2]{B_{#1}\left(#2\right)}                   %B_#1 (#2): open ball centered at #2 with radius #1 
\usetheme[dove]{Boadilla}
\usecolortheme{dolphin}
%\pgfdeclaremask{figmask}{or_circuit.jpg}
%\pgfdeclareimage[mask=figmask,width=0.6\textwidth]{or_circuit}{or_circuit_1.png}
\useoutertheme{miniframes}
\begin{document}
    \usebackgroundtemplate{\tikz\node[opacity=0.25]{
    \includegraphics[width=\paperwidth,
    height=\paperheight]{hamster.jpg}
    };}
\begin{titlepage}
    \begin{center}
        VE203 - Discrete Mathmatics 
    \end{center}
\end{titlepage}
\myfont
\section{Relations}
\begin{frame}
    \frametitle{Relation}
    \hh A subset $R \subset A \times B$ is called a (binary) relation from $A$ to $B$. 
    If $A = B$,	we say that $R$ is a relation on $A$.\\ 
	\vv
    Check the following:
    \begin{itemize}
        \item domain$(R)=\{x \mid \exists y(x R y)\}$
        \item range$(R)=\{y \mid \exists x(x R y)\}$
        \item $R = \emptyset$: the empty relation
        \item $A = B$: identity relation
        \item The relation $A \times B$ itself? 
    \end{itemize}
\end{frame}
\begin{frame}
    \frametitle{Functions}
    \hh A function is a relation $F$ such that $$\forall x \in \operatorname{dom} F(\exists ! y(x F y))$$.
	Check the following:
    \begin{itemize}
        \item  For a function $F$ and a point $x \in \operatorname{dom} (F)$, the unique $y$ such that $xFy$ is called the \blue{value} of $F$ at $x$ and is denoted $F(y)$. 
        \item Given function $F : A \to B$, then $\forall x, y \in A(x = y \Rightarrow F(x) = F(y))$.
        \item Partial Function/ Total Tunction.
    \end{itemize}
\end{frame}
\begin{frame}
    \frametitle{Operations on Functions}
    \par For \red{arbitrary} sets A, relations F, and functions G,
	\vv
    \begin{itemize}
		\item \blue{Inverse:} $F^T = F^{-1} = \{(y, x) \mid xFy\}$.
        \item \blue{Composition:} $F \circ G = \{(x, z) \mid \exists y \in A(xFy \wedge yGz)\}$.
		\item \blue{Restriction:} $F \mid A = \{(x, y) \in F \mid x \in A \}$.
		\item \blue{Image:} $F(A) = \operatorname{ran}(F \mid A) = \{y \mid (\exists x \in A) xFy \}$. 
	\end{itemize}
    \vv
    If $F$ is a function, then $F(A) = \{F(x) \mid x \in A\}$.
\end{frame}
\begin{frame}
    \frametitle{Exercise}
    1. Let $A: \bR^3 \to \bR^3 $ be given by
    $$A=\maThree{1}{2}{3}{2}{1}{2}{1}{1}{1}.$$
    Let 
    $$U=\text{span} {\left \{\begin{pmatrix}1\\1\\0\\\end{pmatrix}\right \} }$$
    \begin{block}{Question}
        \begin{itemize}
            \item[-]What's the restriction of $A$ to $U\,$? 
            \item[-]What's the image of the restriction?
            \item[-]What's the inverse of the restriction? 
        \end{itemize}
    \end{block}
    (Take from vv286 lecture slides)
\end{frame}
\begin{frame}
    \frametitle{-jectivity}
    \parbox{\textwidth}{
		\par Given a function $F : A \to B$, with $\operatorname{dom} F = A$ and $ \operatorname{ran}(F) \subset B$, then
		\begin{itemize}
			\item[-] $F$ is \blue{injective or \red{one-to-one}} if $\forall x, y \in A(F(x) = F(y) \Rightarrow x = y)$;
			\item[-] $F$ is \blue{surjective or \red{onto}} if ran$(F) = B$;
			\item[-] $F$ is \yellow{bijective} if it is both injective and surjective.
		\end{itemize}
		\vs{0.3em}
		\par Given a function $F : A \to B$, $A \neq \varnothing$, then
		\begin{itemize}
			\item[-] There exists a function $G : B \to A$ (a \blue{``left inverse''}) such that
			$G \circ F = id_A \Leftrightarrow F$ is one-to-one;
			\item[-] There exists a function $G : B \to A$ (a \blue{``right inverse''}) such that
			$F \circ G = id_B \Leftrightarrow F$ is onto.
		\end{itemize}
        \begin{block}{Let $f : A \to B, g : B \to C$,}
            \begin{itemize}
                \item[-] If $g \circ f$ is injective, then $f$ is injective.
                \item[-] If $g \circ f$ is surjective, then $g$ is surjective
            \end{itemize}
        \end{block}
		%\par 
	}
\end{frame}
\begin{frame}
    \frametitle{Exercise}
    2. Recall that $\mathbb{Z}$ denotes the set of integers, $\mathbb{Z}^+$ the set of positive integers, and $\mathbb{Q}$ the set of rational numbers. 
    Define a function: 
    \begin{equation*}
    f: \mathbb{Z} \times \mathbb{Z}^+ \to \mathbb{Q},~~~~~~~~~~
    f~(p,q) = \frac{p}{q}.
    \end{equation*}
    \begin{enumerate}
    	\item Is $f$ an injection? Why?
    	\item Is $f$ an surjection? Why?
    	\item Is $f$ an bijection? Why?
    \end{enumerate}
\end{frame}
\begin{frame}
    \frametitle{Definition}
    \parbox{\textwidth}{
		\par A (binary) relation $R$ on $A$, \textit{i.e.,} $R \subset A \times A$, is
		\begin{itemize}
			\item[-] \textbf{ref\mbox{l}exive} if $aRa \Rightarrow \top$.
			\item[-] \textbf{symmetric} if $aRb \Leftrightarrow bRa$.
			\item[-] \textbf{transitive} if $aRb \wedge bRc \Rightarrow aRc$.
			\item[-] \textbf{anti-symmetric} if $aRb \wedge bRa \Rightarrow a = b$.
			\item[-] asymmetric if $aRb \wedge bRa \Rightarrow \perp$.
			\item[-] total if $aRb \vee bRa \Rightarrow \top$.
		\end{itemize}
		\par \phantom{ji}
		\par \textbf{(Non-strict) Partial order}: reflexive, antisymmetric, and transitive.
		\par \textbf{Equivalence relation}: reflexive, symmetric, and transitive.
        \par \textbf{Total order:} Partial order + total.
    }
\end{frame}
\begin{frame}
    \frametitle{Exercise}
    3. Recall that $\mathbb{R}$ denotes the set of real numbers, while $\mathbb{Z}$ denotes the set of integers. Define a relation $\sim$ on $\mathbb{R}$ by
    \begin{equation*}
        x \sim y \Leftrightarrow x - y \in \mathbb{Z}
    \end{equation*}
    for any $x, y \in \mathbb{R}$. Prove that $\sim$ is an equivalence relation.
\end{frame}
\begin{frame}
    \frametitle{Equivalence Class}
    Given an equivalence relation $R$ on $A$,
    \begin{itemize}
        \item \blue{Equivalence class containing $x$} $$\left[ x\right]_R = \{t \in A \mid xRt\}.$$
        \item This is also a partition for $A$.
        \item For $x , y \in A$, $$\left[ x \right]_R  = \left[ y \right]_R \Leftrightarrow xRy.$$
        \item Quotient set is given by $$A/R = \{[x]R \mid x \in A\}.$$
    \end{itemize}
\end{frame}
\section{Equinumerosity}
\begin{frame}
    \frametitle{Equinumerosity}
    \yellow{Definition:}\\
    \hh A set $A$ is equinumerous to a set $B$ (written $A \es B$) if there is a 
    \blue{bijection} from A to B.
    \\
    \cha{Examples:}\\
    \begin{itemize}
        \item $\bR \es (0,1)$
        \item $\bN \nes \bR$
        \item $\bN \es \bN^2$
        \item $\bN \es \bN^3$
        \item $\bN \es \bN^\bN?$
    \end{itemize}
    \begin{block}{Question}
        \begin{itemize}
            \item[-] Why isn't it an \blue{equivalence relation}?
            \item[-] How to prove/disprove a equinumerosity?
            \item[-] How is $\bN,\bZ,\bQ,\bR$ constructed respectively? 
        \end{itemize}
    \end{block}
\end{frame}
\begin{frame}
    \frametitle{Cantor's Theorem}
    \fbox{
        \parbox{0.95\textwidth}{
            \textbf{Cantor’s Theorem}:
            \begin{itemize}
                \item $\mathbb{R} \not\approx \mathbb{N}$.
                \item For every set $A$, $A \not\approx \mathcal{P}(A)$.
            \end{itemize}
        }
    }
    \vs{2em}
    \begin{block}{Top asked questions}
        \begin{itemize}
            \item $\calP(\bN) = \bR$? 
            \item Why is $\bR$ not countable? 
            \item How to prove cantor's theorem? 
        \end{itemize}
    \end{block}
\end{frame}
\begin{frame}
    \frametitle{Example}
    \hh Let $A = \left\{ {a,b,c,d,e} \right\}$. The mapping $f:A \to \mathcal{P}\left( A \right)$ is defined by
    \begin{equation*}
        {f\left( a \right) = \left\{ {a,d,e} \right\},\;}\kern0pt{f\left( b \right) = \left\{ {a,c} \right\},\;}\kern0pt{f\left( c \right) = \left\{ {a,b,d,e} \right\},\;}\kern0pt{f\left( d \right) = \varnothing,\;}\kern0pt{f\left( e \right) = \left\{ {b,c,e} \right\}.}
    \end{equation*}
    \par Determine the set $B = \left\{ {x \in A \mid x \not\in f\left( x \right)} \right\}.$
    \vv
    \begin{block}{Solution}
        This set is used in the proof of Cantor’s theorem. We see that
        \begin{equation*}
            {a \in f\left( a \right),\;}\kern0pt{b \notin f\left( b \right),\;}\kern0pt{c \notin f\left( c \right),\;}\kern0pt{d \notin f\left( d \right),\;}\kern0pt{e \in f\left( e \right).}
        \end{equation*}
    \par Hence, ${B = \left\{ {x \in A \mid x \notin f\left( x \right)} \right\} }={ \left\{ {b,c,d} \right\}.}$
    \end{block}
\end{frame}
\begin{frame}
    \frametitle{Exercise}
    4. The mapping function $f:\mathbb{N} \to \mathcal{P}\left( \mathbb{N} \right)$ is defined by 
    \begin{equation*}
        f\left( n \right) = \mathbb{N}\backslash \left\{ {{n^2} - \left( {2m - 1} \right)n} \right\},
    \end{equation*}
    where $n,m \in \mathbb{N}$. Determine the set $B = \left\{ {x \in A \mid x \not\in f\left( x \right)} \right\}.$
    \vs{4em}\\
    5. Prove that the set of all sets does not exist.
\end{frame}
\section{Pigeonhole Principle}
\begin{frame}
    \frametitle{Pigeonhole Principle}
    \yellow{Two versions¿:}
    \vv 
    \begin{enumerate}
        \item[xxs] Let $r,s \in \mathbb{N} \backslash\!\left\lbrace 0 \right\rbrace $, 
        if a set containing at least ${rs} + 1$ elements is partitioned into $r$ subsets, 
        then some subsets contains at least $s + 1$ elements.
        \item[dxs] No set of the form $[n] = \left\lbrace 1, \cdots, n\right\rbrace$ 
        is equinumerous to a proper subset of itself, 
        where $n \in \mathbb{N}$.
	\end{enumerate}
\end{frame}
\begin{frame}
    \frametitle{Exercise}
    6. Consider a sequence $\left\lbrace \sqrt{3}, 2\sqrt{3}, 3\sqrt{3}, \cdots \right\rbrace$.
    Prove that there are infinite number of terms have a mantissa of less than 0.01.
    

\end{frame}
\section{*Extra Topic}
\begin{frame}
    \frametitle{Sorting Algorithms}
    We have lots lots of sorting algorithms!
    \vv
    \begin{itemize}
        \item Demostration: \url{https://www.byteflying.com/archives/6171}
        \item Classification
        \item Costs
                \begin{itemize}
                    \item Time complexity (Worst/Averge/Best)
                    \item Space complexity
                    \item Stability
                \end{itemize}
    \end{itemize}
    \begin{block}{Question}
        \hh How to choose a sorting algorithms that most satisfy your need?
    \end{block}
\end{frame}

\definecolor{mygreen}{rgb}{0,0.6,0}
\definecolor{mygray}{rgb}{0.5,0.5,0.5}
\definecolor{mymauve}{rgb}{0.58,0,0.82}
\definecolor{mypurple}{rgb}{0.58,0.02,0.82}
\definecolor{myblue}{rgb}{0.1,0.2,0.9}
\definecolor{myorange}{rgb}{0.73,0.38,0.17}
% https://en.wikibooks.org/wiki/LaTeX/Source_Code_Listings
\lstset{%
	language=C++,
	backgroundcolor=\color{white},
	basicstyle=\tiny,
	breakatwhitespace=false,
	breaklines=true,
	captionpos=t,
	commentstyle=\color{mygreen},
	deletekeywords={...},
	escapeinside={\%*}{*)},
	extendedchars=true,
	frame=single,
	keepspaces=true,
	keywordstyle=\color{blue},
	language=Octave,
	%otherkeywords={*,...},
	numbers=left,
	numbersep=5pt,
	numberstyle=\tiny\color{mygray},
	rulecolor=\color{black},
	showspaces=false,
	showstringspaces=false,
	showtabs=false,
	stepnumber=1,
	stringstyle=\color{mymauve},
	tabsize=4,
	title=\lstname
}

\lstdefinestyle{customcpp}{
	belowcaptionskip=0pt,
	breaklines=true,
	%frame=L,
	%xleftmargin=\parindent,
	language=C++,
	showstringspaces=false,
	basicstyle=\scriptsize\ttfamily,
	keywordstyle=\bfseries\color{mypurple},
	commentstyle=\itshape\color{green!40!black},
	identifierstyle=\color{myblue},
	stringstyle=\color{myorange},
}

\lstset{escapechar=@,style=customcpp}
\begin{frame}[fragile]
    \frametitle{Quick Sort with Second key word}
    \hbox{
    \begin{lstlisting}
    void QuickSort(vector<int>& v,int left,int right,vector<int>& o){
        if (left>=right)  return;
        int key = (left+right)/2;// you can also choose right/left
        while (left<right){
            while (left<right&& (v[right] > v[key] || v[right]==v[key] && o[right]>o[key]))  right--;
            while (left<right&& (v[left] <= v[key] || v[left]==v[key] && o[left]<o[key]))   left++;  
            if (left < right) swap(v[left], v[right]);   
        }
        swap(v[left], v[key]);   //left== right
        int meet= key;     //divide into two parts
        QuickSort(v, left, meet-1); 
        QuickSort(v, meet+1, right); 
    }\end{lstlisting}
    }
    

\end{frame}
\begin{frame}[fragile]
    \frametitle{Modified Bubble Sort}
    \hbox{
    \begin{lstlisting}
        public static void bubble_sort(int[] intArr) {
            int max = intArr.length - 1;
            int secondCount = max;
            //record where the exchange happen last time
            for (int i = 0; i < max; i++) {
                System.out.println( (i + 1) + "times");
                boolean flag = true;int lastChangeIndex = 0;
                for (int j = 0; j < secondCount; j++) {
                    if (intArr[j] > intArr[j + 1]) {
                        swap(Arr[j],Arr[j+1]);
                        flag = false; lastChangeIndex = j;
                    }
                    System.out.println("Compare:"+(j+1)+", Result:"+ Arrays.toString(intArr));
                }
                if (flag) break;//already well ordering
                secondCount = lastChangeIndex;//update
            }
        }\end{lstlisting}
    }
\end{frame}
\begin{frame}
    \frametitle{Exercise}
    7. Try to implement \blue{three-way merge sort} in C++.
    If you know \blue{Master Theorem}, try to calculate the time complexity and compare with the 
    original merge sort.
    \vs{4em}\\
    8. Finish the following exercise regarding sorting algorithms:
    \begin{itemize}
        \item \url{https://vijos.org/p/1398}
        \item \url{https://vijos.org/p/1257}
    \end{itemize}
\end{frame}
\begin{frame}
    \frametitle{Reference}

    \begin{itemize}
        \item Examples from Vv286 Lecture Slides.
        \item Exercises from 2021-Fall-Ve203 TA Zhao Jiayuan
        \item Stable Quick Sort, \url{https://blog.csdn.net/liuchenjane/article/details/72902325}
        \item Modified Bubble Sort, \url{https://blog.csdn.net/weixin_43168559/article/details/88873585}
    \end{itemize}

\end{frame}
\end{document}