\documentclass{beamer}
\newcommand{\myfont}{\rmfamily\normalsize\upshape\mdseries}
\newcommand{\degree}{^\circ}
\title{\sffamily Review IV(Slides 226 - 274)}
\subtitle{\textbf{Basic Number Theory\& Group Theory}\\}
\institute[UM-SJTU JI]{University of Michigan-Shanghai Jiao Tong University Joint Institute}
\author{HamHam}
\usepackage{graphicx}
\usepackage{picinpar}
\usepackage{ctex}
\usepackage{indentfirst}
\usepackage{chemformula}
\usepackage{geometry}
\usepackage{subfigure}
\usepackage{appendix}
\usepackage{amsfonts}
\usepackage{enumerate}
\usepackage{float}
\usepackage{geometry}
\usepackage{latexsym}
\usepackage{listings}
\usepackage{multicol,multirow,multido}
\usepackage{tabularx}
\usepackage{ulem}
\usepackage{tikz}
\usepackage{xcolor}
\usepackage{cite}
\usepackage{setspace}
\usepackage{hyperref}
\usepackage{textpos}
\usepackage{booktabs}
\usepackage{diagbox}
\usepackage{listings}
\usepackage{graphics}
\usepackage{upgreek}
\usepackage{JI_MathCourse_Notations}
%%%%%%%%%%%%%%%%%%%%%%%%%%%%%%%%%%%%%%%%%%%%%%%%%%%%%%%%%%%%%%%%%%%%%%%%%%%%%%
\iffalse 
Copyright 2021 by Leyang Zhang, Yinchen Ni, Yuxiang Chen, Yue Huang

You should not spread this file without the permission from every one of 
    Leyang Zhang, Yinchen Ni and Yuxiang Chen 
\fi
%%%%%%%%%%%%%%%%%%%%%%%%%%%%%%%%%%%%%%%%%%%%%%%%%%%%%%%%%%%%%%%%%%%%%%%%%%%%%

\ProvidesPackage{JI_MathCourse_Notations}[2021/12/19]

\RequirePackage{amsmath}
\RequirePackage{amsthm}
\RequirePackage{amssymb}
\RequirePackage{bm}
\RequirePackage{bbm}
\RequirePackage{color}


%Typesetting

%colors 
\newcommand{\blue}[1]{\textcolor{blue}{#1}}                         %blue
\newcommand{\cha}[1]{\textcolor[rgb]{0.95,0.75,0.9}{#1}}            %cha-type color 
\newcommand{\gray}[1]{\textcolor{gray}{#1}}                         %grey
\newcommand{\pink}[1]{\textcolor{pink}{#1}}                         %pink
\newcommand{\red}[1]{\textcolor[rgb]{0.75,0,0}{#1}}                 %red
\newcommand{\yellow}[1]{\textcolor{orange}{#1}}                     %yellow

%declarations of Math lines, turn off code checker before using them
\newcommand{\beq}{\begin{equation}}
\newcommand{\beqa}{\begin{equation}\begin{aligned}}
\newcommand{\beqNo}{\begin{equation*}}
\newcommand{\beqaNo}{\begin{equation*}\begin{aligned}}
\newcommand{\eeq}{\end{equation}}
\newcommand{\eeqa}{\end{aligned}\end{equation}}
\newcommand{\eeqNo}{\end{equation*}}
\newcommand{\eeqaNo}{\end{aligned}\end{equation*}}

%space 
\newcommand{\hh}{\hspace{1em}}
\newcommand{\hs}[1]{\hspace{#1}}
\newcommand{\vs}[1]{\vspace{#1}}
\newcommand{\vv}{\vspace{1em}}


%-------------------------------------------------------------------


%Letter-like Notations 

%A - Z notations 
\newcommand{\bA}{\mathbb{A}}    \newcommand{\calA}{\mathcal{A}}     \newcommand{\fA}{\mathfrak{A}}
\newcommand{\bB}{\mathbb{B}}    \newcommand{\calB}{\mathcal{B}}     \newcommand{\fB}{\mathfrak{B}} 
\newcommand{\bC}{\mathbb{C}}    \newcommand{\calC}{\mathcal{C}}     \newcommand{\fC}{\mathfrak{C}} 
\newcommand{\bD}{\mathbb{D}}    \newcommand{\calD}{\mathcal{D}}     \newcommand{\fD}{\mathfrak{D}} 
\newcommand{\bE}{\mathbb{E}}    \newcommand{\calE}{\mathcal{E}}     \newcommand{\fE}{\mathfrak{E}} 
\newcommand{\bF}{\mathbb{F}}    \newcommand{\calF}{\mathcal{F}}     \newcommand{\fF}{\mathfrak{F}} 
\newcommand{\bG}{\mathbb{G}}    \newcommand{\calG}{\mathcal{G}}     \newcommand{\fG}{\mathfrak{G}} 
\newcommand{\bH}{\mathbb{H}}    \newcommand{\calH}{\mathcal{H}}     \newcommand{\fH}{\mathfrak{H}} 
\newcommand{\bI}{\mathbb{I}}    \newcommand{\calI}{\mathcal{I}}     \newcommand{\fI}{\mathfrak{I}} 
\newcommand{\bJ}{\mathbb{J}}    \newcommand{\calJ}{\mathcal{J}}     \newcommand{\fJ}{\mathfrak{J}} 
\newcommand{\bK}{\mathbb{K}}    \newcommand{\calK}{\mathcal{K}}     \newcommand{\fK}{\mathfrak{K}}  
\newcommand{\bL}{\mathbb{L}}    \newcommand{\calL}{\mathcal{L}}     \newcommand{\fL}{\mathfrak{L}} 
\newcommand{\bM}{\mathbb{M}}    \newcommand{\calM}{\mathcal{M}}     \newcommand{\fM}{\mathfrak{M}} 
\newcommand{\bN}{\mathbb{N}}    \newcommand{\calN}{\mathcal{N}}     \newcommand{\fN}{\mathfrak{N}}  
\newcommand{\bO}{\mathbb{O}}    \newcommand{\calO}{\mathcal{O}}     \newcommand{\fO}{\mathfrak{O}} 
\newcommand{\bP}{\mathbb{P}}    \newcommand{\calP}{\mathcal{P}}     \newcommand{\fP}{\mathfrak{P}} 
\newcommand{\bQ}{\mathbb{Q}}    \newcommand{\calQ}{\mathcal{Q}}     \newcommand{\fQ}{\mathfrak{Q}} 
\newcommand{\bR}{\mathbb{R}}    \newcommand{\calR}{\mathcal{R}}     \newcommand{\fR}{\mathfrak{R}} 
\newcommand{\bS}{\mathbb{S}}    \newcommand{\calS}{\mathcal{S}}     \newcommand{\fS}{\mathfrak{S}}     
\newcommand{\bT}{\mathbb{T}}    \newcommand{\calT}{\mathcal{T}}     \newcommand{\fT}{\mathfrak{T}} 
\newcommand{\bU}{\mathbb{U}}    \newcommand{\calU}{\mathcal{U}}     \newcommand{\fU}{\mathfrak{U}} 
\newcommand{\bV}{\mathbb{V}}    \newcommand{\calV}{\mathcal{V}}     \newcommand{\fV}{\mathfrak{V}} 
\newcommand{\bW}{\mathbb{W}}    \newcommand{\calW}{\mathcal{W}}     \newcommand{\fW}{\mathfrak{W}} 
\newcommand{\bX}{\mathbb{X}}    \newcommand{\calX}{\mathcal{X}}     \newcommand{\fX}{\mathfrak{X}} 
\newcommand{\bY}{\mathbb{Y}}    \newcommand{\calY}{\mathcal{Y}}     \newcommand{\fY}{\mathfrak{Y}} 
\newcommand{\bZ}{\mathbb{Z}}    \newcommand{\calZ}{\mathcal{Z}}     \newcommand{\fZ}{\mathfrak{Z}} 

%Math-specific notations 

\newcommand{\ep}{\epsilon}                                          %epsilon
\newcommand{\vep}{\varepsilon}                                      %variable epsilon


%-------------------------------------------------------------------


%Common Operations 

%sets 
\renewcommand{\Cap}{\bigcap}                                        %A intersect with B 
\newcommand{\card}{\text{card }}                                    %cardinality of a set
\renewcommand{\Cup}{\bigcup}                                        %A union with B 
\newcommand{\cut}{\backslash}                                       %A \ B 
\newcommand{\es}{\approx}                                           %equinumerosity between two sets 
\newcommand{\nes}{\not\approx}                                      %not Equinumerosity 
\newcommand{\siq}{\subseteq}                                        %A is contained in B 
\newcommand{\soq}{\supseteq}                                        %A is out of range of B 
\newcommand{\symd}{\Delta}                                          %symmetric difference 

%functions and function-related operations 
\newcommand{\ceil}[1]{\lceil {#1} \rceil}                           %smallest integer no less than #1 
\newcommand{\dist}[2]{\text{dist}\left ({#1},{#2} \right )}         %Distance function 
\newcommand{\floor}[1]{\lfloor {#1} \rfloor}                        %largest integer no greater than #1 
\newcommand{\inv}[1]{{#1}^{-1}}                                     %the inverse of a function, a number or a group element
\renewcommand{\mod}[1]{\,(\text{mod } {#1})}                        %modular of an integer #1 
\newcommand{\ran}{\text{ran }}                                      %the range of a function
\newcommand{\inflim}[1]{\varliminf_{#1}}                            %limit infimum of sequence or set
\newcommand{\limc}[2]{\lim_{#1 \to #2}}                             %limit 
\newcommand{\suplim}[1]{\varlimsup_{#1}}                            %limit supremum of sequence or set

%Others
\newcommand{\larrow}{\leftarrow}                                    %single left arrow, implies
\newcommand{\rarrow}{\rightarrow}                                   %single right arrow, implies 
\newcommand{\Larrow}{\Leftarrow}                                    %double left arrow, implies 
\newcommand{\Rarrow}{\Rightarrow}                                   %double right arrow, implies 
\newcommand{\lrarrow}{\leftrightarrow}                              %single left-right arrow 
\newcommand{\LRarrow}{\Leftrightarrow}                              %double left-right arrow, shorter than \iff 
\newcommand{\lowBrace}[2]{\underset{#1}{\underbrace{#2}}}           %under-brace with notes 
\renewcommand{\(}{\left (}                                          %left-half bracket, turn off code checker before using it 
\renewcommand{\)}{\right )}                                         %right-half bracket, turn off code checker before using it 
\newcommand{\<}{\langle}                                            %left-half angular bracket 
\renewcommand{\>}{\rangle}                                          %right-half angular bracket


%-------------------------------------------------------------------


%Patches  

%Analysis and Calculus 
\newcommand{\df}[2]{\frac{d{#1}}{d{#2}}}                            %one-variable differential operator
%\newcommand{\indf}[2]{\dfrac{d{#1}}{d{#2}}}                         %differential operator for in-line math mode
\newcommand{\im}{\text{Im }}                                        %taking the imaginary part of a number or a function 
\newcommand{\inrPdct}[2]{\left \langle{#1},{#2}\right \rangle}      %inner product 
\newcommand{\nDf}[2]{{#1}^{({#2})}}                                 %n-th derivative of some function 
\newcommand{\norm}[1]{\left \|#1\right \|}                                       %norm 
\newcommand{\parf}[2]{\frac{\partial{#1}}{\partial{#2}}}            %partial derivative of a function 
\newcommand{\parfwide}[2]{\partial{#1} / \partial{#2}}              %partial derivative of a function, wide form 
\newcommand{\re}{\text{Re }}                                        %taking the real part of a number or a function 

%Combinatorics 
\newcommand{\al}{{\aleph}_{0}}                                      %Aleph zero 
\newcommand{\all}{{\aleph}_{1}}                                     %Aleph one
\newcommand{\dbinomial}[2]{\begin{pmatrix}\!\!
                           \begin{pmatrix}{#1}\\{#2} 
                           \end{pmatrix}
                           \!\!\!
                           \end{pmatrix}}                           %double binomial 
\newcommand{\kpermu}[2]{{#1}^\text{\underline{#2}}}                 %k-permutation 
\newcommand{\logs}{\log^{*}}                                        %Iterated Logarithm 
\newcommand{\stirling}[2]{\left \{\begin{aligned} 
                                    {#1} \\ {#2} 
                                  \end{aligned} 
                          \right \}}                                %Stirling numbers of second kind 

%Graph Theory 
\newcommand{\comp}[1]{\text{comp} \left({#1}\right)}                %Component of graph 
\newcommand{\conj}{\overline}                                       %Complement graph of a given graph

%intersection of a collection of sets indexed by I 
\newcommand{\AI}{\cap_{i \in I} A_i} 
\newcommand{\BI}{\cap_{i \in I} B_i} 
\newcommand{\CI}{\cap_{i \in I} C_i} 
\newcommand{\DI}{\cap_{i \in I} D_i} 
\newcommand{\EI}{\cap_{i \in I} E_i} 
\newcommand{\FI}{\cap_{i \in I} F_i} 
\newcommand{\GI}{\cap_{i \in I} G_i} 
\newcommand{\HI}{\cap_{i \in I} H_i} 
\newcommand{\II}{\cap_{i \in I} I_i} 
\newcommand{\JI}{\cap_{i \in I} J_i} 
\newcommand{\KI}{\cap_{i \in I} K_i} 
\newcommand{\LI}{\cap_{i \in I} L_i} 
\newcommand{\MI}{\cap_{i \in I} M_i} 
\newcommand{\NI}{\cap_{i \in I} N_i} 
\newcommand{\OI}{\cap_{i \in I} O_i} 
\newcommand{\PI}{\cap_{i \in I} P_i} 
\newcommand{\QI}{\cap_{i \in I} Q_i} 
\newcommand{\RI}{\cap_{i \in I} R_i} 
%\renewcommand{\SI}{\cap_{i \in I} S_i}                             %\SI already defined, delete '%' if you want to renew it 
\newcommand{\TI}{\cap_{i \in I} T_i} 
\newcommand{\UI}{\cap_{i \in I} U_i} 
\newcommand{\VI}{\cap_{i \in I} V_i} 
\newcommand{\WI}{\cap_{i \in I} W_i} 
\newcommand{\XI}{\cap_{i \in I} X_i} 
\newcommand{\YI}{\cap_{i \in I} Y_i}
\newcommand{\ZI}{\cap_{i \in I} Z_i}                                

%limit as "letter" tends to infinity 
\newcommand{\limftya}{\lim_{a \to \infty}}                          
\newcommand{\limftyb}{\lim_{b \to \infty}} 
\newcommand{\limftyc}{\lim_{c \to \infty}} 
\newcommand{\limftyd}{\lim_{d \to \infty}} 
\newcommand{\limftye}{\lim_{e \to \infty}} 
\newcommand{\limftyf}{\lim_{f \to \infty}} 
\newcommand{\limftyg}{\lim_{g \to \infty}} 
\newcommand{\limftyh}{\lim_{h \to \infty}}
\newcommand{\limftyi}{\lim_{i \to \infty}}                          
\newcommand{\limftyj}{\lim_{j \to \infty}}                          
\newcommand{\limftyk}{\lim_{k \to \infty}}
\newcommand{\limftyl}{\lim_{l \to \infty}}
\newcommand{\limftym}{\lim_{m \to \infty}}
\newcommand{\limftyn}{\lim_{n \to \infty}}                          
\newcommand{\limftyo}{\lim_{o \to \infty}} 
\newcommand{\limftyp}{\lim_{p \to \infty}} 
\newcommand{\limftyq}{\lim_{q \to \infty}}
\newcommand{\limftyr}{\lim_{r \to \infty}}
\newcommand{\limftys}{\lim_{s \to \infty}}
\newcommand{\limftyt}{\lim_{t \to \infty}}
\newcommand{\limftyu}{\lim_{u \to \infty}}
\newcommand{\limftyv}{\lim_{v \to \infty}}
\newcommand{\limftyw}{\lim_{w \to \infty}} 
\newcommand{\limftyx}{\lim_{x \to \infty}}                          
\newcommand{\limftyy}{\lim_{y \to \infty}}                          
\newcommand{\limftyz}{\lim_{z \to \infty}} 

%matrices 
\newcommand{\colTwo}[2]{\begin{pmatrix}
                        #1 \\ #2 
                        \end{pmatrix}}                              %length-two column vector 
\newcommand{\colThree}[3]{\begin{pmatrix}
                        #1 \\ #2 \\ #3
                        \end{pmatrix}}                              %length-three column vector 
\newcommand{\colFour}[4]{\begin{pmatrix}
                        #1 \\ #2 \\ #3 \\ #4 
                        \end{pmatrix}}                              %length-four column vector 
\newcommand{\colFive}[5]{\begin{pmatrix}
                        #1 \\ #2 \\ #3 \\ #4 \\ #5
                        \end{pmatrix}}                              %length-five column vector 
\newcommand{\diagTwo}[2]{\begin{pmatrix} 
                         #1 &\, \\ 
                         \, &#2 \\ 
                         \end{pmatrix}}                             %two-by-two diagonal matrix 
\newcommand{\diagThree}[3]{\begin{pmatrix} 
                           #1 &\, &\, \\ 
                           \, &#2 &\, \\ 
                           \, &\, &#3 \\ 
                           \end{pmatrix}}                           %three-by-three diagonal matrix 
\newcommand{\diagFour}[4]{\begin{pmatrix} 
                           #1 &\, &\, &\, \\ 
                           \, &#2 &\, &\, \\ 
                           \, &\, &#3 &\, \\ 
                           \, &\, &\, &#4 \\ 
                           \end{pmatrix}}                           %four-by-four diagonal matrix }
\newcommand{\maTwo}[4]{\begin{pmatrix} 
                        #1 &#2 \\  
                        #3 &#4 \\ 
                       \end{pmatrix}}                               %two-by-two matrix
\newcommand{\maTwoThree}[6]{\begin{pmatrix} 
                            #1 &#2 &#3 \\ 
                            #4 &#5 &#6 \\ 
                            \end{pmatrix}}                          %two-by-three matrix
\newcommand{\maThreeTwo}[6]{\begin{pmatrix}
                            #1 &#2 \\ 
                            #3 &#4 \\ 
                            #5 &#6 \\ 
                            \end{pmatrix}}                          %three by two matrix 
\newcommand{\maThree}[9]{\begin{pmatrix}
                         #1 &#2 &#3 \\ 
                         #4 &#5 &#6 \\ 
                         #7 &#8 &#9 \\ 
                         \end{pmatrix}}                             %three by three matrix

%determinants 
\newcommand{\deTwo}[4]{\det \begin{pmatrix}
                            #1 &#2 \\ 
                            #3 &#4 \\ 
                            \end{pmatrix}}                          %determinant of two by two matrix 
\newcommand{\deThree}[9]{\det \begin{pmatrix}
                              #1 &#2 &#3 \\ 
                              #4 &#5 &#6 \\ 
                              #7 &#8 &#9 \\ 
                              \end{pmatrix}}                        %determinant of three by three matrix 

%span 
\renewcommand{\span}{\text{span }}                                  %the linear span of a set
\newcommand{\spans}[1]{\text{span} \{#1\}}                          %the linear span of elements 

%Topology
\newcommand{\hBall}[2]{B_{#1}\left(#2\right)}                   %B_#1 (#2): open ball centered at #2 with radius #1 
\ctexset{today=old}
\usetheme[dove]{Boadilla}
\usecolortheme{dolphin}
\useoutertheme{miniframes}
\begin{document}
    \usebackgroundtemplate{\tikz\node[opacity=0.25]{
    \includegraphics[width=\paperwidth,
    height=\paperheight]{hamster.jpg}
    };}
\begin{titlepage}
    \begin{center}
        VE203 - Discrete Mathmatics 
    \end{center}
\end{titlepage}
\myfont
\section{Divisibility}
\begin{frame}
    \frametitle{Divisibility}
    \sffamily\blue{Definition}\\
    \hh \myfont
        Let $n,d \in \bZ$, we say that $d$ divides n 
    if $n = dk$ for some $k \in \bZ$. That is
        $$d \mid n \LRarrow \exists k\in \bZ ~(n=dk)$$
    \hh
        We can see that $\mid$ is a \textbf{pre-order} on 
    $\bZ$ but $\left(\bN,\mid\right)$ is a \textbf{poset}. \red{(Why?)} For
    simplicity we will discuss divisibility on natural numbers more in
    number theory. \red{Zero?}
    \\ \vv \yellow{Properties:}\\
    \begin{itemize}
        \item $a\mid a$
        \item $a\mid b \wedge b\mid c \Rarrow a \mid c$
        \item $a \mid b \wedge b \mid a \Rarrow a=\pm b$
    \end{itemize}
\end{frame}
\newcommand{\mydef}[1]{\sffamily\blue{#1}\myfont\\}
\begin{frame}
    \frametitle{Prime Numbers}
    \mydef{Definition}
    \hh \myfont A natural number $p \in \bN$ is a \textbf{prime number} if $p \geq 2$ and is
    divisible by $1$ and itself only. The set of primes is denoted by $\bP$.\\\vv
    \mydef{Unique Factorization}
    \hh \myfont Every positive integer $n \geq 2$ can be \textbf{uniquely} expressed 
    in the form
        $$n = \prod_{i=1}^k p_i ^ {\alpha_i},~p_i\in \bP ,~ \alpha_i \in \bZ^{+}$$
    \\
    \mydef{Typical Example}
    \hh Better to memoerize if you can:
    \begin{itemize}
        \item Fermat Primes
        \item Mersenne Primes
    \end{itemize}
\end{frame}
\begin{frame}
    \frametitle{Infinitude of Prime}
    \mydef{Theorem}
    \hh There are \textit{infinitely} many primes.
    \begin{itemize}
        \item \yellow{Proof of Euclid.
        \item Proof by contraposition}
    \end{itemize}
    \vs{0.3em}
    \mydef{Theorem (Dirichlet)}
    \hh There are \textit{infinitely} many primes of the 
    form $an + b$ where $a, b$ are coprime. Speical cases include:
    \begin{itemize}
        \item $2n+1$
        \item $3n+2$ (Consider the product of all primes of such form)
        \item $4n+1$ (Slides 235)
        \item $4n-1$ (How to construct the product?)
    \end{itemize}
    \vs{0.3em}
    \mydef{Distance Grows}
    \hh For any large $n$, there exists successive $n$ numbers that are not prime.
\end{frame}
\begin{frame}
    \frametitle{Exericse}
    1. Prove part of \textbf{Wilson's Theorem}: \\\vs{0.3em}
    \hh Let $p>1$,  $p$ is a not prime iff $p \mid (p-1)!$.
    \pause
    \\\vv \yellow{Solution:}\\
    \begin{itemize}
        \item If $p$ is a prime, then $p$ is not divisible 
        by any number from $1$ to $p-1$, so $p \nmid (p-1)!$.
        \item If $p$ is not prime and is a square, $p=k^2, k>2$. Since $k<p-1$, $2k<p-1$,
         we conclude that $k\cdot (2k) \mid (p-1)!$.
        \item Otherwise, there exists two factors $a,b, a\neq b$, 
        so that $p=ab$. Since $a<p-1$, $b<p-1$, we have $p=ab\mid (p-1)!$.
    \end{itemize}
\end{frame}
\section{GCD}
\begin{frame}
    \frametitle{Division Algorithm}
    \mydef{Theorem ((Long) Division Algorithm)}
    \hh Given $m, n \in \bN \cut \{0\}$, there \textbf{exist} \textit{unique} 
    integers $q$ and $r$ with $q \geq 0$ and $0 \leq r < m$ so that $n = qm + r$.
    \\\vs{0.5em}\large{\red{The proof may sounds simple, but it is meaningful!}}
    \\\vs{0.5em}\yellow{Uniqueness:}
    $$m \mid (r_1-r_2) \wedge |r_1-r_2| < m\Rarrow F$$
    \yellow{Existence:}
    $$k=qm+r \Rarrow k+1 =
        \begin{cases}
            qm + (r+1) \text{, if }r+1<m\\
            q(m+1) + 0 \text{, if }r+1=m
        \end{cases}
    $$
    Also pay attention to the case that
    $1= 1 \cdot 1+0 \text{ or } 1 = 0 \cdot m +1$
\end{frame}
\begin{frame}
    \frametitle{Greatest Common Divisor}
    \mydef{Definition}
    \hh Let $a, b \in \bZ \cut \{0\}$, The greatest common divisor of $a$ and $b$,
    denoted by $\gcd(a, b)$, is the greatest positive integer $d$ such that
    $d\mid a \wedge d \mid b$.\\
    \vv
    \hh Notice that $$(N, \mid, \wedge := \gcd, \vee := (a, b) \mapsto \frac{ab}{\gcd(a,b)}
    )$$ is a \textbf{lattice} where $\top  = 0$ and $\bot = 1$.
    \\ \vv
    \mydef{Least Common Multiple}
    $$\operatorname{lcm}(a,b)=\frac{ab}{\gcd(a,b)}$$
\end{frame}
\begin{frame}
    \frametitle{Euclidean Algorithm}
    \hh A recursive algorithm to calculate the greatest
    common divisor between two integers:
    $$\operatorname{gcd}(a, b)= \begin{cases}\operatorname{gcd}(b, r\,(a, b)) & b \neq 0 \\ a & b=0\end{cases}$$
    \\ \hh *Interesting names: {\kaishu 辗转相除法、更相减损术}\\\vv
    \mydef{Theorem (Bezout)}
    \hh The equation $ax + by = c$ of $x$ and $y$ where $a, b, c \in \bZ$ has integer
    solutions iff $\gcd(a, b)\mid c$.
\end{frame}
\begin{frame}
    \frametitle{Exercise}
    2. Let $F_n$ be Fermat Primes, i.e. $F_n=2^{2^n} +1$. Prove that they are 
    pairwise coprime, namely $\gcd(F_n,F_m)=1.$
    \\\vs{5em}
    3. Use the \textbf{Euclidean Algorithm} to find a integer pair $(x,y)$ that 
    $111x-321y=75.$ 
\end{frame}
\section{Group Def.}
\begin{frame}
    \frametitle{Groups}
    \fbox{
    \parbox{0.95\textwidth}{
	\par A \textbf{group} is a pair $(G, \cdot)$, where $G$ is a set, and $\cdot : G \times G \to G$ is a law of composition that has the following properties:
	\begin{itemize}
        \item[-] \blue{Closure:} The generalized product is defined as $\cdot: G\times G \to G$
		\item[-] \blue{Associativety:} $(a \cdot b) \cdot c = a \cdot (b \cdot c)$ for all $a, b, c \in G$;
		\item[-] \blue{Identity:} $G$ contains an identity element 1, such that $1 \cdot a = a \cdot 1 = a$ for all $a \in G$;
		\item[-] \blue{Inverse:} Every element $a \in G$ has an inverse, an element $b$ such that $a \cdot b = b \cdot a = 1$.
	\end{itemize}
	\par \phantom{ji}
	\par An \textbf{abelian group} is a group whose law of composition is commutative $(a \cdot b = b \cdot a)$. 
	} 
    }
\end{frame}
\begin{frame}
    \frametitle{Properties}
    \fbox{
	\parbox{0.95\textwidth}{
		\par Given a group $G$, $a, b, c \in G$, then
		\begin{itemize}
			\item[-] there exists a \textbf{unique} identity element; 
            \\$\to$ suppose there are two distinct identity $i$ and $j$, then $i \cdot j = i = j$ 
			\item[-] $ba = ca \Rightarrow b = c$ and $ab = ac \Rightarrow b = c$;  \\$\to$ multiply by $a^{-1}$ on both sides, note that a group does not necessarily satisfy the commutative law
			\item[-] For all $a \in G$, there exists a unique element $b \in G$ such that $ab = ba = 1$; \\ $\to$ prove existence ($b = a^{-1}$) first, then prove uniqueness by contradiction
			\item[-]  $(ab)^{−1} = b^{-1}a^{-1}$. \\$\to$$(ab)^{−1}(ab) = 1$; $(b^{-1}a^{-1})(ab) = b^{-1}(a^{-1}a)b = 1$	
		\end{itemize}
	}
    }
\end{frame}
\begin{frame}
    \frametitle{Subgroup}
    \fbox{
	\parbox{0.95\textwidth}{
		\par A subset $H$ of a group $G$ is a subgroup if it has the following properties:
		\begin{itemize}
			\item[-] Closure: If $a, b \in H$, then $ab \in H$;
			\item[-] Identity: $1 \in H$;
			\item[-] Inverses: If $a \in H$, then $a^{-1} \in H$.
		\end{itemize}
		\par Look carefully at the identity and inverse axioms for a subgroup:
		\begin{itemize}
			\item[-] In verifying the identity axiom for a subgroup, the issue is not the existence of an identity 
            but \textbf{whether the identity for the group is actually contained in the subgroup}.
			\item[-] Likewise, for subgroups the issue of inverses is not whether inverses exist (every element of a group has an inverse) 
            but \textbf{whether the inverse of an element in the subgroup is actually contained in the subgroup}.
		\end{itemize}
	}
}
\end{frame}
\begin{frame}
    \frametitle{Exercise}
    4.  Given a group $G$ and its two distinct subgroups $H_1$ and $H_2$. 
    Check whether the following sentences are true or false:
    \begin{itemize}
        \item The identity element in $G$ and $H_1$ must be the same.
        \item $H_1\cup H_2 $ is a group.
        \item $H_1 \cap H_2$ cannot be empty and it is a group.
        \item A subset in $G$ that is not a subgroup may be a group.
    \end{itemize}
    \vv \hh
    \textit{Comment.} Compare to the concept of \textbf{vector space}.
    If necessary, take Vv186.
\end{frame}
\begin{frame}
    \frametitle{Exercise}
    5. $\mathbb{Z}^2 = \mathbb{Z} \times \mathbb{Z}$ denotes the set of pairs of integers: 
    $$ \mathbb{Z}^2 = \{(m, n) \mid m, n \in \mathbb{Z}\}.$$ 
    It is a group under ``vector addition'', that is, 
    $$(a, b) + (c, d) = (a + c, b + d).$$ 
    Consider the set $$H = \{(x, y) \mid x + y \geqslant 0\}.$$ 
    Check if $H$ is a subgroup of $\mathbb{Z}^2$.
\end{frame}
\begin{frame}
    \frametitle{Exercise}
    6. Let $G$ be a group and let $H \subset G$ with $H \neq \varnothing$. 
    If $\forall a, b \in H$ we have $ab^{-1} \in H$ then $H$ is a subgroup of $G$.
    \pause
    \\\vv\yellow{Solution:}\\
    \hh Since $H \subset G$, any operation in $H$ has associativity. Then, we need to verify closure, identity, and inverses requirements but we need to
    do these in a particular order. 
    \begin{enumerate}
        \item  Since $H \neq \varnothing$, pick any $a \in H$. Then $aa^{-1} = e \in H$, so $H$ has the identity.
        \item Pick any $a \in H$. Since the identity $e \in H$, then $ea^{-1} = a^{-1} \in H$ so we have inverses. 
        \item Pick any $a, b \in H$. Then $b^{-1} \in H$ and denote as $c \in H$. So, $ac^{-1} \in H$ according to the problem statement. So, $ab = a(b^{-1})^{-1} = ac^{-1} \in H$ and we have closure.	
    \end{enumerate}
\end{frame}
\begin{frame}
    \frametitle{Exercise}
    7.$^*$ Let $G$ be a group. If $\forall x \in G : x^2 = e$, show that $G$ is an abelian group.
    \\ \vv
    \yellow{Solution:}
    \\ \vv
    From $\forall x \in G : x^2 = e$, we obtain $x = x^{-1}$.
    \par Therefore, taking $\forall x,y \in G$, we have $$xy = (xy)^{-1} = y^{-1}x^{-1} = yx.$$
    \par This completes the proof.
    
\end{frame}
\section{Cyclic Group}
\begin{frame}
    \frametitle{Cyclic Group}
    \fbox{
	\parbox{0.95\textwidth}{
	\par The cyclic subgroup generated by $g$ is
	$$
	\left\langle g \right\rangle = \{g^k \mid k \in \mathbb{Z}\}.
	$$
	\par In other words, $\left\langle g \right\rangle$ consists of all (positive or negative) powers of $g$.
	$$
	\left\langle g \right\rangle = \{k \cdot g \mid k \in \mathbb{Z}\}.
	$$
	\par Be sure you understand that the difference between the two forms is simply notational: It’s the same concept.
	\par \phantom{ji}
    \par Let $G$ be a group, $g \in G$. The order of $g$ is the smallest positive 
    integer $n$ such that $g^n = 1$ $(ng = 0)$. 
    If there is no positive integer $n$ such that $g^n = 1 $ $(ng = 0)$, 
    then $g$ has \textbf{infinite} order.
	}
}
\end{frame}
\begin{frame}
    \frametitle{Exercise}
    8. List the elements of the subgroups generated by 
    elements of $\mathbb{Z}_8 = \{0,1,2,3,4,5,6,7\}$.
    \pause
    \\\vv \yellow{Solution:}
    $$
    \begin{aligned}
    &\langle 0\rangle=\{0\} \\
    &\langle 2\rangle=\langle 6\rangle=\{0,2,4,6\} \\
    &\langle 4\rangle=\{0,4\} \\
    &\langle 1\rangle=\langle 3\rangle=\langle 5\rangle=\langle 7\rangle=\{0,1,2,3,4,5,6,7\} = \mathbb{Z}_8
    \end{aligned}
    $$
    \begin{block}{Question}
        \hh What is the identity? Order?
    \end{block}
\end{frame}
\begin{frame}
    \frametitle{Exercise}  
    9. Prove that
    \begin{enumerate}
        \item Let $G = \left\langle g \right\rangle$ be a finite cyclic group, where $g$ has order $n \neq 0$. Then the powers $\{1, g, \cdots , g^{n−1}\}$ are distinct.
        \item Let $G = \left\langle g\right\rangle $ be infinite cyclic. If $m$ and $n$ are integers and $m \neq n$, then $g^m \neq g^n$.
    \end{enumerate}
    \pause
    \yellow{Solution:}\\
    \begin{enumerate}
        \item Since $g$ has order $n$, $g, g^2, \cdots, g^{n−1}$ are all different from 1.
        \par Suppose $g^i = g^j$ where $0 \leqslant j < i < n$. Then $0 < i − j < n$ and $g^{i−j} = 1$, contrary to the preceding observation.
        \par Therefore, the powers $\{1, g, \cdots , g^{n−1}\}$ are distinct.
        \item Suppose without loss of generality that $m > n$. We want to show that $g^m \neq g^n$.
        \par Suppose this is false, so  $g^m = g^n$. Then $g^{m−n} = 1$, so $g$ has finite order $m-n$. This contradicts the fact that a	generator of an infinite cyclic group has infinite order. Therefore,  $g^m \neq g^n$.
    \end{enumerate}
\end{frame}
\section{*Extra Topic}
\begin{frame}
    \frametitle{Divide \& Conquer}
    \red{This is just standby. If we really have time.}
    \\
    Recall Manuel's h5 ex6:
    \begin{enumerate}
        \item Detail Karatsuba algorithm in the README file (search it on internet).
        \item Add comments to the code to describe what is done, line by line.
        \item Explain in the README file what specific adjustments were made to the algorithm in order to
        improve the efficiency.
        \item Search online what is a divide an conquer strategy.
        \item Using a divide and conquer approach, together with the operators \&, $\mid$, << and >> , write an
        efficient function to replace the for loops marked as ``not optimal''.
    \end{enumerate}
    Link: \url{https://www.bilibili.com/video/BV1jS4y197PV}
\end{frame}
\definecolor{mygreen}{rgb}{0,0.6,0}
\definecolor{mygray}{rgb}{0.5,0.5,0.5}
\definecolor{mymauve}{rgb}{0.58,0,0.82}
\definecolor{mypurple}{rgb}{0.58,0.02,0.82}
\definecolor{myblue}{rgb}{0.1,0.2,0.9}
\definecolor{myorange}{rgb}{0.73,0.38,0.17}
% https://en.wikibooks.org/wiki/LaTeX/Source_Code_Listings
\lstset{%
	language=C++,
	backgroundcolor=\color{white},
	basicstyle=\tiny,
	breakatwhitespace=false,
	breaklines=true,
	captionpos=t,
	commentstyle=\color{mygreen},
	deletekeywords={...},
	escapeinside={\%*}{*)},
	extendedchars=true,
	frame=single,
	keepspaces=true,
	keywordstyle=\color{blue},
	language=Octave,
	%otherkeywords={*,...},
	numbers=left,
	numbersep=5pt,
	numberstyle=\tiny\color{mygray},
	rulecolor=\color{black},
	showspaces=false,
	showstringspaces=false,
	showtabs=false,
	stepnumber=1,
	stringstyle=\color{mymauve},
	tabsize=4,
	title=\lstname
}

\lstdefinestyle{customcpp}{
	belowcaptionskip=0pt,
	breaklines=true,
	%frame=L,
	%xleftmargin=\parindent,
	language=C++,
	showstringspaces=false,
	basicstyle=\scriptsize\ttfamily,
	keywordstyle=\bfseries\color{mypurple},
	commentstyle=\itshape\color{green!40!black},
	identifierstyle=\color{myblue},
	stringstyle=\color{myorange},
}

\lstset{escapechar=@,style=customcpp}
\begin{frame}[fragile]
    \frametitle{Probelm}
    \hbox{
    \begin{lstlisting}
    unsigned long int mult(unsigned long int a, unsigned long int b) {
        int i, n, N;
        unsigned long int x0,y0,z0,z1=1;
        if(a<b) SWAP(a,b);
        if(b==0) return 0;
        for(n=-1, i = 1; i <= b; i<<=1, n++); /* not optimal */
        for(N=n; i <= a; i<<=1, N++);
        
        y0=b&((1<<n)-1);
        x0=a&((1<<N)-1);
        z0=mult(x0,y0);
        i=N+n;
        return ((z1<<i)+(x0<<n)+(y0<<N)+z0);
    }\end{lstlisting}
    }
\end{frame}
\begin{frame}
    \frametitle{Reference}
    \begin{itemize}
        \item Problem from Vg101 Manuel Homework 5.
        \item Examples from Dr. Cai Runze's Sildes.
        \item Exercises from 2021-Fall-Ve203 TA Zhao Jiayuan
        \item Contents from 2021-Fall-Ve203 Mid\_2 RC by Xue Runze
    \end{itemize}
\end{frame}
\end{document}