\documentclass{beamer}
\newcommand{\myfont}{\rmfamily\normalsize\upshape\mdseries}
\newcommand{\degree}{^\circ}
\title{\sffamily Review V(Slides 280 - 311)}
\subtitle{\textbf{Homormophism \& Coset }\\Hold on! This is difficult!}
\institute[UM-SJTU JI]{University of Michigan-Shanghai Jiao Tong University Joint Institute}
\author{HamHam}
\usepackage{graphicx}
\usepackage{picinpar}
\usepackage{indentfirst}
\usepackage{chemformula}
\usepackage{geometry}
\usepackage{subfigure}
\usepackage{appendix}
\usepackage{amsfonts}
\usepackage{enumerate}
\usepackage{float}
\usepackage{geometry}
\usepackage{latexsym}
\usepackage{listings}
\usepackage{multicol,multirow,multido}
\usepackage{tabularx}
\usepackage{ulem}
\usepackage{tikz}
\usepackage{xcolor}
\usepackage{cite}
\usepackage{setspace}
\usepackage{hyperref}
\usepackage{textpos}
\usepackage{booktabs}
\usepackage{diagbox}
\usepackage{listings}
\usepackage{graphics}
\usepackage{upgreek}
\usepackage{JI_MathCourse_Notations}
\usepackage{mathrsfs}
\usepackage[OT2,OT1]{fontenc}
%%%%%%%%%%%%%%%%%%%%%%%%%%%%%%%%%%%%%%%%%%%%%%%%%%%%%%%%%%%%%%%%%%%%%%%%%%%%%%
\iffalse 
Copyright 2021 by Leyang Zhang, Yinchen Ni, Yuxiang Chen, Yue Huang

You should not spread this file without the permission from every one of 
    Leyang Zhang, Yinchen Ni and Yuxiang Chen 
\fi
%%%%%%%%%%%%%%%%%%%%%%%%%%%%%%%%%%%%%%%%%%%%%%%%%%%%%%%%%%%%%%%%%%%%%%%%%%%%%

\ProvidesPackage{JI_MathCourse_Notations}[2021/12/19]

\RequirePackage{amsmath}
\RequirePackage{amsthm}
\RequirePackage{amssymb}
\RequirePackage{bm}
\RequirePackage{bbm}
\RequirePackage{color}


%Typesetting

%colors 
\newcommand{\blue}[1]{\textcolor{blue}{#1}}                         %blue
\newcommand{\cha}[1]{\textcolor[rgb]{0.95,0.75,0.9}{#1}}            %cha-type color 
\newcommand{\gray}[1]{\textcolor{gray}{#1}}                         %grey
\newcommand{\pink}[1]{\textcolor{pink}{#1}}                         %pink
\newcommand{\red}[1]{\textcolor[rgb]{0.75,0,0}{#1}}                 %red
\newcommand{\yellow}[1]{\textcolor{orange}{#1}}                     %yellow

%declarations of Math lines, turn off code checker before using them
\newcommand{\beq}{\begin{equation}}
\newcommand{\beqa}{\begin{equation}\begin{aligned}}
\newcommand{\beqNo}{\begin{equation*}}
\newcommand{\beqaNo}{\begin{equation*}\begin{aligned}}
\newcommand{\eeq}{\end{equation}}
\newcommand{\eeqa}{\end{aligned}\end{equation}}
\newcommand{\eeqNo}{\end{equation*}}
\newcommand{\eeqaNo}{\end{aligned}\end{equation*}}

%space 
\newcommand{\hh}{\hspace{1em}}
\newcommand{\hs}[1]{\hspace{#1}}
\newcommand{\vs}[1]{\vspace{#1}}
\newcommand{\vv}{\vspace{1em}}


%-------------------------------------------------------------------


%Letter-like Notations 

%A - Z notations 
\newcommand{\bA}{\mathbb{A}}    \newcommand{\calA}{\mathcal{A}}     \newcommand{\fA}{\mathfrak{A}}
\newcommand{\bB}{\mathbb{B}}    \newcommand{\calB}{\mathcal{B}}     \newcommand{\fB}{\mathfrak{B}} 
\newcommand{\bC}{\mathbb{C}}    \newcommand{\calC}{\mathcal{C}}     \newcommand{\fC}{\mathfrak{C}} 
\newcommand{\bD}{\mathbb{D}}    \newcommand{\calD}{\mathcal{D}}     \newcommand{\fD}{\mathfrak{D}} 
\newcommand{\bE}{\mathbb{E}}    \newcommand{\calE}{\mathcal{E}}     \newcommand{\fE}{\mathfrak{E}} 
\newcommand{\bF}{\mathbb{F}}    \newcommand{\calF}{\mathcal{F}}     \newcommand{\fF}{\mathfrak{F}} 
\newcommand{\bG}{\mathbb{G}}    \newcommand{\calG}{\mathcal{G}}     \newcommand{\fG}{\mathfrak{G}} 
\newcommand{\bH}{\mathbb{H}}    \newcommand{\calH}{\mathcal{H}}     \newcommand{\fH}{\mathfrak{H}} 
\newcommand{\bI}{\mathbb{I}}    \newcommand{\calI}{\mathcal{I}}     \newcommand{\fI}{\mathfrak{I}} 
\newcommand{\bJ}{\mathbb{J}}    \newcommand{\calJ}{\mathcal{J}}     \newcommand{\fJ}{\mathfrak{J}} 
\newcommand{\bK}{\mathbb{K}}    \newcommand{\calK}{\mathcal{K}}     \newcommand{\fK}{\mathfrak{K}}  
\newcommand{\bL}{\mathbb{L}}    \newcommand{\calL}{\mathcal{L}}     \newcommand{\fL}{\mathfrak{L}} 
\newcommand{\bM}{\mathbb{M}}    \newcommand{\calM}{\mathcal{M}}     \newcommand{\fM}{\mathfrak{M}} 
\newcommand{\bN}{\mathbb{N}}    \newcommand{\calN}{\mathcal{N}}     \newcommand{\fN}{\mathfrak{N}}  
\newcommand{\bO}{\mathbb{O}}    \newcommand{\calO}{\mathcal{O}}     \newcommand{\fO}{\mathfrak{O}} 
\newcommand{\bP}{\mathbb{P}}    \newcommand{\calP}{\mathcal{P}}     \newcommand{\fP}{\mathfrak{P}} 
\newcommand{\bQ}{\mathbb{Q}}    \newcommand{\calQ}{\mathcal{Q}}     \newcommand{\fQ}{\mathfrak{Q}} 
\newcommand{\bR}{\mathbb{R}}    \newcommand{\calR}{\mathcal{R}}     \newcommand{\fR}{\mathfrak{R}} 
\newcommand{\bS}{\mathbb{S}}    \newcommand{\calS}{\mathcal{S}}     \newcommand{\fS}{\mathfrak{S}}     
\newcommand{\bT}{\mathbb{T}}    \newcommand{\calT}{\mathcal{T}}     \newcommand{\fT}{\mathfrak{T}} 
\newcommand{\bU}{\mathbb{U}}    \newcommand{\calU}{\mathcal{U}}     \newcommand{\fU}{\mathfrak{U}} 
\newcommand{\bV}{\mathbb{V}}    \newcommand{\calV}{\mathcal{V}}     \newcommand{\fV}{\mathfrak{V}} 
\newcommand{\bW}{\mathbb{W}}    \newcommand{\calW}{\mathcal{W}}     \newcommand{\fW}{\mathfrak{W}} 
\newcommand{\bX}{\mathbb{X}}    \newcommand{\calX}{\mathcal{X}}     \newcommand{\fX}{\mathfrak{X}} 
\newcommand{\bY}{\mathbb{Y}}    \newcommand{\calY}{\mathcal{Y}}     \newcommand{\fY}{\mathfrak{Y}} 
\newcommand{\bZ}{\mathbb{Z}}    \newcommand{\calZ}{\mathcal{Z}}     \newcommand{\fZ}{\mathfrak{Z}} 

%Math-specific notations 

\newcommand{\ep}{\epsilon}                                          %epsilon
\newcommand{\vep}{\varepsilon}                                      %variable epsilon


%-------------------------------------------------------------------


%Common Operations 

%sets 
\renewcommand{\Cap}{\bigcap}                                        %A intersect with B 
\newcommand{\card}{\text{card }}                                    %cardinality of a set
\renewcommand{\Cup}{\bigcup}                                        %A union with B 
\newcommand{\cut}{\backslash}                                       %A \ B 
\newcommand{\es}{\approx}                                           %equinumerosity between two sets 
\newcommand{\nes}{\not\approx}                                      %not Equinumerosity 
\newcommand{\siq}{\subseteq}                                        %A is contained in B 
\newcommand{\soq}{\supseteq}                                        %A is out of range of B 
\newcommand{\symd}{\Delta}                                          %symmetric difference 

%functions and function-related operations 
\newcommand{\ceil}[1]{\lceil {#1} \rceil}                           %smallest integer no less than #1 
\newcommand{\dist}[2]{\text{dist}\left ({#1},{#2} \right )}         %Distance function 
\newcommand{\floor}[1]{\lfloor {#1} \rfloor}                        %largest integer no greater than #1 
\newcommand{\inv}[1]{{#1}^{-1}}                                     %the inverse of a function, a number or a group element
\renewcommand{\mod}[1]{\,(\text{mod } {#1})}                        %modular of an integer #1 
\newcommand{\ran}{\text{ran }}                                      %the range of a function
\newcommand{\inflim}[1]{\varliminf_{#1}}                            %limit infimum of sequence or set
\newcommand{\limc}[2]{\lim_{#1 \to #2}}                             %limit 
\newcommand{\suplim}[1]{\varlimsup_{#1}}                            %limit supremum of sequence or set

%Others
\newcommand{\larrow}{\leftarrow}                                    %single left arrow, implies
\newcommand{\rarrow}{\rightarrow}                                   %single right arrow, implies 
\newcommand{\Larrow}{\Leftarrow}                                    %double left arrow, implies 
\newcommand{\Rarrow}{\Rightarrow}                                   %double right arrow, implies 
\newcommand{\lrarrow}{\leftrightarrow}                              %single left-right arrow 
\newcommand{\LRarrow}{\Leftrightarrow}                              %double left-right arrow, shorter than \iff 
\newcommand{\lowBrace}[2]{\underset{#1}{\underbrace{#2}}}           %under-brace with notes 
\renewcommand{\(}{\left (}                                          %left-half bracket, turn off code checker before using it 
\renewcommand{\)}{\right )}                                         %right-half bracket, turn off code checker before using it 
\newcommand{\<}{\langle}                                            %left-half angular bracket 
\renewcommand{\>}{\rangle}                                          %right-half angular bracket


%-------------------------------------------------------------------


%Patches  

%Analysis and Calculus 
\newcommand{\df}[2]{\frac{d{#1}}{d{#2}}}                            %one-variable differential operator
%\newcommand{\indf}[2]{\dfrac{d{#1}}{d{#2}}}                         %differential operator for in-line math mode
\newcommand{\im}{\text{Im }}                                        %taking the imaginary part of a number or a function 
\newcommand{\inrPdct}[2]{\left \langle{#1},{#2}\right \rangle}      %inner product 
\newcommand{\nDf}[2]{{#1}^{({#2})}}                                 %n-th derivative of some function 
\newcommand{\norm}[1]{\left \|#1\right \|}                                       %norm 
\newcommand{\parf}[2]{\frac{\partial{#1}}{\partial{#2}}}            %partial derivative of a function 
\newcommand{\parfwide}[2]{\partial{#1} / \partial{#2}}              %partial derivative of a function, wide form 
\newcommand{\re}{\text{Re }}                                        %taking the real part of a number or a function 

%Combinatorics 
\newcommand{\al}{{\aleph}_{0}}                                      %Aleph zero 
\newcommand{\all}{{\aleph}_{1}}                                     %Aleph one
\newcommand{\dbinomial}[2]{\begin{pmatrix}\!\!
                           \begin{pmatrix}{#1}\\{#2} 
                           \end{pmatrix}
                           \!\!\!
                           \end{pmatrix}}                           %double binomial 
\newcommand{\kpermu}[2]{{#1}^\text{\underline{#2}}}                 %k-permutation 
\newcommand{\logs}{\log^{*}}                                        %Iterated Logarithm 
\newcommand{\stirling}[2]{\left \{\begin{aligned} 
                                    {#1} \\ {#2} 
                                  \end{aligned} 
                          \right \}}                                %Stirling numbers of second kind 

%Graph Theory 
\newcommand{\comp}[1]{\text{comp} \left({#1}\right)}                %Component of graph 
\newcommand{\conj}{\overline}                                       %Complement graph of a given graph

%intersection of a collection of sets indexed by I 
\newcommand{\AI}{\cap_{i \in I} A_i} 
\newcommand{\BI}{\cap_{i \in I} B_i} 
\newcommand{\CI}{\cap_{i \in I} C_i} 
\newcommand{\DI}{\cap_{i \in I} D_i} 
\newcommand{\EI}{\cap_{i \in I} E_i} 
\newcommand{\FI}{\cap_{i \in I} F_i} 
\newcommand{\GI}{\cap_{i \in I} G_i} 
\newcommand{\HI}{\cap_{i \in I} H_i} 
\newcommand{\II}{\cap_{i \in I} I_i} 
\newcommand{\JI}{\cap_{i \in I} J_i} 
\newcommand{\KI}{\cap_{i \in I} K_i} 
\newcommand{\LI}{\cap_{i \in I} L_i} 
\newcommand{\MI}{\cap_{i \in I} M_i} 
\newcommand{\NI}{\cap_{i \in I} N_i} 
\newcommand{\OI}{\cap_{i \in I} O_i} 
\newcommand{\PI}{\cap_{i \in I} P_i} 
\newcommand{\QI}{\cap_{i \in I} Q_i} 
\newcommand{\RI}{\cap_{i \in I} R_i} 
%\renewcommand{\SI}{\cap_{i \in I} S_i}                             %\SI already defined, delete '%' if you want to renew it 
\newcommand{\TI}{\cap_{i \in I} T_i} 
\newcommand{\UI}{\cap_{i \in I} U_i} 
\newcommand{\VI}{\cap_{i \in I} V_i} 
\newcommand{\WI}{\cap_{i \in I} W_i} 
\newcommand{\XI}{\cap_{i \in I} X_i} 
\newcommand{\YI}{\cap_{i \in I} Y_i}
\newcommand{\ZI}{\cap_{i \in I} Z_i}                                

%limit as "letter" tends to infinity 
\newcommand{\limftya}{\lim_{a \to \infty}}                          
\newcommand{\limftyb}{\lim_{b \to \infty}} 
\newcommand{\limftyc}{\lim_{c \to \infty}} 
\newcommand{\limftyd}{\lim_{d \to \infty}} 
\newcommand{\limftye}{\lim_{e \to \infty}} 
\newcommand{\limftyf}{\lim_{f \to \infty}} 
\newcommand{\limftyg}{\lim_{g \to \infty}} 
\newcommand{\limftyh}{\lim_{h \to \infty}}
\newcommand{\limftyi}{\lim_{i \to \infty}}                          
\newcommand{\limftyj}{\lim_{j \to \infty}}                          
\newcommand{\limftyk}{\lim_{k \to \infty}}
\newcommand{\limftyl}{\lim_{l \to \infty}}
\newcommand{\limftym}{\lim_{m \to \infty}}
\newcommand{\limftyn}{\lim_{n \to \infty}}                          
\newcommand{\limftyo}{\lim_{o \to \infty}} 
\newcommand{\limftyp}{\lim_{p \to \infty}} 
\newcommand{\limftyq}{\lim_{q \to \infty}}
\newcommand{\limftyr}{\lim_{r \to \infty}}
\newcommand{\limftys}{\lim_{s \to \infty}}
\newcommand{\limftyt}{\lim_{t \to \infty}}
\newcommand{\limftyu}{\lim_{u \to \infty}}
\newcommand{\limftyv}{\lim_{v \to \infty}}
\newcommand{\limftyw}{\lim_{w \to \infty}} 
\newcommand{\limftyx}{\lim_{x \to \infty}}                          
\newcommand{\limftyy}{\lim_{y \to \infty}}                          
\newcommand{\limftyz}{\lim_{z \to \infty}} 

%matrices 
\newcommand{\colTwo}[2]{\begin{pmatrix}
                        #1 \\ #2 
                        \end{pmatrix}}                              %length-two column vector 
\newcommand{\colThree}[3]{\begin{pmatrix}
                        #1 \\ #2 \\ #3
                        \end{pmatrix}}                              %length-three column vector 
\newcommand{\colFour}[4]{\begin{pmatrix}
                        #1 \\ #2 \\ #3 \\ #4 
                        \end{pmatrix}}                              %length-four column vector 
\newcommand{\colFive}[5]{\begin{pmatrix}
                        #1 \\ #2 \\ #3 \\ #4 \\ #5
                        \end{pmatrix}}                              %length-five column vector 
\newcommand{\diagTwo}[2]{\begin{pmatrix} 
                         #1 &\, \\ 
                         \, &#2 \\ 
                         \end{pmatrix}}                             %two-by-two diagonal matrix 
\newcommand{\diagThree}[3]{\begin{pmatrix} 
                           #1 &\, &\, \\ 
                           \, &#2 &\, \\ 
                           \, &\, &#3 \\ 
                           \end{pmatrix}}                           %three-by-three diagonal matrix 
\newcommand{\diagFour}[4]{\begin{pmatrix} 
                           #1 &\, &\, &\, \\ 
                           \, &#2 &\, &\, \\ 
                           \, &\, &#3 &\, \\ 
                           \, &\, &\, &#4 \\ 
                           \end{pmatrix}}                           %four-by-four diagonal matrix }
\newcommand{\maTwo}[4]{\begin{pmatrix} 
                        #1 &#2 \\  
                        #3 &#4 \\ 
                       \end{pmatrix}}                               %two-by-two matrix
\newcommand{\maTwoThree}[6]{\begin{pmatrix} 
                            #1 &#2 &#3 \\ 
                            #4 &#5 &#6 \\ 
                            \end{pmatrix}}                          %two-by-three matrix
\newcommand{\maThreeTwo}[6]{\begin{pmatrix}
                            #1 &#2 \\ 
                            #3 &#4 \\ 
                            #5 &#6 \\ 
                            \end{pmatrix}}                          %three by two matrix 
\newcommand{\maThree}[9]{\begin{pmatrix}
                         #1 &#2 &#3 \\ 
                         #4 &#5 &#6 \\ 
                         #7 &#8 &#9 \\ 
                         \end{pmatrix}}                             %three by three matrix

%determinants 
\newcommand{\deTwo}[4]{\det \begin{pmatrix}
                            #1 &#2 \\ 
                            #3 &#4 \\ 
                            \end{pmatrix}}                          %determinant of two by two matrix 
\newcommand{\deThree}[9]{\det \begin{pmatrix}
                              #1 &#2 &#3 \\ 
                              #4 &#5 &#6 \\ 
                              #7 &#8 &#9 \\ 
                              \end{pmatrix}}                        %determinant of three by three matrix 

%span 
\renewcommand{\span}{\text{span }}                                  %the linear span of a set
\newcommand{\spans}[1]{\text{span} \{#1\}}                          %the linear span of elements 

%Topology
\newcommand{\hBall}[2]{B_{#1}\left(#2\right)}                   %B_#1 (#2): open ball centered at #2 with radius #1 

%\usepackage{ctex} %插入中文
%\ctexset{today=old}

\newcommand{\mydef}[1]{\sffamily\blue{#1}\myfont\\} %for define
\newcommand{\mysol}{\yellow{Solution:}\\}
\usetheme[dove]{Boadilla}
\usecolortheme{dolphin}
\useoutertheme{miniframes}
\begin{document}
    \usebackgroundtemplate{\tikz\node[opacity=0.25]{
    \includegraphics[width=\paperwidth,
    height=\paperheight]{hamster.jpg}
    };}
\begin{titlepage}
    \begin{center}
        VE203 - Discrete Mathmatics 
    \end{center}
\end{titlepage}
\myfont
\section{Symmetric Group}
\begin{frame}
    \frametitle{Symmetric Group}
    \mydef{Definition}
    \hh Given $n \in \mathbb{N} \backslash \{0\}$, we have the following symmetric group of degree $n$,
    $$
    \begin{aligned}
    S_{n} &=\{\text {All permutations on } n \text { letters/numbers}\} \\
    &=\operatorname{Sym}\{1,2,3, \ldots, n\} \\
    &=\{f:[n] \rightarrow[n] \mid f \text { bijective}\}
    \end{aligned}
    $$

    Note that it is a finite group of order $n!$ (the number of bijections from $[n]$ to $[n]$), 
    \textit{i.e.}, $|S_n| = n!$.
    \\
    \vv 
    \begin{itemize}
        \item  A subgroup of $S_n$ is called a \blue{permutation group}.
        \item A permutation of the form $(ab)$ where $a \neq b$ 
        is called a \blue{transposition}. 
    \end{itemize}
\end{frame}

\begin{frame}
    \frametitle{Permutation}
    \hh A permutation that can be expressed as a product of an \textbf{even/odd number 
    of transpositions} is called an even/odd permutation. \\
    \hh The set of even permutations in $S_n$ forms 
    a subgroup of $S_n$, denoted as $A_n$, 
    is called the alternating group of degree $n$. \\\vv

    \yellow{Permutation $\to$ transportation:} $(132)(5648)=(13)(32)(56)(64)(48)$ (not unique, but only can be either all odd or all even).
	\\\vs{0.3em}
	\yellow{Inverse of permutation:} $\sigma=(132)(5648) \ \Rightarrow \ \sigma^{-1}=(8465)(231)$ 
    (Separate permutations to be \textbf{\red{disjoint}} first. Since $\sigma(a_i) = a_j$ implies $\sigma^{-1}(a_j) = a_i$, we only need to reverse the order of the cyclic pattern).
	\\\vs{0.3em}
	\yellow{Composition:} $(12)(245)(13)(125) = (14532).$ \\
    (Apply the \textbf{\red{right}} permutation first. Demo!).
\end{frame}
\begin{frame}
    \frametitle{Exercise}
    1. True or false:
    \begin{itemize}
        \item Can an abelian group have a non-abelian subgroup?
        \item Can a non-abelian group have an abelian subgroup?
        \item Can a non-abelian group have a non-abelian subgroup?
    \end{itemize}
    \vs{3em}
    \yellow{Answer: No; Yes; Yes.}
\end{frame}
\begin{frame}
    \frametitle{Exercise}
        2. Prove the following:
        \begin{enumerate}
            \item $S_n$ is non-abelian for $n\geq 3$;~\\
            \item $A_n$ is a subgroup of $S_n$;~\\
            \item $|A_n|=n!/2.$~\\
        \end{enumerate}
\end{frame}

\section{Homomorphism}
\begin{frame}
    \frametitle{Homomorphism}
    \hh  Given groups $\red{G},\blue{G\,'}$, a homomorphism is a map $f : \red{G} \to \blue{G\,'}$ such that for
		$$f\(x\red{\cdot} y\)=f\(x\) \blue{\cdot} f\(y\)$$
    We have:
    \begin{itemize}
        \item $f\(a_1\cdots a_k \)=f\(a_1\)\cdots f\(a_k\)$
        \item $f\,(1_G)=1_{G\,'}$
        \item $f\(a^{-1}\)=f^{\,-1}\(a\)$
    \end{itemize}
    \mydef{Compare and Contrast}
    \hh Recall the concept of \blue{structure preserving}
    \begin{equation*}
        \begin{aligned}
            &~y \xrightarrow{~~~f~~~}f\(y\) \\
            x\, \cdot &\downarrow      ~~~~~~~~~\downarrow f\(x\)\cdot \\
            x\,\cdot &y\xleftarrow{~~f^{\,-1}~~}f\(x\cdot y\) 
        \end{aligned}        
    \end{equation*}
\end{frame}
\begin{frame}
    \frametitle{Image \& Kernel}
    \hh The \blue{image} of a homomorphism $f : G \to G\,'$, often denoted by $\operatorname{im} f$, or $f\(G\)$, is simply the image of as a map of sets:
    $$
    \operatorname{im} f=\left\{x \in G\,^{\prime} \mid x=f\(a\) \text { for some } a \in G\right\}.
    $$
    \\ \hh The \blue{kernel} of $f$ , denoted by $\operatorname{ker}f$, is the set of elements of $G$ that are	mapped to the identity in $G\,'$:
    $$
    \operatorname{ker} f=\left\{a \in G \mid f\(a\)=1_{G\,^{\prime}}\right\}.
    $$
    \mydef{Comapre and Contrast}
    \hh Let $U, V$ be real or complex vector spaces and $L \in \mathscr{L}\(U,V\,\)$, 
    then we define the range and kernel of $L$ by:
    \begin{equation*}
        \begin{aligned}
            \ran L&:=\{v\in V :\underset{u\in U}{\exists}~ v=Lu\}\\
            \text{ker } L&:=\{u\in U : Lu=0\}
        \end{aligned}
    \end{equation*} 
\end{frame}
\begin{frame}
    \frametitle{Properties}
    \hh Let $f : G \to G\,'$ be a group homomorphism, and let $a, b \in G$. 
    Let $K$ = $\ker f$. The following are equivalent:
		\begin{enumerate}
			\item $f\(a\)=f\(b\)$
			\item $a^{-1} b \in K$
			\item $b \in a K$
			\item $a K=b K$
		\end{enumerate}
    \vv
    \red{!} A homomorphism $f : G \to G\,'$ is injective 
    if\mbox{f} $\ker f = \{1_G\}$.\\
    \red{!} Isomorphism $G \cong G\,'$ $~~\LRarrow~~$ $f$ is \textbf{bijective}. \\
    \red{!} How to check if a \textbf{homomorphism} is an \textbf{isomorphism}:
    \begin{center}
        verify $\ker f= \{1_G\}$ (injection) and $\operatorname{im} f = G\,'$ (bijection)
    \end{center}
\end{frame}
\begin{frame}
    \frametitle{Exercise}
    3. Prove: Let a homomorphism $f:G \to G\,'$. 
    If $H$ is a subgroup of $G$, then $f\(H\)^{-1}$ is a subgroup of $G\, '$.
    %\pause 
    \\\vs{3em}
    \yellow{Solution:}\\
    \par Let $x,y, a \in H$.
    \begin{enumerate}
    	\item Closure: $f(x)^{-1} f(y)^{-1} =  f\,(x^{-1})f\,(y^{-1}) = f\(x^{-1}y^{-1}\) = f\((yx)^{-1}\) = f(yx)^{-1}.$
    	\item Identity: $1_G \in H, 1_{G\,'} = f\(1_{G}\) \in f(H)^{-1}$.
    	\item Inverse: $f(a)^{-1} = f\(a^{-1}\) \in f(H)^{-1}$. 
    \end{enumerate}
\end{frame}
\begin{frame}
    \frametitle{Exercise}
    4. Let $(G,\cdot)$ be a group. Let $g,h \in G$ both have
    order $n$, prove that  $\<g\>\cong\<h\>$.
    \pause
    \\\vs{2em}
    \yellow{Solution:}\\
    \hh Define $f: \<g\> \to \<h\>$ by $f\,(g)=h$ and for all $0\leq k \leq n, f\(g^k\)=f\,(g)^k$.
    So, $f$ is a well-defined function, and, by definition, $f$ preserves the group
    product. It is clear that the function $f$ sends $1_G \mapsto 1_G$, $g \mapsto h$, $\dots$, $g^{n-1}\mapsto h^{n-1}$,
    and so $f$ is a bijection.
    \\\vv
    (Directly taken from Zach's slides)
\end{frame}
\section{Cosets}
\begin{frame}
    \frametitle{\red{Cosets}}
    \fbox{
    \parbox{0.95\textwidth}{
		\hh Given a group $G$, if $H$ is a subgroup of $G$ and $a \in G$, the notation $aH$ will stand for the set of all products $ah$ with $h \in H$,
		$$
		a H=\{g \in G \mid g=ah \text { for some } h \in H\}
		$$
		This set is called a \textbf{left coset} of $H$ in $G$.	
	}}
    \\\vv
    \fbox{
    \parbox{0.95\textwidth}{
    \hh The number of left cosets of a subgroup is called the index of $H$ in $G$. The index is denoted by $[G : H\,]$ (can be infinite, why?).
    \par \hh  All left cosets $aH$ of a subgroup $H$ of a group $G$ have the same order.
    }
    }
    \begin{itemize}
        \item[-] \blue{Counting formula:} $|G| = |H| \cdot [G:H\,]$.
        \item[-] \blue{Lagrange's Theorem:} Let $H$ be a subgroup of a finite group $G$. 
        The order of $H$ divides the order of $G$.
    \end{itemize}
\end{frame}
\begin{frame}
    \frametitle{Exercise}
    5. Verify Lagrange's Theorem for the subgroup 
    $H = \{0, 3\}$ of $\mathbb{Z}_6$ .
    \pause
    \\\vs{2em}
    \mysol
    The cosets are
    $$0 + H = \{0, 3\}, \quad 1 + H = \{1, 4\}, \quad 2 + H = \{2, 5\}.$$
    
    Notice there are 3 cosets, each containing 2 elements, 
    and that the cosets form a \textbf{partition} of the group.
\end{frame}
\begin{frame}
    \frametitle{An important consequence of Lagrange's Theorem}
    \mydef{Theorem}
    \hh Let $(G,\cdot)$ be a group and let $g \in G$ have order $n$. 
    If there exists $m,k \in \bN\cut\{0\}$ with $n = mk$,
    then the order of $g^m$ is $k$. \\ \vs{0.3em}
    \mydef{Proof.}
    \hh Let $m,k \in \bN\cut\{0\}$ with n = mk. 
    Now, $(g^m)^k=g^{mk}=g^n=1_G=$. If $0<q<k$ is 
    such that $(g^m)^q=1_G$, then $g^{mq}=1_G$. But 
    $mq<mk=n$, 
    which is a contradiction.\\ \vs{0.3em}
    \mydef{Theorem}
    \hh If $(G,\cdot)$ is a finite group with order $n$, 
    then for all $g \in G, g^n = 1_G$.\\ \vs{0.3em}
    \mydef{Proof.}
    \hh Let $(G,\cdot)$ be a finite group with order $n$. 
    Let $g \in G$. We know that the order of $g$ must be finite, 
    so let $k$ be the order of $g$. Now, $k$ must divide $n$,
    so the exists $m \in N$ such that $n = mk$. So 
    $g^n=g^{mk}=(g^k)^m$ $=1_G ^ m=1_G$.
\end{frame}
\begin{frame}
    \frametitle{Exercise}
    6. Prove that for any subgroup $H \leq G$, the
    (left) cosets of $H$ partition the group $G$.
    \\\vs{3em}
    \yellow{Hint:}\\
    \hh We need to show that the union of the left cosets is the whole group, and that
    different cosets do not overlap.
\end{frame}
\begin{frame}
    \frametitle{\red{Normal Subgroup}}
    \fbox{
	\parbox{0.95\textwidth}{
		\hh Given group $G$, and $a, g \in G$, the element $gag^{-1} \in G$ is called the
		conjugate of a by $g$.
		\par \hh A subgroup $N$ of $G$ is a normal subgroup,
        denoted by $N \unlhd G$, if for all $a \in N$ and $g \in G$, 
        $gag^{-1} \in N$. 
	}}
    \\\vs{1em}
    \blue{Properties:} 
    \begin{itemize}
        \item $f : G \to G\,'$ a homomorphism, then $\ker f \trianglelefteq G$.
        \item Every subgroup of an abelian group is normal.
        \item The \blue{center} is always a normal subgroup.
        \item $gH=Hg$ for all $g\in G$ if\mbox{f} $H \trianglelefteq G $. 
        \item $A_n\trianglelefteq S_n$.
    \end{itemize}
    \red{\textbf{Try your best! Remember them!}}
\end{frame}
\begin{frame}
    \frametitle{Exercise}
    \red{Important result:}\\
    \vv
    7. Show that any subgroup of index 2 in a group
    is a normal subgroup.\\
    \vv\pause
    \mysol
    \hh Denote the subgroup as $H$. Obviously, the left cosets of a subgroup of index 2 are
    $1_H H= H$ and $aH$, where $a \not\in H$; (\red{why?}) the right 
    cosets are $H 1_H=H$ and $Ha$. Since the cosets form a 
    partition of the origin group, and $1_H H = H 1_H =H $, so 
    the remaining is another coset, namely $aH=Ha$. (\blue{left=right}) 
    So $H$ is normal.
    \\\vs{2em}
    \cha{University of zhihu: \url{https://zhuanlan.zhihu.com/p/163548084}}\\
\end{frame}
\section{*Extra Topic}
\begin{frame}
    \frametitle{Distribution of Primes (Part I)}
    \mydef{Proposition}
    \hh Let $K$ be any positive integer larger than $2$, then 
    there exists two adjacent primes $p$ and $p'$ $(p'<p)$, such that
    $p-p'\geq K$. 
    \\\vs{2em}
    \mydef{Proof:}
    \hh Let $K!+2=M$, then $2 \mid M, 2+1 \mid M+1, \cdots , K \mid M+K-2$. Since 
    $M>2$, we conclude that $M, M+1,\cdots,M+K-2$ are all composite. Let $p'$ be the 
    largest prime that is smaller than $M$, but the next prime $p$ is denifitely larger than $M+K-2$, namely 
    $$p-p'\geq (M+K-1)-(M-1)=K.$$
\end{frame}
\begin{frame}
    \frametitle{Distribution of Primes (Part I)}
    \mydef{Definition}
    \hh We denote $\pi(x)$ as the number of primes no larger than
    $x$. Namely
    $$\pi(x)=\sum_{p\leq x} 1.$$ 
    We already know that as $x\to \infty$, $\pi(x)\to \infty$. But
    how fast it grows? 
    \\ \hh Here, we're going to prove that $\pi(x) = \Theta (x/\ln x)$.
    Namely, there exists two positive numbers $A_1$ and $A_2$, such that
    $$A_1 \frac{x}{\ln x}< \pi(x)<A_2 \frac{x}{\ln x} ~~(x\geq 2)$$ 
    This is so called 
    {\fontencoding{OT2}\selectfont \red{Qebyxev}} (Chebyshev) inequality in number theory.
\end{frame}

\begin{frame}
    \frametitle{Distribution of Primes (Part I)}
    Before prove the above inequality, we need to prove the following two lemmas.
    \\\vv
    \mydef{Lemma 1}
    \hh Let $n$ be any positive integer, set
    $$N=\frac{(2n)!}{\(n!\)^2}$$  
    then 
    $$\(\pi(2n)-\pi(n)\)\ln n \leq \ln N \leq \pi(2n) \ln(2n).$$
\end{frame}
\renewcommand{\[}{\left[}
\renewcommand{\]}{\right]}
\begin{frame}
    \frametitle{Proof of Lemma 1}
    Let 
    $$N = \prod_{p\leq 2n}p^\alpha_p$$
    to be the standard decomposition of $N$, we have
    $$
        \alpha_p = \sum_{r=1}^\infty \left[\frac{2n}{p^r}\right]
        -2 \sum_{r=1}^\infty \[\frac{n}{p^r}\] =
        \sum_{r=1}^{\[\frac{\ln(2n)}{\ln p}\]}
        \(\[\frac{2n}{p^r}\]-2\[\frac{n}{p^r}\]\),
    $$
    (this is because when $r > \floor{\ln(2n)/\ln(p)}$, $p^r > 2n > n$). Obviously,
    $$
        \alpha_p \leq \sum_{r=1}^{\[\frac{\ln(2n)}{\ln p}\]} 1
        = \[ \frac{\ln(2n)}{\ln p}\]\leq \frac{\ln(2n)}{\ln p}.
    $$
\end{frame}
\begin{frame}
    \frametitle{Proof (Cont.)}
    Therefore
    $$
        \ln N = \sum_{p\leq 2n} 
        \alpha_p \ln p \leq 
        \sum_{p\leq 2n} \ln (2n) =
        \pi(2n)\ln(2n).
    $$
    On the other hand, if $n< p \leq 2n$, then $p \mid (2n)!, \(p,\(n!\)^2\)=1$,
    so $p \mid N$. We have 
    $$ N \geq \prod_{n<p\leq 2n} p.$$
    Take logrithm on both side,
    $$
        \ln N \geq \sum_{n<p\leq 2n} \ln p > \ln n \sum_{n<p\leq 2n} 1 = 
        \(\pi(2n)-\pi(n)\)\ln n,
    $$
    this complete our proof.
\end{frame}
\begin{frame}
    \frametitle{Esitimation of $\ln N$}
    Now it's time to esitimate how large $\ln N$ is.\\\vv
    \mydef{Lemma 2}
    \hh For the same $n, N $ defined in Lemma 1, we have
    $$
        n \ln 2 \leq \ln N \leq 2n \ln 2.
    $$
    \mydef{Proof:}
    \hh Considering that $N$ is the coefficient of term $x^n$
    when expanding $(1+x)^{2n}$, so 
    $$ N\leq (1+1)^{2n} = 2^{2n}$$
    On the other hand,
    $$ N = \frac{2n(2n-1)\cdots(n+1)}{n!}=2\(2+\frac{1}{n-1}\)\cdots 
    (2+\frac{n-1}{1})\geq 2^n.
    $$
\end{frame}
\begin{frame}
    \frametitle{Reference}
    \begin{itemize}
        \item Examples From Zach's Slides (P196)
        \item Exercises from 2021-Fall-Ve203 TA Zhao Jiayuan
        \item Yan Shijian, etc. \textit{Basic Number Theory},
        fourth edition. Beijing: Higher Education Press, 2020.5 print.
    \end{itemize}
\end{frame}
\end{document}