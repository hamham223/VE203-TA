\documentclass{beamer}
\newcommand{\myfont}{\rmfamily\normalsize\upshape\mdseries}
\newcommand{\degree}{^\circ}
\title{\sffamily Review VI(Slides 312 - 351)}
\subtitle{\textbf{Modular Arithmetic}\\}
\institute[UM-SJTU JI]{University of Michigan-Shanghai Jiao Tong University Joint Institute}
\author{HamHam}
\usepackage{graphicx}
\usepackage{picinpar}
\usepackage{indentfirst}
\usepackage{chemformula}
\usepackage{geometry}
\usepackage{subfigure}
\usepackage{appendix}
\usepackage{amsfonts}
\usepackage{enumerate}
\usepackage{float}
\usepackage{geometry}
\usepackage{latexsym}
\usepackage{listings}
\usepackage{multicol,multirow,multido}
\usepackage{tabularx}
\usepackage{ulem}
\usepackage{tikz}
\usepackage{xcolor}
\usepackage{cite}
\usepackage{setspace}
\usepackage{hyperref}
\usepackage{textpos}
\usepackage{booktabs}
\usepackage{diagbox}
\usepackage{listings}
\usepackage{graphics}
\usepackage{upgreek}
\usepackage{JI_MathCourse_Notations}
\usepackage{mathrsfs}
\usepackage[OT2,OT1]{fontenc}
%%%%%%%%%%%%%%%%%%%%%%%%%%%%%%%%%%%%%%%%%%%%%%%%%%%%%%%%%%%%%%%%%%%%%%%%%%%%%%
\iffalse 
Copyright 2021 by Leyang Zhang, Yinchen Ni, Yuxiang Chen, Yue Huang

You should not spread this file without the permission from every one of 
    Leyang Zhang, Yinchen Ni and Yuxiang Chen 
\fi
%%%%%%%%%%%%%%%%%%%%%%%%%%%%%%%%%%%%%%%%%%%%%%%%%%%%%%%%%%%%%%%%%%%%%%%%%%%%%

\ProvidesPackage{JI_MathCourse_Notations}[2021/12/19]

\RequirePackage{amsmath}
\RequirePackage{amsthm}
\RequirePackage{amssymb}
\RequirePackage{bm}
\RequirePackage{bbm}
\RequirePackage{color}


%Typesetting

%colors 
\newcommand{\blue}[1]{\textcolor{blue}{#1}}                         %blue
\newcommand{\cha}[1]{\textcolor[rgb]{0.95,0.75,0.9}{#1}}            %cha-type color 
\newcommand{\gray}[1]{\textcolor{gray}{#1}}                         %grey
\newcommand{\pink}[1]{\textcolor{pink}{#1}}                         %pink
\newcommand{\red}[1]{\textcolor[rgb]{0.75,0,0}{#1}}                 %red
\newcommand{\yellow}[1]{\textcolor{orange}{#1}}                     %yellow

%declarations of Math lines, turn off code checker before using them
\newcommand{\beq}{\begin{equation}}
\newcommand{\beqa}{\begin{equation}\begin{aligned}}
\newcommand{\beqNo}{\begin{equation*}}
\newcommand{\beqaNo}{\begin{equation*}\begin{aligned}}
\newcommand{\eeq}{\end{equation}}
\newcommand{\eeqa}{\end{aligned}\end{equation}}
\newcommand{\eeqNo}{\end{equation*}}
\newcommand{\eeqaNo}{\end{aligned}\end{equation*}}

%space 
\newcommand{\hh}{\hspace{1em}}
\newcommand{\hs}[1]{\hspace{#1}}
\newcommand{\vs}[1]{\vspace{#1}}
\newcommand{\vv}{\vspace{1em}}


%-------------------------------------------------------------------


%Letter-like Notations 

%A - Z notations 
\newcommand{\bA}{\mathbb{A}}    \newcommand{\calA}{\mathcal{A}}     \newcommand{\fA}{\mathfrak{A}}
\newcommand{\bB}{\mathbb{B}}    \newcommand{\calB}{\mathcal{B}}     \newcommand{\fB}{\mathfrak{B}} 
\newcommand{\bC}{\mathbb{C}}    \newcommand{\calC}{\mathcal{C}}     \newcommand{\fC}{\mathfrak{C}} 
\newcommand{\bD}{\mathbb{D}}    \newcommand{\calD}{\mathcal{D}}     \newcommand{\fD}{\mathfrak{D}} 
\newcommand{\bE}{\mathbb{E}}    \newcommand{\calE}{\mathcal{E}}     \newcommand{\fE}{\mathfrak{E}} 
\newcommand{\bF}{\mathbb{F}}    \newcommand{\calF}{\mathcal{F}}     \newcommand{\fF}{\mathfrak{F}} 
\newcommand{\bG}{\mathbb{G}}    \newcommand{\calG}{\mathcal{G}}     \newcommand{\fG}{\mathfrak{G}} 
\newcommand{\bH}{\mathbb{H}}    \newcommand{\calH}{\mathcal{H}}     \newcommand{\fH}{\mathfrak{H}} 
\newcommand{\bI}{\mathbb{I}}    \newcommand{\calI}{\mathcal{I}}     \newcommand{\fI}{\mathfrak{I}} 
\newcommand{\bJ}{\mathbb{J}}    \newcommand{\calJ}{\mathcal{J}}     \newcommand{\fJ}{\mathfrak{J}} 
\newcommand{\bK}{\mathbb{K}}    \newcommand{\calK}{\mathcal{K}}     \newcommand{\fK}{\mathfrak{K}}  
\newcommand{\bL}{\mathbb{L}}    \newcommand{\calL}{\mathcal{L}}     \newcommand{\fL}{\mathfrak{L}} 
\newcommand{\bM}{\mathbb{M}}    \newcommand{\calM}{\mathcal{M}}     \newcommand{\fM}{\mathfrak{M}} 
\newcommand{\bN}{\mathbb{N}}    \newcommand{\calN}{\mathcal{N}}     \newcommand{\fN}{\mathfrak{N}}  
\newcommand{\bO}{\mathbb{O}}    \newcommand{\calO}{\mathcal{O}}     \newcommand{\fO}{\mathfrak{O}} 
\newcommand{\bP}{\mathbb{P}}    \newcommand{\calP}{\mathcal{P}}     \newcommand{\fP}{\mathfrak{P}} 
\newcommand{\bQ}{\mathbb{Q}}    \newcommand{\calQ}{\mathcal{Q}}     \newcommand{\fQ}{\mathfrak{Q}} 
\newcommand{\bR}{\mathbb{R}}    \newcommand{\calR}{\mathcal{R}}     \newcommand{\fR}{\mathfrak{R}} 
\newcommand{\bS}{\mathbb{S}}    \newcommand{\calS}{\mathcal{S}}     \newcommand{\fS}{\mathfrak{S}}     
\newcommand{\bT}{\mathbb{T}}    \newcommand{\calT}{\mathcal{T}}     \newcommand{\fT}{\mathfrak{T}} 
\newcommand{\bU}{\mathbb{U}}    \newcommand{\calU}{\mathcal{U}}     \newcommand{\fU}{\mathfrak{U}} 
\newcommand{\bV}{\mathbb{V}}    \newcommand{\calV}{\mathcal{V}}     \newcommand{\fV}{\mathfrak{V}} 
\newcommand{\bW}{\mathbb{W}}    \newcommand{\calW}{\mathcal{W}}     \newcommand{\fW}{\mathfrak{W}} 
\newcommand{\bX}{\mathbb{X}}    \newcommand{\calX}{\mathcal{X}}     \newcommand{\fX}{\mathfrak{X}} 
\newcommand{\bY}{\mathbb{Y}}    \newcommand{\calY}{\mathcal{Y}}     \newcommand{\fY}{\mathfrak{Y}} 
\newcommand{\bZ}{\mathbb{Z}}    \newcommand{\calZ}{\mathcal{Z}}     \newcommand{\fZ}{\mathfrak{Z}} 

%Math-specific notations 

\newcommand{\ep}{\epsilon}                                          %epsilon
\newcommand{\vep}{\varepsilon}                                      %variable epsilon


%-------------------------------------------------------------------


%Common Operations 

%sets 
\renewcommand{\Cap}{\bigcap}                                        %A intersect with B 
\newcommand{\card}{\text{card }}                                    %cardinality of a set
\renewcommand{\Cup}{\bigcup}                                        %A union with B 
\newcommand{\cut}{\backslash}                                       %A \ B 
\newcommand{\es}{\approx}                                           %equinumerosity between two sets 
\newcommand{\nes}{\not\approx}                                      %not Equinumerosity 
\newcommand{\siq}{\subseteq}                                        %A is contained in B 
\newcommand{\soq}{\supseteq}                                        %A is out of range of B 
\newcommand{\symd}{\Delta}                                          %symmetric difference 

%functions and function-related operations 
\newcommand{\ceil}[1]{\lceil {#1} \rceil}                           %smallest integer no less than #1 
\newcommand{\dist}[2]{\text{dist}\left ({#1},{#2} \right )}         %Distance function 
\newcommand{\floor}[1]{\lfloor {#1} \rfloor}                        %largest integer no greater than #1 
\newcommand{\inv}[1]{{#1}^{-1}}                                     %the inverse of a function, a number or a group element
\renewcommand{\mod}[1]{\,(\text{mod } {#1})}                        %modular of an integer #1 
\newcommand{\ran}{\text{ran }}                                      %the range of a function
\newcommand{\inflim}[1]{\varliminf_{#1}}                            %limit infimum of sequence or set
\newcommand{\limc}[2]{\lim_{#1 \to #2}}                             %limit 
\newcommand{\suplim}[1]{\varlimsup_{#1}}                            %limit supremum of sequence or set

%Others
\newcommand{\larrow}{\leftarrow}                                    %single left arrow, implies
\newcommand{\rarrow}{\rightarrow}                                   %single right arrow, implies 
\newcommand{\Larrow}{\Leftarrow}                                    %double left arrow, implies 
\newcommand{\Rarrow}{\Rightarrow}                                   %double right arrow, implies 
\newcommand{\lrarrow}{\leftrightarrow}                              %single left-right arrow 
\newcommand{\LRarrow}{\Leftrightarrow}                              %double left-right arrow, shorter than \iff 
\newcommand{\lowBrace}[2]{\underset{#1}{\underbrace{#2}}}           %under-brace with notes 
\renewcommand{\(}{\left (}                                          %left-half bracket, turn off code checker before using it 
\renewcommand{\)}{\right )}                                         %right-half bracket, turn off code checker before using it 
\newcommand{\<}{\langle}                                            %left-half angular bracket 
\renewcommand{\>}{\rangle}                                          %right-half angular bracket


%-------------------------------------------------------------------


%Patches  

%Analysis and Calculus 
\newcommand{\df}[2]{\frac{d{#1}}{d{#2}}}                            %one-variable differential operator
%\newcommand{\indf}[2]{\dfrac{d{#1}}{d{#2}}}                         %differential operator for in-line math mode
\newcommand{\im}{\text{Im }}                                        %taking the imaginary part of a number or a function 
\newcommand{\inrPdct}[2]{\left \langle{#1},{#2}\right \rangle}      %inner product 
\newcommand{\nDf}[2]{{#1}^{({#2})}}                                 %n-th derivative of some function 
\newcommand{\norm}[1]{\left \|#1\right \|}                                       %norm 
\newcommand{\parf}[2]{\frac{\partial{#1}}{\partial{#2}}}            %partial derivative of a function 
\newcommand{\parfwide}[2]{\partial{#1} / \partial{#2}}              %partial derivative of a function, wide form 
\newcommand{\re}{\text{Re }}                                        %taking the real part of a number or a function 

%Combinatorics 
\newcommand{\al}{{\aleph}_{0}}                                      %Aleph zero 
\newcommand{\all}{{\aleph}_{1}}                                     %Aleph one
\newcommand{\dbinomial}[2]{\begin{pmatrix}\!\!
                           \begin{pmatrix}{#1}\\{#2} 
                           \end{pmatrix}
                           \!\!\!
                           \end{pmatrix}}                           %double binomial 
\newcommand{\kpermu}[2]{{#1}^\text{\underline{#2}}}                 %k-permutation 
\newcommand{\logs}{\log^{*}}                                        %Iterated Logarithm 
\newcommand{\stirling}[2]{\left \{\begin{aligned} 
                                    {#1} \\ {#2} 
                                  \end{aligned} 
                          \right \}}                                %Stirling numbers of second kind 

%Graph Theory 
\newcommand{\comp}[1]{\text{comp} \left({#1}\right)}                %Component of graph 
\newcommand{\conj}{\overline}                                       %Complement graph of a given graph

%intersection of a collection of sets indexed by I 
\newcommand{\AI}{\cap_{i \in I} A_i} 
\newcommand{\BI}{\cap_{i \in I} B_i} 
\newcommand{\CI}{\cap_{i \in I} C_i} 
\newcommand{\DI}{\cap_{i \in I} D_i} 
\newcommand{\EI}{\cap_{i \in I} E_i} 
\newcommand{\FI}{\cap_{i \in I} F_i} 
\newcommand{\GI}{\cap_{i \in I} G_i} 
\newcommand{\HI}{\cap_{i \in I} H_i} 
\newcommand{\II}{\cap_{i \in I} I_i} 
\newcommand{\JI}{\cap_{i \in I} J_i} 
\newcommand{\KI}{\cap_{i \in I} K_i} 
\newcommand{\LI}{\cap_{i \in I} L_i} 
\newcommand{\MI}{\cap_{i \in I} M_i} 
\newcommand{\NI}{\cap_{i \in I} N_i} 
\newcommand{\OI}{\cap_{i \in I} O_i} 
\newcommand{\PI}{\cap_{i \in I} P_i} 
\newcommand{\QI}{\cap_{i \in I} Q_i} 
\newcommand{\RI}{\cap_{i \in I} R_i} 
%\renewcommand{\SI}{\cap_{i \in I} S_i}                             %\SI already defined, delete '%' if you want to renew it 
\newcommand{\TI}{\cap_{i \in I} T_i} 
\newcommand{\UI}{\cap_{i \in I} U_i} 
\newcommand{\VI}{\cap_{i \in I} V_i} 
\newcommand{\WI}{\cap_{i \in I} W_i} 
\newcommand{\XI}{\cap_{i \in I} X_i} 
\newcommand{\YI}{\cap_{i \in I} Y_i}
\newcommand{\ZI}{\cap_{i \in I} Z_i}                                

%limit as "letter" tends to infinity 
\newcommand{\limftya}{\lim_{a \to \infty}}                          
\newcommand{\limftyb}{\lim_{b \to \infty}} 
\newcommand{\limftyc}{\lim_{c \to \infty}} 
\newcommand{\limftyd}{\lim_{d \to \infty}} 
\newcommand{\limftye}{\lim_{e \to \infty}} 
\newcommand{\limftyf}{\lim_{f \to \infty}} 
\newcommand{\limftyg}{\lim_{g \to \infty}} 
\newcommand{\limftyh}{\lim_{h \to \infty}}
\newcommand{\limftyi}{\lim_{i \to \infty}}                          
\newcommand{\limftyj}{\lim_{j \to \infty}}                          
\newcommand{\limftyk}{\lim_{k \to \infty}}
\newcommand{\limftyl}{\lim_{l \to \infty}}
\newcommand{\limftym}{\lim_{m \to \infty}}
\newcommand{\limftyn}{\lim_{n \to \infty}}                          
\newcommand{\limftyo}{\lim_{o \to \infty}} 
\newcommand{\limftyp}{\lim_{p \to \infty}} 
\newcommand{\limftyq}{\lim_{q \to \infty}}
\newcommand{\limftyr}{\lim_{r \to \infty}}
\newcommand{\limftys}{\lim_{s \to \infty}}
\newcommand{\limftyt}{\lim_{t \to \infty}}
\newcommand{\limftyu}{\lim_{u \to \infty}}
\newcommand{\limftyv}{\lim_{v \to \infty}}
\newcommand{\limftyw}{\lim_{w \to \infty}} 
\newcommand{\limftyx}{\lim_{x \to \infty}}                          
\newcommand{\limftyy}{\lim_{y \to \infty}}                          
\newcommand{\limftyz}{\lim_{z \to \infty}} 

%matrices 
\newcommand{\colTwo}[2]{\begin{pmatrix}
                        #1 \\ #2 
                        \end{pmatrix}}                              %length-two column vector 
\newcommand{\colThree}[3]{\begin{pmatrix}
                        #1 \\ #2 \\ #3
                        \end{pmatrix}}                              %length-three column vector 
\newcommand{\colFour}[4]{\begin{pmatrix}
                        #1 \\ #2 \\ #3 \\ #4 
                        \end{pmatrix}}                              %length-four column vector 
\newcommand{\colFive}[5]{\begin{pmatrix}
                        #1 \\ #2 \\ #3 \\ #4 \\ #5
                        \end{pmatrix}}                              %length-five column vector 
\newcommand{\diagTwo}[2]{\begin{pmatrix} 
                         #1 &\, \\ 
                         \, &#2 \\ 
                         \end{pmatrix}}                             %two-by-two diagonal matrix 
\newcommand{\diagThree}[3]{\begin{pmatrix} 
                           #1 &\, &\, \\ 
                           \, &#2 &\, \\ 
                           \, &\, &#3 \\ 
                           \end{pmatrix}}                           %three-by-three diagonal matrix 
\newcommand{\diagFour}[4]{\begin{pmatrix} 
                           #1 &\, &\, &\, \\ 
                           \, &#2 &\, &\, \\ 
                           \, &\, &#3 &\, \\ 
                           \, &\, &\, &#4 \\ 
                           \end{pmatrix}}                           %four-by-four diagonal matrix }
\newcommand{\maTwo}[4]{\begin{pmatrix} 
                        #1 &#2 \\  
                        #3 &#4 \\ 
                       \end{pmatrix}}                               %two-by-two matrix
\newcommand{\maTwoThree}[6]{\begin{pmatrix} 
                            #1 &#2 &#3 \\ 
                            #4 &#5 &#6 \\ 
                            \end{pmatrix}}                          %two-by-three matrix
\newcommand{\maThreeTwo}[6]{\begin{pmatrix}
                            #1 &#2 \\ 
                            #3 &#4 \\ 
                            #5 &#6 \\ 
                            \end{pmatrix}}                          %three by two matrix 
\newcommand{\maThree}[9]{\begin{pmatrix}
                         #1 &#2 &#3 \\ 
                         #4 &#5 &#6 \\ 
                         #7 &#8 &#9 \\ 
                         \end{pmatrix}}                             %three by three matrix

%determinants 
\newcommand{\deTwo}[4]{\det \begin{pmatrix}
                            #1 &#2 \\ 
                            #3 &#4 \\ 
                            \end{pmatrix}}                          %determinant of two by two matrix 
\newcommand{\deThree}[9]{\det \begin{pmatrix}
                              #1 &#2 &#3 \\ 
                              #4 &#5 &#6 \\ 
                              #7 &#8 &#9 \\ 
                              \end{pmatrix}}                        %determinant of three by three matrix 

%span 
\renewcommand{\span}{\text{span }}                                  %the linear span of a set
\newcommand{\spans}[1]{\text{span} \{#1\}}                          %the linear span of elements 

%Topology
\newcommand{\hBall}[2]{B_{#1}\left(#2\right)}                   %B_#1 (#2): open ball centered at #2 with radius #1 

%\usepackage{ctex} %插入中文
%\ctexset{today=old}

\newcommand{\mydef}[1]{\sffamily\blue{#1}\myfont\\} %for define
\newcommand{\mysol}{\yellow{Solution:}\\}
\usetheme[dove]{Boadilla}
\usecolortheme{dolphin}
\useoutertheme{miniframes}
\begin{document}
    \usebackgroundtemplate{\tikz\node[opacity=0.25]{
    \includegraphics[width=\paperwidth,
    height=\paperheight]{hamster.jpg}
    };}
\begin{titlepage}
    \begin{center}
        VE203 - Discrete Mathmatics 
    \end{center}
\end{titlepage}
\myfont
\newcommand{\green}[1]{\textcolor[rgb]{0.3,0.6,0}{#1}}
\section{Modular Arithmetic}
\begin{frame}
    \frametitle{Modular Arithmetic}
    \mydef{Definition}
    \hh For $a,b\in \bZ$, and $m\in\bN\cut\{0\}$
    we say that $a$ 
    \textbf{\textit{\green{ is congruent to $b$ modulo $m$}}}, 
    writing 
    $$
        a \equiv b \mod{m}~~~~ \text{if\mbox{f}}~~~~ 
        m \mid (a-b)
    $$
    The followings are equivalent:
    \begin{itemize}
        \item $a \equiv b \mod{m}$
        \item $\underset{k\in\bZ}{\exists}\, a=b+km$
        \item $a \operatorname{mod} m = b \operatorname{mod} m$
    \end{itemize}
    \par \hh 
    For any $n \in \bZ$
    $\equiv$ is an equivalence relation on $\bZ$. 
    We call such equivalence classes 
    \textbf{congruence classes}, denoted as 
    $a := [a]_\equiv$ for $a \in \bZ$. The set of congruence classes is 
    denoted as $\bZ/n\bZ$ in consistence with the notation in group theory.
\end{frame}
\begin{frame}
    \frametitle{Modular Arithmetic}
    \hh Under a given modulo, the congruence map $a \to \bar{a}$ preserves the
    arithmetic of integers, that is
    \begin{itemize}
        \item $\conj{a+b}=\conj{a}+\conj{b}$
        \item $\conj{ab}=\conj{a}\cdot\conj{b}$
    \end{itemize}
    Or you may prefer to write:
    \begin{itemize}
        \item $a+b \equiv (a \operatorname{mod} m + b \operatorname{mod} m) \mod{m}$
        \item $a\cdot b \equiv (a \operatorname{mod} m)\cdot ( b \operatorname{mod} m) \mod{m}$
    \end{itemize}
    Thus the following two are groups:
    \begin{itemize}
        \item $\(\bZ/n\bZ,+\)$
        \item $\(\{a\in \bZ/n\bZ : \gcd(a,n=1)\},\times\)$  
        (Sometimes $\(\bZ/n\bZ,\times\)$ or $\(\bZ/n\bZ\)^\times$ for short)
    \end{itemize}
\end{frame}
\begin{frame}
    \frametitle{Exercise}
    1. Prove that $41\mid 2^{20} -1$.
    \pause
    \\\vs{2em}
    \mysol
    \hh This is equivalent to showing that
    $$2^{20}-1 \equiv 0 \mod{41}.$$
    We note that $2^5=32\equiv -9 \mod{41}$. Then
    $$2^{20}=(2^5)^4\equiv (-9)^4 \mod{41}$$
    But $(-9)^4=81\cdot 81$ and $81\equiv -1 \mod{41}$. So 
    $$2^{20} \equiv (-1)^2\equiv 1 \mod{41}$$
\end{frame}
\begin{frame}
    \frametitle{Division in Modular Arithmetic}
    \hh We have seen the addition and multiplication in 
    modular arithemetic, what about division?\\\vv
    \mydef{Theorem}
    \hh Let $a,b,c \in \bZ$ and $m \in \bN \cut \{0\}$. Then 
    $$
        ac \equiv bc \mod{m} ~~~~\Rarrow~~~~ a\equiv b \mod {m/d}
    $$
    where $d=\gcd(c,m)$.
    \\\vv 
    \mydef {Proof}
    \hh There exist integers $r,s$ with $\gcd(r,s) = 1$ such that 
    $c = rd, m = sd$. Insert them to the equation $ac - bc=k\cdot m$.
\end{frame}
\begin{frame}
    \frametitle{Modular Inverse}
    \mydef{Definition}
    \hh Let $a\in\bZ$ and $m \in \bN \cut \{ 0,1\}$ be given
    . Then an integer $a^{-1} \in \bZ $ such that 
    $$a a^{-1} \equiv 1 \mod {m}.$$
    s said to be an \green{inverse of $a$ modulo $m$}.
    \\\vs{0.5em}
    \mydef{Theorem} 
    \hh  Let $a \in \bN \cut \{0\}$ and $m \in \bN \cut \{ 0,1\}$. 
    If gcd(a, m) = 1, an inverse of $a$ modulo $m$ exists. 
    This inverse is unique modulo $m$.
    \\ \vs{0.5em}
    \mydef{Proof}
    \begin{itemize}
        \item Existence: Bézout's Theorem 
        \item Uniqueness: Prove by contradiction
    \end{itemize}
    \begin{block}{How to find the inverse}
        \hh Solve $7x\equiv 1 \mod{31}$ $\LRarrow$ Solve $7\cdot x-t\cdot 31=1$
    \end{block}
\end{frame}
\section{Arithmetic Function}
\begin{frame}
    \frametitle{Arithmetic Function}
    \mydef{Definition}
    \hh \textbf{Arithmetic function}, any mathematical function defined for integers (sometimes positive integers only)
    and dependent upon those \red{properties of the integer itself as a number}, 
    in contrast to functions that are defined for other \red{values} 
    (real numbers, complex numbers, etc.) 
    and that involve various operations from algebra and calculus.
    \\\vs{0.5em}
    \mydef{Example}
    \begin{itemize}
        \item Euler's Totient Function $\varphi(n)$
        \item $\pi(x)$, number of primes no larger than $x$
        \item $\tau(a)$, number of positive factors of $a$
        \item $\sigma(a)$, sum of all positive factors of $a$
        \item Mobius Function 
            $\mu(a)=
                \begin{cases}
                    1,~~~~~~a=1,\\
                    (-1)^r,\text{product of $r$ different primes}\\
                    0,~~~~~~\text{divisible by a prime square}
                \end{cases} 
            $ 
    \end{itemize}
\end{frame}
\begin{frame}
    \frametitle{Multiplicative Function}
    \mydef{Definition}
    \hh A function $f : \bN \cut \{0\} \to \bN\{0\}$ is multiplicative 
    if $f \(1\) = 1$ and
    $f \(m_1m_2\) = f \(m_1\)f \(m_2\)$ for $\gcd(m_1,m_2) = 1$.
    \begin{block}{Exercise}
        Check whether the followings are multiplicative functions:
        \begin{itemize}
            \item $f(n)=n^c$, where $c$ is an arbitrary constant.
            \item $f(n)=[\text{For any integer } k>1, k^2\nmid n]$.
            \item $f(n)=c^k$, where $k$ is the number of primes that divides $n$.
            \item The product of any two multiplicative functions.
        \end{itemize}
    \end{block}
    \yellow{Answer: True;~~False;~~True;~~True.}
    \\\vs{0.5em}
    \red{\textit{Comment.} This does appear in the slides!}
\end{frame}

\begin{frame}
    \frametitle{Exercise}
    2. Let $f\(x\)$ be a multiplicative function, and the 
    standard decomposition for $a$ is $a=p_1^{\alpha_1} 
    p_2^{\alpha_2}\cdots p_k^{\alpha_k}$, 
    then 
    \begin{equation*}
        \begin{aligned}
            \sum_{d\mid a} f\(d\) &= \prod_{i=1}^k
            \(1+f\(p_i\)+f\(p_i^2\)+\cdots+f\(p_i^{\alpha_i}\)\)\\
            & = \prod_{i=1}^k \,\sum_{j=0}^{\alpha_i} f\(p_i^j\)
        \end{aligned}
    \end{equation*}
    \textit{Comment.} It is easy to see that, $f\(1\)$ must be $1$. 
\end{frame}
\begin{frame}
    \frametitle{Proof}
        \hh All the positive factor of $a$ is 
        $$ 
            p_1 ^ {\beta_1} p_2 ^ {\beta_2}\cdots p_k ^ {\beta_k},
            \beta_i = 0,1,2,\cdots,\alpha_i,i=1,2,\cdots,k
        $$
        So that 
        \begin{equation*}
            \begin{aligned}
                \sum_{d\mid a} f\(d\) 
                &= \sum_{\beta_1=0}^{\alpha_1} \sum_{\beta_2=0}^{\alpha_2} 
                \cdots \sum_{\beta_k=0}^{\alpha_k} 
                f\(p_1 ^ {\beta_1} p_2 ^ {\beta_2}\cdots p_k ^ {\beta_k}\)\\
                &= \sum_{\beta_1=0}^{\alpha_1} \sum_{\beta_2=0}^{\alpha_2} 
                \cdots \sum_{\beta_k=0}^{\alpha_k} 
                f\(p_1 ^ {\beta_1}\) f\(p_2 ^ {\beta_2}\) \cdots 
                f\(p_k ^ {\beta_k}\)\\
                &= \prod_{i=1}^k
                \(f\(p_i^0\)+f\(p_i\)+f\(p_i^2\)+\cdots+f\(p_i^{\alpha_i}\)\)
            \end{aligned}
        \end{equation*}
\end{frame}
\begin{frame}
    \frametitle{Euler's Totient Function}
    \mydef{Definition}
    \hh The \blue{Euler's Totient Function}
     counts the number of positive integers less than $n$ 
     and relatively prime to $n$, i.e.
     $$ \varphi(n)= |\{k\in\bN \mid \gcd(k,n)=1,1\leq k\leq n\}| = |\(\bZ/n\bZ\)^*| $$
    \green{Properties:}
    \begin{itemize}
        \item $\varphi(p)=p-1$
        \item $\varphi(p^k)=p^k-p^{k-1} (k\geq 1)$
        \item $\varphi(mn)=\varphi(m)\cdot \varphi(n)$, if $\gcd(m,n)=1$
        \item $\varphi(n)=\sum_{d\mid n} \varphi(d)$
        \item $\varphi(a)=\varphi\(\prod_{i=1}^k p_i^{\alpha_i}\) = \prod_{i=1}^k (p_i-1)p_i^{\alpha_i-1}$
        \item $\varphi(a)=a\(1-\frac{1}{p_1}\)\(1-\frac{1}{p_2}\)\cdots\(1-\frac{1}{p_k}\)$
    \end{itemize}
\end{frame}
\begin{frame}
    \frametitle{Exercise}
    \red{This is challenging!}\\\vv
    3. Let $S_{p,q}=\{f: \bZ/p\bZ \to \bZ/q\bZ\mid
    f \text{ is a group homomorphism}\}$. Given 
    $p,q$ primes, $p<q $, then 
    \begin{itemize}
        \item[(A)] $f\(1\)=1$ if $f \in S_{p,q}$
        \item[(B)] $f$ is an isomorphism if $f \in S_{p,q}$
        \item[(C)] $|S_{p,q}|\leq |S_{q,p}|$
        \item[(D)] $|S_{p,q}| = \varphi(q)^{\varphi(p)}$  
    \end{itemize}
    \vv
    \yellow{Answer: C}
\end{frame}
\section{The. \& App.}
\begin{frame}
    \frametitle{Euler's Theorem}
    \mydef{Theorem (Euler)}
    \hh For $m \in \bN \cut \{0\}$ and $a \in \bZ$ such that $\gcd(a, m) = 1$,
    $$a ^ {\varphi(m)}\equiv 1 \mod{m}$$
    \\\vs{0.5em}
    \mydef{Proof}
    \hh Let $G=\(\bZ/m\bZ\)^*$, then $\forall a \in G$, $a^{|G|}=1_G$.
    \\\vs{0.5em}
    \green{Remark}\\
    \hh When $m\in \bP$, this becomes \blue{Fermat's Little Theorem}.
\end{frame}
\begin{frame}
    \frametitle{Exercise}
    4. Given $a, n \in \bN$ and $a, n > 1$, show that 
    $n \mid \varphi(a^n - 1)$.\\ \vv
    \pause 
    \yellow{Solution 1:}\\
    \hh Let $m=a^n-1$, consider the multiplicative group 
    $G = \(\bZ/m\bZ\)^\times$. 
    \\ \hh 
        First we prove the order of $a$ is $n$.
        Indeed, $a^n\equiv 1\mod{m}$ and 
        $a^x \not \equiv 1 \mod{m}$ for $1<x<m$
        since $1<a^x<a^n=m$.
    \\ \hh 
        According to Lagrange's theorem, therefore 
    the order of a divides the order of $G$, 
    that is, $n \mid \varphi(a^n - 1)$.
    \\ \vv
    \yellow{Solution 2:}\\
    \hh 
    \begin{equation*}
        \left.
        \begin{aligned}
            m=a^n-1 \Rarrow a^n    &\equiv 1 \mod{m}\\
            \text{Euler} \Rarrow a^{\varphi(m)}&\equiv 1 \mod{m} 
        \end{aligned}
        \right\}
        \Rarrow n \mid \varphi(m) \text{\red{~(why?)}}
    \end{equation*}
\end{frame}
\begin{frame}
    \frametitle{Fermat's Little Theorem}
    \mydef{Theorem}
    \hh Given $a \in \bZ$ and $p \in \bP$, 
    such that $(a, p) = 1$, then
    \begin{equation*}
        \begin{aligned}
            a^{p-1}&\equiv 1\mod{p}\\
            a^p    &\equiv a\mod{p}
        \end{aligned}
    \end{equation*}
    \mydef{Proof}
    \hh Induction on $a$.\\\vs{0.5em}
    \mydef{Fermat Primality Test}
    \begin{itemize}
        \item If $2^n\not\equiv 2 \mod{n}$, then n is \red{NOT} prime.
        \item If $2^n\equiv 2\mod{n}$, then n is \red{PROBABLY} prime.
    \end{itemize}
    \begin{block}{Fast Exponentiation}
        \hh Express power in binary and play with your CASIO 991CN.
    \end{block}
\end{frame}
\begin{frame}
    \frametitle{Another Proof}
    \hh Here is another proof of \blue{Fermat's Little Theorem}. \\ 
    \hh Consider the set 
        $S=\{a,2a,\cdots, (p-1)a\}$. 
    For any $ma,na$ in S, there doesn't exist 
        $ma\equiv na \mod{p}$. \red{(Why?)} Therefore
        $$ 
            S\textrm{ mod }p=
            \{0\leqslant k\leqslant p-1|ma\equiv k (\textrm{mod }p), 
            ma\in S\}=\{1,2,\cdots ,p-1\}
        $$
    Then,
        $$a\cdot 2a \cdots (p-1)a\equiv (p-1)! (\textrm{mod }p)$$
    which implies 
        $$a^{p-1}(p-1)!\equiv (p-1)! \mod{p}$$ 
    Since $\gcd\((p-1)!, p\)=1$, we conclude $a^{p-1}\equiv 1\mod{p}$.
    \begin{block}{Exercise}
        \hh Prove \blue{Euler's Theorem} by considering  
        $$S=\{ka|\textrm{gcd}(k,n)=1, 1\leqslant k\leqslant n\}$$
    \end{block}
\end{frame}
\begin{frame}
    \frametitle{Wilson's Theorem}
    \mydef{Theorem (Wilson)}
    \hh Let $p \in \bN $ be prime. Then 
        $$ (p-1)! \equiv -1 \mod{p}.$$
    \\ \vv 
    \mydef{Proof:}
    \hh \red{Key idea: find the inverse and match in pair.} \\
    \vv
    \hh The inverse $a^{-1}$ modulo $p$ of $a$ exists and is unique
    modulo $p$,
    $$a^{-1} a\equiv 1 \mod{p}$$
    
\end{frame}
\begin{frame}
    \frametitle{Proof}
     We first show that $a = a^{-1}$
    if and only if $a = 1$ or $a = p - 1$. 
    \par \vv
    It is easily checked that
    $$
        1 \cdot 1 \equiv 1 \quad\mod{p} \quad \text { and } \quad(p-1)^{2} \equiv 1 \quad\mod{p}
    $$
    Now suppose that $a^{2} \equiv 1(\bmod p)$. Then
    $$
        (a-1)(a+1) \equiv 0 \quad\mod{p}
    $$
    Since $p$ is prime, this implies $a-1 \equiv 0$ or $a+1 \equiv 0\mod{p}$. 
    Hence, either $a=1$ or $a=p-1$.
\end{frame}
\begin{frame}
    \frametitle{Proof (Cont.)}
    Next, consider the remaining $p-3$ integers
    $$
        2,3, \ldots, p-2
    $$
    Since the inverse $a^{-1}$ of $a$ is unique, 
    it follows that these numbers can be grouped 
    into pairs $a, a^{-1}$ where $a^{-1} \neq a$ and $a^{-1} a \equiv 1\mod{p}$.
    Therefore,
    $$
        (p-2) !=2 \cdot 3 \cdots(p-2) \equiv 1 \quad\mod{p}
    $$
    Multiplying by $p-1$,
    $$
        (p-1) ! \equiv p-1 \equiv-1 \quad\mod{p}
    $$
    which is what we wanted to show.
\end{frame}
\begin{frame}
    \frametitle{Chinese Remainder Theorem}
    This is from Xue Runze.\\
    \mydef{Lemma}
    \hh For $\operatorname{gcd}(m, n)=1$, we have
    $\(\bZ / m n \mathbb{Z}\) \cong \(\mathbb{Z} / m \mathbb{Z}\) \times \(\mathbb{Z} / n \mathbb{Z}\)$
    \\\vv
    \mydef{Theorem (Chinese Remainder Theorem)}
    \hh For $\operatorname{gcd}\left(m_{i}, m_{j}\right)=1$ for all $i \neq j$, we have
    $$
        \mathbb{Z} /\left(\prod_{i=1}^{n} m_{i}\right) \mathbb{Z} \cong \prod_{i=1}^{n} \mathbb{Z} / m_{i} \mathbb{Z}
    $$
    and
    $$
        \left(\mathbb{Z} /\left(\prod_{i=1}^{n} m_{i}\right) \mathbb{Z}\right)^{\times} \cong \prod_{i=1}^{n}\left(\mathbb{Z} / m_{i} \mathbb{Z}\right)^{\times}
    $$
\end{frame}
\begin{frame}
    \frametitle{Exercise}
    \green{It's better to do an exercise and make
    friends with CASIO 991CN.}\\ \vv
    5. Solve the following system of 
    \blue{linear Diophantine equations},
    $$ 
        x \equiv 3 \mod{8}, \quad 
        x \equiv 1 \mod{15}, \quad 
        x \equiv 11 \mod{20} 
    $$
   
    \textit{Comment.} Please note that $\{m_i\}^r_{i=1}$
    should be  \textbf{pairwise coprime} before 
    you apply the formula.
    \begin{block}{Recipe}
        $$
            x \equiv \sum_{i=1}^r a_i y_i \mod{m}
        $$
        where $m=\prod_{i=1}^{r} m_{i} \text { and } 
        y_{i}=\left(m / m_{i}\right)^{\varphi\left(m_{i}\right)}
         \equiv \delta_{i j} \mod{m_j}$.
    \end{block}
\end{frame}
\begin{frame}
    \frametitle{RSA Cryptography}
    For your better preparation for the exam, we here omit the technical details 
    and gives the operation steps only.\\\vv
    \begin{itemize}
        \item The \textbf{public key} to be published is a pair of positive integers 
        $(n:=p q, E)$ where $p, q \in \mathbb{P}$ and 
        $p \neq q$, and $E<\varphi(n)$, $\operatorname{gcd}(E, \varphi(n))$.
        \item The \textbf{encryption function} is
        $$
        y=e(x):=x^{E} \quad \mod{n}
        $$
        \item The private key $D:=E^{-1} \bmod \varphi(n)$. 
        The \textbf{decryption function} is therefore
        $$
        d(y):=y^{D}=x^{E D}=x \quad\mod{n}
        $$
        \item Be careful and play with CASIO 991CN.
    \end{itemize}
\end{frame}
\section{*Extra Topic}
\begin{frame}
    \frametitle{Distribution of Primes (Part II)}
    Last time we proved two lemmas:\\
    \mydef{Lemma 1}
    \hh Let $n$ be any positive integer, set
    $$N=\frac{(2n)!}{\(n!\)^2}$$  
    then 
    $$\(\pi(2n)-\pi(n)\)\ln n \leq \ln N \leq \pi(2n) \ln(2n).$$
    \mydef{Lemma 2}
    \hh For the same $n, N $ defined in Lemma 1, we have
    $$
        n \ln 2 \leq \ln N \leq 2n \ln 2.
    $$
\end{frame}
\begin{frame}
    \frametitle{Distribution of Primes (Part II)}
    Now, it's time to prove the inequality.\\
    \vv
    \mydef{Proposition}
    \hh Let $x\geq 2$, then
    $$ 0.2 \frac{x}{\ln x} \leq \pi(x) \leq 5 \frac{x}{\ln x}$$
    \mydef{Proof: (Left)}
    \hh If $x\geq 6$, let $n= \floor{x/2}$, then $x\geq 2n, n > x/3$. 
    From Lemma 1,2 we immediately obtain that 
    $$
        \pi(x)\ln x\geq \pi (2n)\ln(2n) \geq 
        \ln N \geq n \ln 2 > \frac{\ln 2 }{3}\cdot x > 0.2x
    $$
    Considering that the maximum of $x/\ln x$ on the interval 
    $[2,6]$ is $6/\ln 6$, so when $2\leq x \leq 6$,
    $$
        0.2 \frac{x}{\ln x}\leq 0.2 \frac{6}{\ln 6} <1
        = \pi(2) \leq \pi(x).
    $$
\end{frame}
\begin{frame}
    \frametitle{Proof (Right)}
    \hh From Lemma 1,2, we know
    $$
        \(\pi(2n)-\pi(n)\)\ln (n) \leq \ln N \leq 2n \ln2.
    $$
    Plug in $n=2^r$,
    $$ r \(\pi\(2^{r+1}\)-\pi\(2^r\)\) \leq 2^{r+1}.$$
    Since $\pi (2^{r+1})\leq 2^r$,
    $$
        (r+1)\pi\(2^{r+1}\)-r\pi\(2^r\) \leq 2^{r+1} + 
        \pi\(2^{r+1}\) \leq 3 \cdot 2^r
    $$
    \hh For any positive integer $m$, let $r=0,1,\cdots,m-1$, we can
    obtain $m$ inequalities.
\end{frame}
%\renewcommand{\[}{\left[}
%\renewcommand{\]}{\right]}
\begin{frame}
    \frametitle{Proof (Right)}
    \hh Add the above $m$ inequalities together,
    $$
        m \pi\(2^m\) \leq 3 (1+2+\cdots +2 ^{m-1}) < 3 \times 2^m.
    $$
    When $x\geq 2$, there exists a unique positive integer $m$
    , such that $2^{m-1} \leq x < 2 ^ m$, so $1/m < \ln 2 /\ln x$. We get 
    $$
        \pi(x) \leq \pi\(2 ^ m\) \leq \frac{1}{m}\cdot 3 \cdot 2^m\leq 
        6\ln 2 \cdot \frac{x}{\ln x }\leq 5 \frac{x}{\ln x}
    $$
    This complete out proof.\\
    \vv
    \mydef{Corollary}
    \hh Almost all the numbers are composite, namely 
    $$
        \underset{x\to \infty}{\lim} \frac{\pi(x)}{x} =0.
    $$
\end{frame}
\begin{frame}
    \frametitle{More Result}
    \mydef{Theorem}
    \hh A more advanced result, the \blue{prime number theorem}
    $$
        \underset{x\to \infty}{\lim} \frac{\pi(x)}{\frac{x}{\ln x}}=1.
    $$
    \mydef{The Best result so far}
    Let 
    $$
        li(x):=\int_2^x \frac{\mathrm{d}t}{\ln t}
    $$
    Then 
    $$
        |\pi(x) - li(x)| \leq B x e^ {-A \(\ln x \) 3/5\times (\ln \ln x)-3/5}
    $$
    \mydef{A Guess}
    $$
        |\pi(x)-li(x)| \leq B x^ {\frac{1}{2}+\varepsilon}
    $$
\end{frame}
\begin{frame}
    \frametitle{Exercise}
    6. Let $p_n$ be the n-th prime. Prove that: there exists two 
    positive numbers $B_1, B_2$, such that 
    $$ B_a n \ln n \leq p_n \leq B_2 n \ln n$$
\end{frame}
\begin{frame}
    \frametitle{Reference}
    \begin{itemize}
        \item Example From Horst's Slides FA2021.
        \item Exercises from 2021-Fall-Ve203 Mid\_2 Exam.
        \item Exercises from 2019-Fall-Ve203 TA Yan Xinyu.
        \item Contents from 2021-Fall Mid\_2\_RC by Xue Runze.
        \item Yan Shijian, etc. \textit{Basic Number Theory},
        fourth edition. Beijing: Higher Education Press, 2020.5 print.
    \end{itemize}
\end{frame}
\end{document}