\documentclass{beamer}
\newcommand{\myfont}{\rmfamily\normalsize\upshape\mdseries}
\newcommand{\degree}{^\circ}
\title{\sffamily Review VII(Slides 458 - 521)}
\subtitle{\textbf{Group Theory} \\ This is the most difficult part!}
\institute[UM-SJTU JI]{University of Michigan-Shanghai Jiao Tong University Joint Institute}
\author{HamHam}
\usepackage{graphicx}
\usepackage{picinpar}
\usepackage{indentfirst}
\usepackage{chemformula}
\usepackage{geometry}
\usepackage{subfigure}
\usepackage{appendix}
\usepackage{amsfonts}
\usepackage{enumerate}
\usepackage{float}
\usepackage{geometry}
\usepackage{latexsym}
\usepackage{listings}
\usepackage{multicol,multirow,multido}
\usepackage{tabularx}
\usepackage{ulem}
\usepackage{tikz}
\usepackage{xcolor}
\usepackage{cite}
\usepackage{setspace}
\usepackage{hyperref}
\usepackage{textpos}
\usepackage{booktabs}
\usepackage{diagbox}
\usepackage{listings}
\usepackage{graphics}
\usepackage{upgreek}
\usepackage{JI_MathCourse_Notations}
\usepackage{mathrsfs}
\usepackage[thicklines]{cancel}
%定义绘制椭圆的命令,带有四个参数
%第一个参数:椭圆位置的横坐标
%第二、三参数:椭圆x位置的半径、y位置的半径
%第四个参数:填充的透明度
\def\drawell#1#2#3#4{
	\draw[draw=black,fill=gray,fill opacity=#4,line width=3pt] 
	(#1,0) circle [x radius=#2, y radius=#3];}

%\usepackage{ctex} %插入中文
%\ctexset{today=old}

\newcommand{\mydef}[1]{\sffamily\blue{#1}\myfont\\} %for define
\newcommand{\mysol}{\yellow{Solution:}\\}
\usetheme[dove]{Boadilla}
\usecolortheme{dolphin}
\useoutertheme{miniframes}
\begin{document}
    \usebackgroundtemplate{\tikz\node[opacity=0.25]{
    \includegraphics[width=\paperwidth,
    height=\paperheight]{hamster.jpg}
    };}
\begin{titlepage}
    \begin{center}
        VE203 - Discrete Mathmatics 
    \end{center}
\end{titlepage}
\myfont
\newcommand{\binomial}[2]{\begin{pmatrix} {#1}\\{#2}	\end{pmatrix}}
\newcommand{\green}[1]{\textcolor[rgb]{0.3,0.6,0}{#1}}


\section{Why?}
\begin{frame}
    \frametitle{Why group theory?}

    Group theory is soooooo abstract. Why we need it?

    \begin{block}{From GPT:}
        \begin{itemize}
            \item Philosophical view of Number theory: Group theory is used to study properties of numbers and explore topics such as modular arithmetic, Diophantine equations, and primality testing.
            \item Particle physics: Group theory is crucial in the field of particle physics, particularly in the study of the fundamental forces and particles. Symmetry groups, like the Standard Model gauge group, describe the interactions between particles.
            \item Geometry: Group theory is used to study the symmetries and transformations of geometric objects, leading to a deeper understanding of geometry and its applications in computer graphics and computer-aided design (CAD).
        \end{itemize}
    \end{block}

\end{frame}

\begin{frame}
    \frametitle{Why group theory?}

    \begin{block}{Fron GPT:}
        \begin{itemize}
            \item Coding theory: Group theory is employed in coding theory to construct error-correcting codes used in data transmission and storage systems.
            \item Music theory: Group theory has applications in music theory to analyze musical structures, symmetries, and chord progressions.
        \end{itemize}
    \end{block}
    More reference:
    \begin{itemize}
        \item \url{https://www.zhihu.com/question/29102364}
        \item \url{https://www.scienceforums.net/topic/51581-real-life-applications-of-group-theory/}
    \end{itemize}
\end{frame}

\section{Group Def.}
\begin{frame}
    \frametitle{Groups}
    \fbox{
    \parbox{0.95\textwidth}{
	\par A \textbf{group} is a pair $(G, \cdot)$, where $G$ is a set, and $\cdot : G \times G \to G$ is a law of composition that has the following properties:
	\begin{itemize}
        \item[-] \blue{Closure:} The generalized product is defined as $\cdot: G\times G \to G$
		\item[-] \blue{Associativety:} $(a \cdot b) \cdot c = a \cdot (b \cdot c)$ for all $a, b, c \in G$;
		\item[-] \blue{Identity:} $G$ contains an identity element 1, such that $1 \cdot a = a \cdot 1 = a$ for all $a \in G$;
		\item[-] \blue{Inverse:} Every element $a \in G$ has an inverse, an element $b$ such that $a \cdot b = b \cdot a = 1$.
	\end{itemize}
	\par \phantom{ji}
	\par An \textbf{\green{abelian group}} is a group whose law of composition is commutative $(a \cdot b = b \cdot a)$. 
	} 
    }
\end{frame}
\begin{frame}
    \frametitle{Properties}
    \fbox{
	\parbox{0.95\textwidth}{
		\par Given a group $G$, $a, b, c \in G$, then
		\begin{itemize}
			\item[-] there exists a \textbf{unique} identity element; 
            \\$\to$ suppose there are two distinct identity $i$ and $j$, then $i \cdot j = i = j$ 
			\item[-] $ba = ca \Rightarrow b = c$ and $ab = ac \Rightarrow b = c$;  \\$\to$ multiply by $a^{-1}$ on both sides, note that a group does not necessarily satisfy the commutative law
			\item[-] For all $a \in G$, there \textbf{exists} a \textbf{unique} element $b \in G$ such that $ab = ba = 1$; \\ $\to$ prove existence ($b = a^{-1}$) first, then prove uniqueness by contradiction
			\item[-]  $(ab)^{−1} = b^{-1}a^{-1}$. \\$\to$$(ab)^{−1}(ab) = 1$; $(b^{-1}a^{-1})(ab) = b^{-1}(a^{-1}a)b = 1$	
		\end{itemize}
	}
    }
\end{frame}
\begin{frame}
    \frametitle{Subgroup}
    \fbox{
	\parbox{0.95\textwidth}{
		\par A subset $H$ of a group $G$ is a subgroup if it has the following properties:
		\begin{itemize}
			\item[-] Closure: If $a, b \in H$, then $ab \in H$;
			\item[-] Identity: $1 \in H$;
			\item[-] Inverses: If $a \in H$, then $a^{-1} \in H$.
		\end{itemize}
		\par Look carefully at the identity and inverse axioms for a subgroup:
		\begin{itemize}
			\item[-] In verifying the identity axiom for a subgroup, the issue is not the existence of an identity 
            but \textbf{whether the identity for the group is actually contained in the subgroup}.
			\item[-] Likewise, for subgroups the issue of inverses is not whether inverses exist (every element of a group has an inverse) 
            but \textbf{whether the inverse of an element in the subgroup is actually contained in the subgroup}.
		\end{itemize}
	}
}
\end{frame}
\begin{frame}
    \frametitle{Exercise}
    1.  Given a group $G$ and its two distinct subgroups $H_1$ and $H_2$. 
    Check whether the following sentences are true or false:
    \begin{itemize}
        \item The identity element in $G$ and $H_1$ must be the same.
        \item $H_1\cup H_2 $ is a group.
        \item $H_1 \cap H_2$ cannot be empty and it is a group.
        \item A subset in $G$ that is not a subgroup may be a group.
    \end{itemize}
    \vv \hh
    \textit{Comment.} Compare to the concept of \textbf{\green{vector space}} in Vv186.
\end{frame}
\begin{frame}
    \frametitle{Exercise}
    2. $\mathbb{Z}^2 = \mathbb{Z} \times \mathbb{Z}$ denotes the set of pairs of integers: 
    $$ \mathbb{Z}^2 = \{(m, n) \mid m, n \in \mathbb{Z}\}.$$ 
    It is a group under ``vector addition'', that is, 
    $$(a, b) + (c, d) = (a + c, b + d).$$ 
    Consider the set $$H = \{(x, y) \mid x + y \geqslant 0\}.$$ 
    Check if $H$ is a subgroup of $\mathbb{Z}^2$.
\end{frame}
\begin{frame}
    \frametitle{Exercise}
    3. Let $G$ be a group and let $H \subset G$ with $H \neq \varnothing$. 
    If $\forall a, b \in H$ we have $ab^{-1} \in H$ then $H$ is a subgroup of $G$.
    
    \vv\yellow{Solution:}\\
    \hh Since $H \subset G$, any operation in $H$ has associativity. Then, we need to verify closure, identity, and inverses requirements but we need to
    do these \textbf{in a particular order}. 
    \begin{enumerate}
        \item  Since $H \neq \varnothing$, pick any $a \in H$. Then $aa^{-1} = e \in H$, so $H$ has the identity.
        \item Pick any $a \in H$. Since the identity $e \in H$, then $ea^{-1} = a^{-1} \in H$ so we have inverses. 
        \item Pick any $a, b \in H$. Then $b^{-1} \in H$ and denote as $c \in H$. So, $ac^{-1} \in H$ according to the problem statement. So, $ab = a(b^{-1})^{-1} = ac^{-1} \in H$ and we have closure.	
    \end{enumerate}
\end{frame}
\begin{frame}
    \frametitle{Exercise}
    4. Let $G$ be a group. If $\forall x \in G : x^2 = e$, show that $G$ is an abelian group.
    \\ \vv
    \yellow{Solution:}
    \\ \vv
    From $\forall x \in G : x^2 = e$, we obtain $x = x^{-1}$.
    \par Therefore, taking $\forall x,y \in G$, we have $$xy = (xy)^{-1} = y^{-1}x^{-1} = yx.$$
    \par This completes the proof.
    
\end{frame}
\section{Cyclic Group}
\begin{frame}
    \frametitle{Cyclic Group}
    \fbox{
	\parbox{0.95\textwidth}{
	\par The cyclic subgroup generated by $g$ is
	$$
	\left\langle g \right\rangle = \{g^k \mid k \in \mathbb{Z}\}.
	$$
	\par In other words, $\left\langle g \right\rangle$ consists of all (positive or negative) powers of $g$.
	$$
	\left\langle g \right\rangle = \{k \cdot g \mid k \in \mathbb{Z}\}.
	$$
	\par Be sure you understand that the difference between the two forms is simply \textbf{notational}: It’s the same concept.
	\par \phantom{ji}
    \par Let $G$ be a group, $g \in G$. The order of $g$ is the smallest positive 
    integer $n$ such that $g^n = 1$ $(ng = 0)$. 
    If there is no positive integer $n$ such that $g^n = 1 $ $(ng = 0)$, 
    then $g$ has \textbf{infinite} order.
	}
}
\end{frame}
\begin{frame}
    \frametitle{Exercise}
    5. List the elements of the subgroups generated by 
    elements of $\mathbb{Z}_8 = \{0,1,2,3,4,5,6,7\}$.
    \\\vv \yellow{Solution:}
    $$
    \begin{aligned}
    &\langle 0\rangle=\{0\} \\
    &\langle 2\rangle=\langle 6\rangle=\{0,2,4,6\} \\
    &\langle 4\rangle=\{0,4\} \\
    &\langle 1\rangle=\langle 3\rangle=\langle 5\rangle=\langle 7\rangle=\{0,1,2,3,4,5,6,7\} = \mathbb{Z}_8
    \end{aligned}
    $$
    \begin{block}{Question}
        \hh What is the identity? Order?
    \end{block}
\end{frame}
\begin{frame}
    \frametitle{Exercise}  
    6. Prove that
    \begin{enumerate}
        \item Let $G = \left\langle g \right\rangle$ be a finite cyclic group, where $g$ has order $n \neq 0$. Then the powers $\{1, g, \cdots , g^{n-1}\}$ are distinct.
        \item Let $G = \left\langle g\right\rangle $ be infinite cyclic. If $m$ and $n$ are integers and $m \neq n$, then $g^m \neq g^n$.
    \end{enumerate}
    \yellow{Solution:}\\
    \begin{enumerate}
        \item Since $g$ has order $n$, $g, g^2, \cdots, g^{n-1}$ are all different from 1.
        \par Suppose $g^i = g^j$ where $0 \leqslant j < i < n$. Then $0 < i - j < n$ and $g^{i-j} = 1$, contrary to the preceding observation.
        \par Therefore, the powers $\{1, g, \cdots , g^{n-1}\}$ are distinct.
        \item Suppose without loss of generality that $m > n$. We want to show that $g^m \neq g^n$.
        \par Suppose this is false, so  $g^m = g^n$. Then $g^{m-n} = 1$, so $g$ has finite order $m-n$. This contradicts the fact that a	generator of an infinite cyclic group has infinite order. Therefore,  $g^m \neq g^n$.
    \end{enumerate}
\end{frame}


\section{Symmetric Group}
\begin{frame}
    \frametitle{Symmetric Group}
    \mydef{Definition}
    \hh Given $n \in \mathbb{N} \backslash \{0\}$, we have the following symmetric group of degree $n$,
    $$
    \begin{aligned}
    S_{n} &=\{\text {All permutations on } n \text { letters/numbers}\} \\
    &=\operatorname{Sym}\{1,2,3, \ldots, n\} \\
    &=\{f:[n] \rightarrow[n] \mid f \text { bijective}\}
    \end{aligned}
    $$

    Note that it is a finite group of order $n!$ (the number of bijections from $[n]$ to $[n]$), 
    \textit{i.e.}, $|S_n| = n!$.
    \\
    \vv 
    \begin{itemize}
        \item  A subgroup of $S_n$ is called a \textbf{\green{permutation group}}.
        \item A permutation of the form $(ab)$ where $a \neq b$ 
        is called a \textbf{\green{transposition}}. 
    \end{itemize}
\end{frame}

\begin{frame}
    \frametitle{Permutation}
    \hh A permutation that can be expressed as a product of an \textbf{even/odd number 
    of transpositions} is called an even/odd permutation. \\
    \hh The set of even permutations in $S_n$ forms 
    a subgroup of $S_n$, denoted as $A_n$, 
    is called the alternating group of degree $n$. \\\vv

    \yellow{Permutation $\to$ transportation:} $(132)(5648)=(13)(32)(56)(64)(48)$ (not unique, but only can be either all odd or all even).
	\\\vs{0.3em}
	\yellow{Inverse of permutation:} $\sigma=(132)(5648) \ \Rightarrow \ \sigma^{-1}=(8465)(231)$ 
    (Separate permutations to be \textbf{\red{disjoint}} first. Since $\sigma(a_i) = a_j$ implies $\sigma^{-1}(a_j) = a_i$, we only need to reverse the order of the cyclic pattern).
	\\\vs{0.3em}
	\yellow{Composition:} $(12)(245)(13)(125) = (14532).$ \\
    (Apply the \textbf{\red{right}} permutation first. Demo!).
\end{frame}
\begin{frame}
    \frametitle{Exercise}
    7. True or false:
    \begin{itemize}
        \item Can an abelian group have a non-abelian subgroup?
        \item Can a non-abelian group have an abelian subgroup?
        \item Can a non-abelian group have a non-abelian subgroup?
    \end{itemize}
    \vs{3em}
    %\yellow{Answer: No; Yes; Yes.}
\end{frame}
\begin{frame}
    \frametitle{Exercise}
        8. Prove the following:
        \begin{enumerate}
            \item $S_n$ is non-abelian for $n\geq 3$;~\\
            \item $A_n$ is a subgroup of $S_n$;~\\
            \item $|A_n|=n!/2.$~\\
        \end{enumerate}
\end{frame}

\section{Homomorphism}
\begin{frame}
    \frametitle{Homomorphism}
    \hh  Given groups $\red{G},\blue{G\,'}$, a homomorphism is a map $f : \red{G} \to \blue{G\,'}$ such that for
		$$f\(x\red{\cdot} y\)=f\(x\) \blue{\cdot} f\(y\)$$
    We have:
    \begin{itemize}
        \item $f\(a_1\cdots a_k \)=f\(a_1\)\cdots f\(a_k\)$
        \item $f\,(1_G)=1_{G\,'}$
        \item $f\(a^{-1}\)=\(f\(a\)\)^{-1}$
    \end{itemize}
    \mydef{Compare and Contrast}
    \hh Recall the concept of \textbf{\green{structure preserving}}
    \begin{equation*}
        \begin{aligned}
            &~y \xrightarrow{~~~f~~~}f\(y\) \\
            x\, \cdot &\downarrow      ~~~~~~~~~\downarrow f\(x\)\cdot \\
            x\,\cdot &y\xleftarrow{~~f^{\,-1}~~}f\(x\cdot y\) 
        \end{aligned}        
    \end{equation*}
\end{frame}
\begin{frame}
    \frametitle{Image \& Kernel}
    \hh The \textbf{\green{image}} of a homomorphism $f : G \to G\,'$, often denoted by $\operatorname{im} f$, or $f\(G\)$, is simply the image of as a map of sets:
    $$
    \operatorname{im} f=\left\{x \in G\,^{\prime} \mid x=f\(a\) \text { for some } a \in G\right\}.
    $$
    \\ \hh The \textbf{\green{kernel}} of $f$ , denoted by $\operatorname{ker}f$, is the set of elements of $G$ that are	mapped to the identity in $G\,'$:
    $$
    \operatorname{ker} f=\left\{a \in G \mid f\(a\)=1_{G\,^{\prime}}\right\}.
    $$
    \mydef{Comapre and Contrast}
    \hh Let $U, V$ be real or complex vector spaces and $L \in \mathscr{L}\(U,V\,\)$, 
    then we define the range and kernel of $L$ by:
    \begin{equation*}
        \begin{aligned}
            \ran L&:=\{v\in V :\underset{u\in U}{\exists}~ v=Lu\}\\
            \text{ker } L&:=\{u\in U : Lu=0\}
        \end{aligned}
    \end{equation*} 
\end{frame}
\begin{frame}
    \frametitle{Properties}
    \hh Let $f : G \to G\,'$ be a group homomorphism, and let $a, b \in G$. 
    Let $K$ = $\ker f$. The following are equivalent:
		\begin{enumerate}
			\item $f\(a\)=f\(b\)$
			\item $a^{-1} b \in K$
			\item $b \in a K$
			\item $a K=b K$
		\end{enumerate}
    \vv
    \red{!} A homomorphism $f : G \to G\,'$ is injective 
    if\mbox{f} $\ker f = \{1_G\}$.\\
    \red{!} Isomorphism $G \cong G\,'$ $~~\LRarrow~~$ $f$ is \textbf{bijective}. \\
    \red{!} How to check if a \textbf{\green{homomorphism}} is an \textbf{\green{isomorphism}}:
    \begin{center}
        verify $\ker f= \{1_G\}$ (injection) and $\operatorname{im} f = G\,'$ (surjection)
    \end{center}
\end{frame}
\begin{frame}
    \frametitle{Exercise}
    9. Let a homomorphism $f:G \to G\,'$,
    if $H$ is a subgroup of $G$, prove that $f\(H\)^{-1}$ is a subgroup of $G\, '$.
    $$
    f\,(H\,)^{-1} := \left\{f\,(a)^{-1} \mid a \in H\right\}
    $$
    \pause 
    \\\vs{1em}
    \yellow{Solution:}\\
    \par Let $x,y, a \in H$.
    \begin{enumerate}
    	\item Closure: $f\,(x)^{-1} f\,(y)^{-1} =  f\,(x^{-1})f\,(y^{-1}) = f\(x^{-1}y^{-1}\) = f\((yx)^{-1}\) = f\,(yx)^{-1}.$
    	\item Identity: $1_G \in H, 1_{G\,'} = f\(1_{G}\) = f\(1_{G}\)^{-1} \in f\,(H\,)^{-1}$.
    	\item Inverse: $f\,(a)^{-1} = f\(a^{-1}\) \in f\,(H\,)^{-1}$. 
    \end{enumerate}
\end{frame}
\begin{frame}
    \frametitle{Exercise}
    10. Let $(G,\cdot)$ be a group. Let $g,h \in G$ both have
    order $n$, prove that  $\<g\>\cong\<h\>$.
    \\\vs{2em}
    \yellow{Solution:}\\
    \hh Define $f: \<g\> \to \<h\>$ by $f\,(g)=h$ and for all $0\leq k \leq n, f\(g^k\)=f\,(g)^k$.
    So, $f$ is a well-defined function, and, by definition, $f$ preserves the group
    product. It is clear that the function $f$ sends $1_G \mapsto 1_G$, $g \mapsto h$, $\dots$, $g^{n-1}\mapsto h^{n-1}$,
    and so $f$ is a bijection.
    \\\vv
    (Directly taken from Zach's slides)
\end{frame}
\section{Cosets}
\begin{frame}
    \frametitle{\red{Cosets}}
    \fbox{
    \parbox{0.95\textwidth}{
		\hh Given a group $G$, if $H$ is a subgroup of $G$ and $a \in G$, the notation $aH$ will stand for the set of all products $ah$ with $h \in H$,
		$$
		a H=\{g \in G \mid g=ah \text { for some } h \in H\}
		$$
		This set is called a \textbf{\green{left coset}} of $H$ in $G$.	
	}}
    \\\vv
    \fbox{
    \parbox{0.95\textwidth}{
    \hh The number of left cosets of a subgroup is called the \textbf{\green{index}} of $H$ in $G$. The index is denoted by $[G : H\,]$ (can be infinite, why?).
    \par \hh  All left cosets $aH$ of a subgroup $H$ of a group $G$ have the same \textbf{\green{order}}.
    }
    }
    \begin{itemize}
        \item[-] \blue{Counting formula:} $|G| = |H| \cdot [G:H\,]$.
        \item[-] \blue{Lagrange's Theorem:} Let $H$ be a subgroup of a finite group $G$. 
        The order of $H$ divides the order of $G$.
    \end{itemize}
\end{frame}
\begin{frame}
    \frametitle{Exercise}
    11. Verify Lagrange's Theorem for the subgroup 
    $H = \{0, 3\}$ of $\mathbb{Z}_6$.
    \\\vs{2em}
    \mysol
    The cosets are
    $$0 + H = \{0, 3\}, \quad 1 + H = \{1, 4\}, \quad 2 + H = \{2, 5\}.$$
    
    Notice there are 3 cosets, each containing 2 elements, 
    and that the cosets form a \textbf{partition} of the group.
\end{frame}
\begin{frame}
    \frametitle{An important consequence of Lagrange's Theorem}
    \mydef{Theorem}
    \hh Let $(G,\cdot)$ be a group and let $g \in G$ have order $n$. 
    If there exists $m,k \in \bN\cut\{0\}$ with $n = mk$,
    then the order of $g^m$ is $k$. \\ \vs{0.3em}
    \mydef{Proof.}
    \hh Let $m,k \in \bN\cut\{0\}$ with n = mk. 
    Now, $(g^m)^k=g^{mk}=g^n=1_G=$. If $0<q<k$ is 
    such that $(g^m)^q=1_G$, then $g^{mq}=1_G$. But 
    $mq<mk=n$, 
    which is a contradiction.\\ \vs{0.3em}
    \mydef{Theorem}
    \hh If $(G,\cdot)$ is a finite group with order $n$, 
    then for all $g \in G, g^n = 1_G$.\\ \vs{0.3em}
    \mydef{Proof.}
    \hh Let $(G,\cdot)$ be a finite group with order $n$. 
    Let $g \in G$. We know that the order of $g$ must be finite, 
    so let $k$ be the order of $g$. Now, $k$ must divide $n$,
    so the exists $m \in N$ such that $n = mk$. So 
    $g^n=g^{mk}=(g^k)^m$ $=1_G ^ m=1_G$.
\end{frame}
\begin{frame}
    \frametitle{Exercise}
    12. Prove that for any subgroup $H \leq G$, the
    (left) cosets of $H$ partition the group $G$.
    \\\vs{3em}
    \yellow{Hint:}\\
    \hh We need to show that the union of the left cosets is the whole group, and that
    different cosets do not overlap.
\end{frame}
\begin{frame}
    \frametitle{\red{Normal Subgroup}}
    \fbox{
	\parbox{0.95\textwidth}{
		\hh Given group $G$, and $a, g \in G$, the element $gag^{-1} \in G$ is called the
		\textbf{\green{conjugate}} of a by $g$.
		\par \hh A subgroup $N$ of $G$ is a \textbf{\green{normal subgroup}},
        denoted by $N \unlhd G$, if for all $a \in N$ and $g \in G$, 
        $gag^{-1} \in N$. 
	}}
    \\\vs{1em}
    \blue{Properties:} 
    \begin{itemize}
        \item $f : G \to G\,'$ a homomorphism, then $\ker f \trianglelefteq G$.
        \item Every subgroup of an abelian group is normal.
        \item The \textbf{\green{center}} is always a normal subgroup.
        \item $gH=Hg$ for all $g\in G$ if\mbox{f} $H \trianglelefteq G $. 
        \item $A_n\trianglelefteq S_n$.
    \end{itemize}
    %\red{\textbf{Try your best! Remember them!}}
\end{frame}
\begin{frame}
    \frametitle{Exercise}
    \red{Important result:}\\
    \vv
    13. Show that any subgroup of index 2 in a group
    is a normal subgroup.\\
    \vv
    \mysol
    \hh Denote the subgroup as $H$. Obviously, the left cosets of a subgroup of index 2 are
    $1_H H= H$ and $aH$, where $a \not\in H$; (\red{why?}) the right 
    cosets are $H 1_H=H$ and $Ha$. Since the cosets form a 
    partition of the origin group, and $1_H H = H 1_H =H $, so 
    the remaining is another coset, namely $aH=Ha$. (\blue{left=right}) 
    So $H$ is normal.
    \\\vs{2em}
    \cha{University of zhihu: \url{https://zhuanlan.zhihu.com/p/163548084}}\\
\end{frame}


\section{End}
\begin{frame}
    \frametitle{Reference}

    \begin{itemize}
        \item Examples From Zach's Slides (P196)
        \item Exercises/graphics from 2021-Fall-Ve203 TA Xue Runze
        \item Exercises from 2021-Fall-Ve203 TA Zhao Jiayuan
        \item Yan Shijian, etc. \textit{Basic Number Theory},
        fourth edition. Beijing: Higher Education Press, 2020.5 print.
    \end{itemize}

\end{frame}
\begin{frame}
    \centering
    \Huge{$\mathcal{THANKS}$!}
\end{frame}

\end{document}