\documentclass{beamer}
\newcommand{\myfont}{\rmfamily\normalsize\upshape\mdseries}
\newcommand{\degree}{^\circ}
\title{\sffamily Review VII(Slides 353 - 416)}
\subtitle{\textbf{Counting! Counting!}\\Counting something is never easy}
\institute[UM-SJTU JI]{University of Michigan-Shanghai Jiao Tong University Joint Institute}
\author{HamHam}
\usepackage{graphicx}
\usepackage{picinpar}
\usepackage{indentfirst}
\usepackage{chemformula}
\usepackage{geometry}
\usepackage{subfigure}
\usepackage{appendix}
\usepackage{amsfonts}
\usepackage{enumerate}
\usepackage{float}
\usepackage{geometry}
\usepackage{latexsym}
\usepackage{listings}
\usepackage{multicol,multirow,multido}
\usepackage{tabularx}
\usepackage{ulem}
\usepackage{tikz}
\usepackage{xcolor}
\usepackage{cite}
\usepackage{setspace}
\usepackage{hyperref}
\usepackage{textpos}
\usepackage{booktabs}
\usepackage{diagbox}
\usepackage{listings}
\usepackage{graphics}
\usepackage{upgreek}
\usepackage{JI_MathCourse_Notations}
\usepackage{mathrsfs}


%\usepackage{ctex} %插入中文
%\ctexset{today=old}
\newcommand{\sweat}[1]{\includegraphics[width={#1}\textwidth]{sweat.png}}
\newcommand{\mydef}[1]{\sffamily\blue{#1}\myfont\\} %for define
\newcommand{\mysol}{\yellow{Solution:}\\}
\usetheme[dove]{Boadilla}
\usecolortheme{dolphin}
\useoutertheme{miniframes}
\begin{document}
    \usebackgroundtemplate{\tikz\node[opacity=0.25]{
    \includegraphics[width=\paperwidth,
    height=\paperheight]{hamster.jpg}
    };}
\begin{titlepage}
    \begin{center}
        VE203 - Discrete Mathmatics 
    \end{center}
\end{titlepage}
\myfont
\newcommand{\binomial}[2]{\begin{pmatrix} {#1}\\{#2}	\end{pmatrix}}
\newcommand{\green}[1]{\textcolor[rgb]{0.3,0.6,0}{#1}}
\section{Binomial Coef\mbox{f}icients}
\begin{frame}
    \frametitle{Twelvefold Way (will be provided)} %(This table will be provided in your final exam paper):
    Distribute $k$ balls into $n$ urns ($f : B \to U$, $|B| = k$, $|U| = n$).
    \begin{table}[H]
        \centering
        \resizebox{10.5cm}{!}{
        \begin{tabular}{ccccc}
            \hline
            \begin{tabular}[c]{@{}c@{}}Balls \\ (domain)\end{tabular} & \begin{tabular}[c]{@{}c@{}}Urns\\ (codomain)\end{tabular} & \begin{tabular}[c]{@{}c@{}}unrestricted\\ (any function)\end{tabular} & \begin{tabular}[c]{@{}c@{}}$\leq 1$\\ (injective)\end{tabular} & \begin{tabular}[c]{@{}c@{}}$\geq 1$\\ (surjective)\end{tabular} \\ \hline
             & & & & \\
            labeled & labeled & $n^k$ & $n^{\underline{k}}$ & $n!\left\{\!\!\!\begin{array}{l} k \\ n\end{array}\!\!\!\right\}$ \\ 
             & & & & \\
            unlabeled & labeled & $\begin{pmatrix}\!\!\begin{pmatrix}	n\\k \end{pmatrix}\!\!\!\end{pmatrix}$ & $\begin{pmatrix} n\\k	\end{pmatrix}$ & $\begin{pmatrix}\!\!\begin{pmatrix}	n\\k-n \end{pmatrix}\!\!\!\end{pmatrix}$ \\ 
             & & & & \\
            labeled & unlabeled & $\sum\limits_{i=1}^{n} \left\{\!\!\!\begin{array}{l} k \\ i\end{array}\!\!\!\right\}$ & $\blue{[k \leqslant n]}$ & $\left\{\!\!\!\begin{array}{l} k \\ n\end{array}\!\!\!\right\}$ \\
             & & & & \\
            unlabeled & unlabeled & $\sum\limits_{i=1}^{n} p_i (k)$ & \blue{$[k \leqslant n]$} & $p_n (k)$\\ 
             & & & & \\ \hline
        \end{tabular}
        }
    \end{table}
\end{frame}
\begin{frame}
    \frametitle{A few names...}
    Notations in \red{red} are suggested.
    \begin{itemize}
        \item Permutation: $\red{n^{\underline{k}} = P(n,k)} = P_k^n = \underbrace{n \cdot (n-1) \cdot (n-2) \cdots(n-k+1)}_{k \text { terms }} = \dfrac{n !}{(n-k) !}$, \phantom{Permutation: } ($A_n^k$ in middle school, $n \geqslant k$) \\
		\item Combination: $\red{\begin{pmatrix} n\\k	\end{pmatrix} = C(n,k)} = C_k^n = \dfrac{n !}{k!\, (n-k) !}$, ($n \geqslant k$) 
		\item Multichoosing: $\begin{pmatrix}\!\!\begin{pmatrix}	n\\k \end{pmatrix}\!\!\!\end{pmatrix} = \red{\begin{pmatrix} n+k-1\\k	\end{pmatrix}}$
        ~~($n<k$ is possible!)
        \item Circular permutation: $\dfrac{P(n,k)}{k!}$
    \end{itemize}
\end{frame}
\begin{frame}
    \frametitle{Basic Properties}
    \begin{itemize}
        \item $\begin{pmatrix} n\\k	\end{pmatrix} = \begin{pmatrix} n\\n-k	\end{pmatrix} = \dfrac{n}{k} \begin{pmatrix} n-1\\k-1 \end{pmatrix}$
        \item $\begin{pmatrix} n\\0	\end{pmatrix} = \begin{pmatrix} n\\n	\end{pmatrix} = 1$, $\begin{pmatrix} n\\1	\end{pmatrix} = n$
        \item \red{$\begin{pmatrix} n\\k	\end{pmatrix} = \begin{pmatrix} n-1\\k-1	\end{pmatrix} + \begin{pmatrix} n-1\\k	\end{pmatrix}$}
        \item $(x+y)^{n} = \sum\limits_{k=0}^{n}\begin{pmatrix} n\\k	\end{pmatrix} x^{k} y^{n-k}$, (binomial theorem)
            \begin{itemize}
                \item Commonly used form: $(x+1)^{n} = \sum\limits_{k=0}^{n}\begin{pmatrix} n\\k	\end{pmatrix} x^{k}$
                \item Application: $\sum\limits_{k=0}^{n}\begin{pmatrix} n\\k	\end{pmatrix} = 2^n$, $\sum\limits_{k=0}^{n} (-1)^k \begin{pmatrix} n\\k	\end{pmatrix} = 0$
            \end{itemize}
        \item \red{$\stirling{n}{k}=\stirling{n-1}{k-1}+k \left\{\!\!\!\begin{array}{c} n-1 \\ k\end{array}\!\!\!\right\}$}
    \end{itemize}
\end{frame}
\begin{frame}
    \frametitle{Exercise}
    1.  Prove the following equations:
    \begin{enumerate}
        \item $$\sum\limits_{k=0}^{n} k \begin{pmatrix} n\\k	\end{pmatrix} = n\cdot 2^{n-1}$$
        \item $$\sum\limits_{k=0}^{n} k^2 \begin{pmatrix} n\\k	\end{pmatrix} = n(n+1)\cdot 2^{n-2}$$
    \end{enumerate}
\end{frame}

\begin{frame}
    \frametitle{Exercise}
    \green{We can have something related to modular arithmetic!}
    \\\vv
    2. If $p$ is a prime, prove the following:
    \begin{itemize}
        \item $\binomial{n}{p}\equiv\floor{\frac{n}{p}}\mod{p}$
        \item $\binomial{p}{k}\equiv0\mod{p}$, for $1\leq k\leq p-1$
        \item $\binomial{p-1}{k}\equiv(-1)^k \mod{p}$, for $0\leq k\leq p-1$
        \item $\binomial{p+1}{k}\equiv 0\mod{p}$, for $2\leq k\leq p-1$
    \end{itemize}
\end{frame}
\section{Multichoosing}
\begin{frame}
    \frametitle{Multichoosing}
    \mydef{Definition}\vs{0.3em}
    \fbox{
	\parbox{0.95\textwidth}{
		\par \green{Multiset}: Let $S$ be a multiset, $$S = \{ n_1*a_1, n_2*a_2, \cdots, n_k*a_k \} , \ n = n_1 + n_2 + \cdots + n_k,$$
		which means there are $n_1$ element $a_1$, $n_2$ element $a_2$, $\cdots$, $n_k$ element $a_k$. $n_i$ is multiplicity of $a_i \in S$.
	}
    }
    \\\vs{0.5em}
    \fbox{
    \parbox{0.95\textwidth}{
		\par \blue{Permutation in multiset:} 
		\begin{itemize}
			\item [-] $P(n,n) = \dfrac{n!}{n_1! n_2! \cdots n_k!}$
			\par $\to$Proof: select by steps $$ P(n,n) = 
            \begin{pmatrix} n\\n_1 \end{pmatrix}\begin{pmatrix} n-n_1\\n_2 \end{pmatrix} \cdots \begin{pmatrix} n-n_{1}-n_{2}-\cdots-n_{k-1} \\ n_k \end{pmatrix}.$$
		\end{itemize}
	} 
}
\end{frame}
\begin{frame}
    \frametitle{Multichoosing}
    \centering
    %\resizebox{!}{6cm}{
    \fbox{
        \parbox{0.95\textwidth}{
		\par \blue{Combination in multiset:}
		\begin{itemize}
			\item [-] $\begin{pmatrix}\!\!\begin{pmatrix} k\\r \end{pmatrix}\!\!\!\end{pmatrix} = \begin{pmatrix} k+r-1\\r \end{pmatrix}$ if $r \leqslant \forall n_i$.
			\par $\to$Proof:
			\par The combination $\begin{pmatrix}\!\!\begin{pmatrix} k\\r \end{pmatrix}\!\!\!\end{pmatrix}$ equals to the number of solutions to the indefinite equation $$x_1 + x_2 + \cdots + x_k = r.$$
			\par A solution corresponds to a permutation 
			\begin{equation}
			\label{eq.Pmulti}
			\underbrace{1\cdots1}_{x_1 \text{ times}}0\underbrace{1\cdots1}_{x_2 \text{ times}}0\cdots0\underbrace{1\cdots1}_{x_k \text{ times}}.
			\end{equation}
			\textit{I.e.,} if $k-1$ 0s divide $r$ 1s into $k$, $x_i$ is the number per partition. 
		\end{itemize}
	    } 
    }
\end{frame}
\begin{frame}
    \frametitle{Properties}
    \fbox{
        \parbox{0.95\textwidth}{
            \begin{itemize}
                \item[-]  $\begin{pmatrix}\!\!\begin{pmatrix} k\\r \end{pmatrix}\!\!\!\end{pmatrix} = \begin{pmatrix}\!\!\begin{pmatrix} r+1\\k-1 \end{pmatrix}\!\!\!\end{pmatrix}$ 
                \par $\to$Proof: number of ways to arrange $k-1$ 0s and $r$ 1s = number of ways to arrange $r$ 1s and $k-1$ 0s.
                $$LHS = \begin{pmatrix}\!\!\begin{pmatrix} (k-1)+1\\r \end{pmatrix}\!\!\!\end{pmatrix} = \begin{pmatrix}\!\!\begin{pmatrix} r+1\\k-1 \end{pmatrix}\!\!\!\end{pmatrix} = RHS.$$           
                \item [-] \blue{Multinomial formula:} 
                    $$
			        \left(a_{1}+\cdots+a_{m}\right)^{n}=
                        \sum_{
                            \substack{
                                k_{1}, \ldots, k_{m} \geqslant 0 \\
                                k_{1}+\cdots+k_{m}=n
                            } 
                        }
                    \left(\begin{array}{c}
			        n \\
			        k_{1}, k_{2}, \ldots, k_{m}
			              \end{array}\right) 
                    a_{1}^{k_{1}} a_{2}^{k_{2}} \cdots a_{m}^{k_{m}}
			        $$
            \end{itemize}
	    } 
    }
\end{frame}
\begin{frame}
    \frametitle{Exercise}
    \hh \green{Of course, we can have something more fancy. 
    But...the exam actually prefers the exercise below} 
    \sweat{0.03}
    \sweat{0.03}
    \sweat{0.03}\\\vv
    3. Find the number of \textbf{non-negative} integers 
    solutions of $$x_1+x_2+x_3+x_4=30,$$ 
    such that $3 \leqslant x_i \leqslant 10$ 
    for every $1 \leqslant i \leqslant 4$.\\ \vv
    \pause 
    \yellow{Answer:}\\
    \begin{itemize}
        \item $\dbinomial{5}{10}=\binomial{14}{10}=1001$.
        \item $\(\binomial{14}{10}-
        \(\binomial{12}{2}+\binomial{10}{2}+\cdots
        +\binomial{2}{2}\)\)/2=420$.
    \end{itemize}
\end{frame}
\section{Inclusion-Exclusion}
\begin{frame}
    \frametitle{Inclusion-Exclusion Principle}
    \mydef{Notation}\vs{0.3em}
    \fbox{
	\parbox{0.95\textwidth}{
		\hh Given $I \subset \{1, \cdots , n\}$, 
        let 
		$$
		A_{I}=\bigcap_{i \in I} A_{i},
		$$
		where $A_i \subset X$ for all $i \in I$. 
        \par \hh For example, $A_{\{1,2,4\}} = A_1 \cap A_2 \cap A_4$. In particular, $A_\varnothing = X$.
    }}
    \\\vs{0.3em}
    \mydef{Theorem (Inclusion-Exclusion Principle)}
    \fbox{
    \parbox{0.95\textwidth}{
        Let $A_1, \cdots, A_n$ be subsets of $X$. Then the number of elements of $X$ which lie in none of the subsets $A_i$ is $$
		\sum_{I \subset\{1, \ldots, n\}}(-1)^{|I\,|}\left|A_{I}\right|.
		$$
    }
    }
\end{frame}
\begin{frame}
    \frametitle{Corollary}
	\par Let $A_1, \cdots , A_n$ be a sequence of (not necessarily distinct) sets, then
	$$
	\left|A_{1} \cup A_{2} \cup \cdots \cup A_{n}\right|=
    \sum_{\varnothing \neq I \subset\{1, \ldots, n\}}
    (-1)^{|I\,|+1}\left|A_{I}\right|.
	$$
    \mydef{Special Case}
    \hh When $|I\,|=|J\,|\Rarrow |A_I|=|A_J|$
    $$
	\left|A_{1} \cup A_{2} \cup \cdots \cup A_{n}\right|=
    \sum_{|I\,|=1}
    (-1)^{|I\,|+1}\binom{n}{\left|I\,\right|}\left|A_{I}\right|.
	$$
    \mydef{Derangement}
    $$
    d_{n}=\sum_{i=0}^{n}(-1)^{i}  \binom{n}{i}
    (n-i) !=n ! \sum_{i=0}^{n} \frac{(-1)^{i}}{i !}
    $$
\end{frame}
\begin{frame}
    \frametitle{Exercise}
    4. Find the number of non-negative integers solutions of 
    $$x_1+x_2+x_3+x_4=30$$ 
    such that $3 \leqslant x_i \leqslant 10$ for every 
    $1 \leqslant i \leqslant 4$.
    \\ \vs{3em} \yellow{Answer:} 
    $1330-1084=246$.
\end{frame}
\begin{frame}
    \frametitle{Counting Surjections}
    \fbox{
	\parbox{0.95\textwidth}{
		\par Let $k \geqslant n$. The number of surjections $f : \{1, \cdots, k\} \to \{1, \cdots , n\}$ is given	by$$
		S_{k, n}=\sum_{i=0}^{n-1}(-1)^{i}\begin{pmatrix} n \\ i \end{pmatrix}(n-i)^{k} = n!\left\{\!\!\!\begin{array}{l} k \\ n\end{array}\!\!\!\right\}.
		$$
	}
    }
    \vs{0.5em}
    \begin{block}{Exercise}
    \hh   Which of the following identities are valid for $n=2021$?
        \begin{itemize}
            \item[\green{(A)}]$\sum_{k=0}^n \binom{n}{k} (n-k)^{n-2} (-1)^k = 0$
            \item[(B)]$\sum_{k=0}^n \binom{n}{k} (n-k)^{n-1} (-1)^k = n!$
            \item[(C)]$\sum_{k=0}^n \binom{n}{k} (n-k)^{n}   (-1)^k = (n+1)!$
            \item[\green{(D)}]$\sum_{k=0}^n \binom{n}{k} (n-k)^{n+1} (-1)^k = n\cdot (n+1)! / 2$
        \end{itemize}
    \end{block}
\end{frame}
\section{Calculator Workshop}
\begin{frame}{Matrix Chain Multiplication}
    \red{This is the boring recipe...}
    \begin{enumerate}
        \item we shall create a 2-dimensional array for storing 
        \blue{$m[i,j]$ costs needed to compute $A_{i \cdots j}$}. 
        \item Remember that if we have only one matrix in a sequence, then there is nothing to multiply. 
        It means that \blue{$m[i,j] = 0$, when $i = j$}.
        \item However, when $i < j$, we split the sequence into $A_{i \cdots k}$ and $A_{k+1 \cdots j}$. 
        Then the total cost for $A_{i \cdots j}$ will be based on the following recursive rule
        \blue{(DP Equation)}:
        $$
        m[i, j]= \min\limits_{i \leqslant k<j}\left\{m[i, j]=m[i, k]+m[k+1, j]+p_{i-1} p_{k} p_{j}\right\}
        $$
        \item We \blue{cannot have $i > j$}, 
        thus the part of the table below the main diagonal will be ignored.
    \end{enumerate}
\end{frame}
\begin{frame}
    \frametitle{Exercise}
    \red{If this does occur in the exam, it would be like this one.}
    \\ \vv
    5. Find an optimal parenthesization of a matrix chain multiplication
    whose sequence of dimension is $(20, 2, 30, 12, 8)$.
    \\ \vv 
    \yellow{Answer:} \\
    \hh The optimal parenthesization is $A_1((A_2A_3)A_4)$.
\end{frame}
\begin{frame}
    \frametitle{Linear Recurrence Relation}
    \hh 
    Forgive me that it is really difficult to 
    summerize a \textbf{general recipe}. Let's 
    just look at a concrete example. I think it 
     contains almost everything.\\ \vv 
    \green{Example}\\
    \hh Find the general solution to the following inhomogeneous 
    linear recurrence equation
    \beq
        a_{n+3}=8a_{n+2}-21a_{n+1}+18a_n+ 5^n + (n^2+1) 3^n 
        \label{recurrence}
    \eeq
\end{frame}
\begin{frame}
    \frametitle{Solution}
     Like solving differential equations,
    the \green{general solution} is given by
    $$a_n = a_n ^ {part} + a_n ^ {hom}$$
    For the \green{homogeneous part} we establish the 
    characteristic equation: 
    $$A^3 = 8 A^2 - 21 A + 18 $$
    Press CASIO or by factorization:
    $$(A-2)\cdot (A-3)^2 =0$$
    Hence, the homogeneous solution is given by:
    $$a_n^{hom}= c_1 2^n + c_2 3^n + c_3 n \cdot 3^n$$
\end{frame}
\begin{frame}
    \frametitle{Solution (Cont.)}
    For the first \green{inhomogeneous part} $5^n$, it is easy to see
    $$a_n^{part_1} = c_4 5^n$$
    Plug in back to Eq. \ref{recurrence},  
    $$c_4  = 8c_4  -21 c_4  +18 c_4  +1  
    \Rarrow c_4 = -\frac{1}{4}$$
    For the second \green{inhomogeneous part} $(n^2+1) 3^n$, 
    the root $A=3$ has \red{multiplicity of 2} and $n^2+1$ is a 
    polynomial of \textbf{degree of 2}, so:
    $$a_n^{part_2} =(c_5 + c_6 n + c_7 n^2) \red{n^2} 3^n $$
\end{frame}
\newcommand{\glass}{\includegraphics[width=0.03\textwidth]{glass.png}}
\newcommand{\thinking}{\includegraphics[width=0.03\textwidth]{thinking.png}}
\newcommand{\confound}{\includegraphics[width=0.03\textwidth]{confound.png}}
\newcommand{\sleep}{\includegraphics[width=0.03\textwidth]{sleep.png}}
\newcommand{\bye}{\includegraphics[width=0.03\textwidth]{bye.png}}
\begin{frame}
    \frametitle{Solution (Cont.)}
    Then combining all the terms,
    $$a_n^{gen}= c_1 2^n + c_2 3^n + c_3 n \cdot 3^n 
    +c_5 n^2 \cdot 3^n + c_6 n ^3 \cdot 3^n + c_7 n^4 \cdot 3^ n
    -\frac{1}{4}5^n
    $$
    After suffering from \red{tremendously TEDIOUS calculation}:
    \pause
    \begin{center}
        $a_n^{gen} =$
        \thinking\thinking\thinking\glass\glass\glass\confound\confound\confound\sleep\sleep\sleep
        \bye\bye\bye\sweat{0.03}\sweat{0.03}\sweat{0.03}
    \end{center} 
    \begin{block}{Exam Question}
        Find the general solution to 
            $$ y_{n+2} - 5 y_{n+1} + 6 y_n = n ^2 \cdot 3^n$$
        \yellow{Answer:}
            $$ y_n = c_1 2^n +\(c_2 + \frac{109}{18} n -\frac{7}{6}n^2 + \frac{1}{8} n^3\) 3^n$$
    \end{block}
\end{frame}
\section{Catalan Numbers}
\begin{frame}
    \frametitle{Catalan Numbers}
    \mydef{Definition}
    $$C_n = \frac{1}{n+1} \binom{2n}{n} \quad \quad n \in \bN$$
    \mydef{Recurrence Relation}
    $$C_n=\sum_{k=0}^{n-1} C_k C_{n-1-k}$$
    \green{Fancy Examples:}
    \begin{itemize}
        \item The number of ways to \blue{parenthesize the product of $n + 1$ numbers}
        \item The number of rooted trees with $n$ edges.
        \item The number of \blue{stack permutation} of $n$ elements.
        \item The number of triangulations of a polygon with $n + 2$ sides.
    \end{itemize}
\end{frame}
\begin{frame}
    \frametitle{Explanation for Recurrence Relation}
        \hh We consider the situation that adding bracket to the product $x_0 \cdot x_1\cdot x_2\cdot \cdot x_n$. Note that no matter how we add the bracket, there is one ``$\cdot$" that is outside all the bracket. 
    [e.g. $(x_0\cdot(x_1\cdot x_2))\cdot x_3$, the last operator] We consider this operator to 
    appear between $x_k$ and $x_{k+1}$ , there exists $C_k C_{n-k-1}$ approaches to add the brackets to determine the order of multiplication. The reason is that, there are $C_k$ ways to adding brackets to $x_0 \cdot x_1 \cdots  x_k$, and $C_{n-k-1}$ ways for $x_{k+1} \cdot x_{k+2} \cdots x_n$. Since the last operator could appear between any two among these $n+1$ numbers, so that
    \begin{equation*}
        \begin{aligned}
            C_n &= C_0 C_{n-1} + C_1 C_{n-2}+ \cdot +C_{n-2}C_1+C_{n-1}C_0\\
            &=\sum_{k=0}^{n-1} C_k C_{n-k-1}
        \end{aligned}
    \end{equation*}
    Note that the initial value should be $C_0=1$ and $C_1=1$.
\end{frame}
\begin{frame}
    \frametitle{Generating Function}
    \hh Let $G(x)=\sum_{n=0}^\infty C_n x^n$ be the generating function for $\{C_n\}$. Then
    \begin{equation*}
        \begin{aligned}
            G(x)^2&=\sum_{n=0}^\infty (\sum_{k=0}^n C_k C_{n-k})x^n\\
            &= \sum_{n=1}^\infty (\sum_{k=0}^{n-1} C_k C_{n-1-k})x^{n-1}\\
            &=\sum_{n=1}^\infty C_n x^{n-1}
        \end{aligned}
    \end{equation*}
    \hh Hence, $xG(x)^2=\sum_{n=1}^\infty C_n x^n$, 
    which implies $x G(x)^2 - G(x) +1 =0$. 
    By solving this equation we can get $G(x)=(1\pm\sqrt{1-4x})/(2x)$ . \\ 
    \hh We choose the minus sign because the plus 
    sign would lead to a division by zero.
\end{frame}
\section{*Extra Topic}
\begin{frame}
    \frametitle{Dynamic Programming}
    \hh Dynamic programming (DP) is both a mathematical \blue{optimization method} 
    and a \blue{computer programming method}. 
    The method was developed by \textbf{Richard Bellman} in the 1950s and has 
    found applications in numerous fields.
    \\ \vv 
    \yellow{DP actually occurs in the slides! Recall:}
    \begin{itemize}
        \item Matrix Chain Multiplication
        \item Longest Non-decresing Subsequence
        \item Linear Recurrence Relation
        \item $\dots$
    \end{itemize}
    \green{Common Features}
    \begin{itemize}
        \item same sub-problem
        \item optimal sub-structure
    \end{itemize}
\end{frame}
\begin{frame}
    \frametitle{Where did the name come from?}
    \hh 
    An interesting question is, \blue{``Where did the name, dynamic programming, come from?"} 
    \\
    \vv \hh The 1950s were not good years for mathematical research.
    We had a very interesting gentleman in Washington named \red{Wilson}. 
    He was \red{Secretary of Defense}, and he actually had 
    a pathological fear and hatred of the word ``research". 
    I'm not using the term lightly; I'm using it precisely. 
    His face would suffuse, he would turn red, and he would get violent if people used the term research in his presence. 
    You can imagine how he felt, then, 
    about the term mathematical.
    \\ \hh The RAND Corporation was employed by the Air Force, and the Air Force had Wilson as its boss, essentially. 
    Hence, I felt I had to do something to shield Wilson and the Air Force from the fact that 
    I was really doing mathematics inside the RAND Corporation. 
    
\end{frame}
\begin{frame}
    \frametitle{Where did the name come from?}
    \hh What title, what name, could I choose? 
    In the first place I was interested in \green{planning, in decision making, in thinking}. 
    But planning, is not a good word for various reasons. 
    I decided therefore to use the word ``programming". 
    I wanted to get across the idea that this was \green{dynamic}, 
    this was \green{multistage}, this was \green{time-varying}. 
    I thought, let's kill two birds with one stone. 
    Let's take a word that has an absolutely precise meaning, namely dynamic, in the classical physical sense. 
    It also has a very interesting property as an adjective, and that is it's impossible to use the word dynamic in a pejorative sense. 
    Try thinking of some combination that will possibly give it a pejorative meaning. 
    It's impossible. Thus, I thought dynamic programming was a good name. 
    It was something \red{not even a Congressman could object to}. 
    So I used it as an umbrella for my activities.
\end{frame}
\definecolor{mygreen}{rgb}{0,0.6,0}
\definecolor{mygray}{rgb}{0.5,0.5,0.5}
\definecolor{mymauve}{rgb}{0.58,0,0.82}
\definecolor{mypurple}{rgb}{0.58,0.02,0.82}
\definecolor{myblue}{rgb}{0.1,0.2,0.9}
\definecolor{myorange}{rgb}{0.73,0.38,0.17}
% https://en.wikibooks.org/wiki/LaTeX/Source_Code_Listings
\lstset{%
	language=C++,
	backgroundcolor=\color{white},
	basicstyle=\tiny,
	breakatwhitespace=false,
	breaklines=true,
	captionpos=t,
	commentstyle=\color{mygreen},
	deletekeywords={...},
	escapeinside={\%*}{*)},
	extendedchars=true,
	frame=single,
	keepspaces=true,
	keywordstyle=\color{blue},
	%language=Octave,
	%otherkeywords={*,...},
	numbers=left,
	numbersep=5pt,
	numberstyle=\tiny\color{mygray},
	rulecolor=\color{black},
	showspaces=false,
	showstringspaces=false,
	showtabs=false,
	stepnumber=1,
	stringstyle=\color{mymauve},
	tabsize=4,
	title=\lstname
}

\lstdefinestyle{customcpp}{
	belowcaptionskip=0pt,
	breaklines=true,
	%frame=L,
	%xleftmargin=\parindent,
	language=C++,
	showstringspaces=false,
	basicstyle=\scriptsize\ttfamily,
	keywordstyle=\bfseries\color{mypurple},
	commentstyle=\itshape\color{green!40!black},
	identifierstyle=\color{myblue},
	stringstyle=\color{myorange},
}

\lstset{escapechar=@,style=customcpp}
\begin{frame}
    \frametitle{Backpack Problem}
    \hh Another typical example of DP is the \blue{Backpack Problem}.
    \\\vv
    \mydef{[Problem Statement]}
    \hh We have There are $N$ items and a backpack of capacity $V$.  
    The i-th item has a weight $w_i$ and a value $v_i$. 
     Figure out which items to put into the backpack,
     so that the total weight of these items does \green{not exceed the backpack capacity}, 
     and \green{the total value reaches maximum}.  \\
    \mydef{[DP Equation]}
    \begin{equation*}
        f\,[v\,]= \max \(f\,[v\,],f\,[v-v_i ]+w_i\) 
    \end{equation*}
    \mydef{[Example Question]}
    \hh\url{https://vijos.org/p/1104}
\end{frame}
\begin{frame}[fragile]
    \frametitle{Sample Code}
    \hbox{
    \begin{lstlisting}
    #include <iostream>
    using namespace std;
    inline int max(int a,int b){return a>b?a:b;}
    int main(){ 
        int t,m,w,c;
        cin >> t >> m;
        int f[1001]={0};
        for (int i = 0; i < m; i++){
            cin >> w >> c;
            for (int j = t; j >= w; j--){
                f[j]=max(f[j-w]+c,f[j]);
            }
        }
        int max_value=0;
        for (int i = 0; i <= t; i++){
            max_value=max(max_value,f[i]);
        }
        cout << max_value;
        return 0;
    }\end{lstlisting}
    }
\end{frame}
\begin{frame}
    \frametitle{Reference}
    \begin{itemize}
        \item Example From Horst's Slides FA2020.
        \item Exercises from Ve203-2021-Fall TA Zhao Jiayuan.
        \item Exercises from 2021-Fall-Ve203 Mid\_2 Exam.
        \item Kenneth, H.Rosen. Translated by Xu Liutong etc. \itshape Discrete Mathematics amd Its Applications\myfont, 
                 Eightth Edition, Chinese Abridgement. China Machine Press, 2019 print.
        \item  Richard Bellman, Eye of the Hurricane: An Autobiography (1984, page 159)
        \url{http://en.volupedia.org/wiki/Dynamic\_programming}
        %\item Exercises from 2019-Fall-Ve203 TA Yan Xinyu.
        %\item Contents from 2021-Fall Mid\_2\_RC by Xue Runze.
        %\item Yan Shijian, etc. \textit{Basic Number Theory},
        %fourth edition. Beijing: Higher Education Press, 2020.5 print.
    \end{itemize}
\end{frame}
\end{document}